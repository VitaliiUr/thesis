\documentclass[a4paper, 14pt]{extarticle}
% \usepackage{simplemargins}
\usepackage[T1]{fontenc}
% \usepackage[polish]{babel}
\usepackage[utf8]{inputenc}
\usepackage{geometry}
\geometry{lmargin=1.9cm, rmargin=1.32cm, tmargin=2.54cm,bmargin=1.9cm}

%\usepackage[square]{natbib}
\usepackage{amsmath}
\usepackage{amsfonts}
\usepackage{amssymb}
\usepackage{graphicx}
\usepackage[resetlabels]{multibib}
\newcites{reg}{\large Regular papers}
\newcites{conf}{\large Conference proceedings}

\usepackage{siunitx}
\sisetup{
  range-phrase= \text{ -- },
  range-units=single,
  % per-mode=power
  per-mode=fraction
}

\DeclareSIUnit\mm{\milli\metre}
\DeclareSIUnit\cm{\centi\metre}
\DeclareSIUnit\ms{\milli\second}
\DeclareSIUnit\um{\micro\metre}
\DeclareSIUnit\ns{\nano\second}
\DeclareSIUnit\mev{\mega\electronvolt}
\DeclareSIUnit\gev{\giga\electronvolt}
\DeclareSIUnit\kev{\kilo\electronvolt}
\DeclareSIUnit\clight{\text{\ensuremath{c}}}

\providecommand{\tmp}[1]
{
\textcolor{red}{\textit{#1}}
}

% \renewcommand\refname{\large Publications}

\begin{document}
\pagenumbering{gobble}

\Large
 \begin{center}
    Photodisintegration and Pion Absorption
    % \newline
    
    in Two- and Three-Nucleon Systems:
    % \newline
    
    A Testing Ground for Chiral Models

\hspace{10pt}

% Author names and affiliations
\large
Vitalii Urbanevych \\
% Arthur Author$^1$, Cecilia CoAuthor$^2$ \\

\hspace{10pt}

% \small  
% $^1$) First affiliation\\
% arthur.author@correspondence.email.com\\
% $^2$) Second affiliation

\end{center}

\hspace{10pt}

\normalsize

Ta praca doktorska przedstawia wszechstronne badania nad zastosowaniem potencja\l{}u chiralnego w celu zrozumienia dw\'och rodzaj\'ow proces\'ow z dwu- i tr\'ojnukleonami.
G\l{}\'owny nacisk k\l{}adzie si\k{e} na zastosowanie najnowocześniejszego potencja\l{}u SMS do proces\'ow fotodysocjacji $^2$H, $^3$H i $^3$He,
oraz do absorpcji pion\'ow przez te same j\k{a}dra.
Potencja\l{} SMS jest brany pod uwag\k{e} do pi\k{a}tego rz\k{e}du rozwini\k{e}cia chiralnego, N$^4$LO+, i uzupe\l{}niany przez konsekwentnie zregularyzowan\k{a} si\l{}\k{e} tr\'ojnukleonow\k{a} na poziomie N$^2$LO.

Badanie korzysta z formalizmu przestrzeni p\k{e}dowej, rozwi\k{a}zuj\k{a}c standardowe r\'ownanie Lippmanna-Schwingera w celu uzyskania operatora $t$ oraz odpowiadaj\k{a}cego mu stanu rozpraszania dwunukleonowego. Dla proces\'ow tr\'ojnukleonowych wykorzystuje si\k{e} r\'ownania Faddeeva, aby uwzgl\k{e}dni\'c zar\'owno interakcje stanu pocz\k{a}tkowego, jak i końcowego, umo\.zliwiaj\k{a}c dok\l{}adne zbadanie obserwowanych wielkości, takich jak ca\l{}kowite przekroje czynne, wsp\'o\l{}czynniki schwytania, r\'o\.zniczkowe przekroje czynne i obserwable polaryzacyjne.

G\l{}\'ownym celem jest ocena jakości i zbie\.zności prognoz opartych na potencjale chiralnym SMS, w szczeg\'olności w por\'ownaniu z p\'o\l{}fenomenologicznym potencja\l{}em AV18. Badania eksploruj\k{a} zbie\.znoś\'c prognoz w stosunku do rz\k{e}du chiralnego, ujawniaj\k{a}c, \.ze prognozy oparte na interakcji SMS s\k{a} og\'olnie dobrze zbie\.zne, co wskazuje na satysfakcjonuj\k{a}c\k{a} wydajnoś\'c modelu. Dodatkowo, w tej pracy badam zale\.znoś\'c prognoz od wewn\k{e}trznego parametru odci\k{e}cia $\Lambda$, dostarczaj\k{a}c cennych spostrze\.zeń na temat wra\.zliwości wynik\'ow na ten parametr.

Ponadto, badanie bada rol\k{e} r\'o\.znych sk\l{}adnik\'ow dynamicznych, w tym interakcji stanu końcowego oraz wk\l{}ad\'ow pr\k{a}d\'ow dwunukleonowych. Analiza podkreśla znaczenie tych sk\l{}adnik\'ow w wp\l{}ywaniu na przewidywane wartości, podkreślaj\k{a}c znaczenie w pe\l{}ni sp\'ojnego modelu uwzgl\k{e}dniaj\k{a}cego si\l{}y 2N, 3N oraz pr\k{a}dy jedno-, dwu- i tr\'ojcia\l{}owe.

Podsumowuj\k{a}c, ta praca przyczynia si\k{e} do naszego zrozumienia proces\'ow elektromagnetycznych i silnych w fizyce j\k{a}drowej, demonstruj\k{a}c wysok\k{a} jakoś\'c i zbie\.znoś\'c potencja\l{}u chiralnego SMS.
Dla wszystkich badanych proces\'ow wskazuj\k{e} szczeg\'olnie interesuj\k{a}ce obserwowalne i konfiguracje kinematyczne, w kt\'orych rola poszczeg\'olnych sk\l{}adnik\'ow hamiltonianu jest podkreślona.

Spośr\'od moich wynik\'ow, chcia\l{}bym podkreśli\'c nast\k{e}puj\k{a}ce wnioski:

\begin{enumerate}
\item Interakcja nukleon-nukleon potencja\l{}u chiralnego SMS na wy\.zszych rz\k{e}dach rozwini\k{e}cia chiralnego (powy\.zej N$^2$LO) prowadzi do bardzo stabilnego zachowania prognoz dla proces\'ow fotodysocjacji i absorpcji pion\'ow.
Potwierdza to wcześniejsze ustalenia dla proces\'ow w czystych systemach nukleonowych.
Nie zaobserwowa\l{}em \.zadnych dziwnych lub ostrzegawczych wzorc\'ow dla obserwowalnych, kt\'ore mog\l{}yby by\'c zwi\k{a}zane z niedoskona\l{}ościami w interakcji nukleon-nukleon.
W konsekwencji, wnioskuj\k{e}, \.ze si\l{}a NN jest znana z wystarczaj\k{a}c\k{a} dok\l{}adności\k{a}, aby mo\.zna j\k{a} by\l{}o wykorzysta\'c w badaniach proces\'ow j\k{a}drowych z zewn\k{e}trznymi sondami.
\item Pr\k{a}d dw\'ojcia\l{}owy (2NC) jest bardzo wa\.zny dla badanych proces\'ow elektromagnetycznych i obserwowabli. Nawet uwzgl\k{e}dnienie go za pomoc\k{a} twierdzenia Siegerta pozwala zauwa\.zy\'c znaczne ulepszenia.
\item To samo dotyczy procesu absorpcji pion\'ow. Uwzgl\k{e}dni\l{}em operator absorpcji 2N wiod\k{a}cego rz\k{e}du, a r\'o\.znica mi\k{e}dzy prognozami dowodzi jego znaczenia.
Ponadto, rozbie\.znoś\'c z istniej\k{a}cymi danymi dla ca\l{}kowitych wsp\'o\l{}czynnik\'ow schwytania wymaga bardziej zaawansowanego modelu operatora absorpcji dwu- i tr\'ojcia\l{}owego.
\item Znaczenie 3NF jest mniej oczywiste patrz\k{a}c na moje wyniki.
Na przyk\l{}ad, dla fotodysocjacji $^3$He 3NF sprawia, \.ze zale\.znoś\'c od odci\k{e}cia jest s\l{}absza,
ale r\'o\.znica mi\k{e}dzy prognozami z i bez 3NF nie jest bardzo du\.za.
Dlatego badane tutaj procesy nie s\k{a} najlepszym polem do badania szczeg\'o\l{}\'ow tr\'ojnukleonowej interakcji.
Jedynym wyj\k{a}tkiem jest absorpcja pion\'ow na $^3$H, patrz poni\.zej.
\item Dla $\pi + ^3$H w stanie końcowym znajduj\k{a} si\k{e} trzy neutrony. Jest to jedna z bardzo niewielu takich sytuacji, co pozwala nam bada\'c
dwu- i tr\'ojcia\l{}ow\k{a} si\l{}\k{e} nukleon-nukleon oraz tr\'ojcia\l{}ow\k{a} interakcj\k{e} neutron-neutron-neutron.
Odpowiednie eksperymenty s\k{a} z pewności\k{a} wymagaj\k{a}ce i trudne do przeprowadzenia, ale
mog\k{a} dostarczy\'c cennych informacji, wa\.znych dla zrozumienia i modelowania
interakcji 2N i 3N.
\item Og\'olnie rzecz bior\k{a}c, zauwa\.zam, \.ze niepewnoś\'c wynikaj\k{a}ca z b\l{}\k{e}du uciania przy najwy\.zszym
badanym rz\k{e}dzie chiralnym N$^4$LO+ jest znacznie ni\.zsza ni\.z rozrzut prognoz z r\'o\.znymi
wartościami odci\k{e}cia.
Podobnie jest obserwowane w przypadku, gdzie wzgl\k{e}dny b\l{}\k{a}d uciania N$^4$LO+ przy $S=\SI{10}{\mev}$ wynosi \SI{2.2}{\percent}, podczas gdy
wzgl\k{e}dny rozrzut odci\k{e}cia wynosi \SI{9.0}{\percent}.
\item Badanie r\'o\.zniczkowego przekroju czynnego jest korzystne w por\'ownaniu z ca\l{}kowitymi przekrojami czynnymi, poniewa\.z pozwala zobaczy\'c drobniejsze szczeg\'o\l{}y mechanizm\'ow reakcji i modelu pod wzgl\k{e}dem zbie\.zności i zale\.zności od odci\k{e}cia. Mo\.zna zauwa\.zy\'c przyczyn\k{e} konkretnych r\'o\.znic.
Jest r\'ownie\.z mniej kosztowne obliczeniowo, poniewa\.z ca\l{}kowity przekr\'oj czynny uzyskuje si\k{e} poprzez ca\l{}kowanie przekroju r\'o\.zniczkowego w ca\l{}ym zakresie.
Tak wi\k{e}c, podczas gdy eksperymenty ekskluzywne lub p\'o\l{}ekskluzywne s\k{a} znacznie trudniejsze do przeprowadzenia ni\.z pomiar ca\l{}kowitych przekroj\'ow czynnych lub wsp\'o\l{}czynnik\'ow schwytania, wysi\l{}ki eksperymentalne powinny skupi\'c si\k{e} na takich rodzajach pomiar\'ow w przysz\l{}ości.
\item Moim zdaniem, dla absorpcji pion\'ow by\l{}oby interesuj\k{a}ce posiada\'c dane pomiarowe dla obszaru FSI(nn). Poniewa\.z nie mamy w pe\l{}ni sp\'ojnego pr\k{a}du dw\'ojcia\l{}owego, korzystne by\l{}oby uwzgl\k{e}dnienie danych eksperymentalnych podczas analizy prognoz uzyskanych przez przybli\.zenia lub (w przysz\l{}ości) przez w pe\l{}ni sp\'ojne 2NC.
\item Analizuj\k{a}c wk\l{}ad FSI do przekroju czynnego fotodysocjacji deuteru, zauwa\.zam, \.ze staje si\k{e} on wi\k{e}kszy przy wy\.zszych energiach foton\'ow.
Dlatego wnioskuj\k{e}, \.ze przekr\'oj czynny dla fotodysocjacji deuteru jest
bardziej wra\.zliwy na FSI przy wi\k{e}kszych energiach.
Podobnie, rośnie dla obserwabli asymetrii fotonowej $\Sigma_\gamma$.
Z kolei nie zaobserwowa\l{}em takiej zale\.zności od energii wk\l{}adu FSI do analizuj\k{a}cych mocy.
R\'o\.znica mi\k{e}dzy prognozami falowymi a "Pe\l{}nymi"
nie zmienia si\k{e} znacznie zwi\k{e}kszaj\k{a}c energi\k{e} fotonu z \SI{30}{\mev} do \SI{100}{\mev}.
Dlatego jeśli ktoś chcia\l{}by zbada\'c wk\l{}ad FSI do fotodysocjacji deuteru,
lepiej jest mierzy\'c (lub oblicza\'c) zar\'owno przekr\'oj czynny, jak i asymetri\k{e} fotonow\k{a} przy du\.zych energiach.
\item Nasz pe\l{}ny model dobrze opisuje dane do oko\l{}o $\text{E}\gamma = \sim\SI{70}{\mev}$. Nie jest to wartoś\'c ściśle określona, ale og\'olny trend jest taki, \.ze dla wy\.zszych energii r\'o\.znica mi\k{e}dzy modelem a danymi wzrasta. Na przyk\l{}ad, prognozy dla ca\l{}kowitego przekroju czynnego i analizuj\k{a}cych mocy tensorowych dla fotodysocjacji deuteru \l{}adnie opisuj\k{a} dane eksperymentalne nawet do $E\gamma = \SI{100}{\mev}$, ale asymetria fotonowa zaczyna si\k{e} pogarsza\'c ju\.z przy $E_\gamma = \SI{40}{\mev}$.
\item Jeśli chodzi o absorpcj\k{e} pion\'ow, zakończy\l{}bym, \.ze najbardziej interesuj\k{a}ca jest, ze wzgl\k{e}du na stan końcowy,
reakcja $\pi^- + ^3\text{H} \rightarrow n + n + n$.
Jest to korzystne w por\'ownaniu z absorpcj\k{a} w $^2$H, poniewa\.z system ma wi\k{e}cej stopni swobody i pozwala
na badanie wi\k{e}kszej liczby konfiguracji.
Moje wyniki dla $^3$He i $^3$H pokazuj\k{a} podobn\k{a} wra\.zliwoś\'c na FSI i operator absorpcji 2N. Niemniej jednak, uzupe\l{}niaj\k{a}cy pomiar tych reakcji otworzy\l{}by drog\k{e} do badań zale\.zności izospinowej operatora absorpcji dwu-cia\l{}owego.
\end{enumerate}

\tmp{Taka jest struktura tej pracy. Rozdzia\l{} 1 przedstawia wprowadzenie do ram teoretycznych i motywacj\k{e} badania. Rozdzia\l{} 2 prezentuje formalizm i metodologi\k{e} u\.zywan\k{a} w badaniu. Rozdzia\l{} 3 prezentuje wyniki obliczeń numerycznych dla badanych proces\'ow, a Rozdzia\l{} 4 zawiera podsumowanie i wnioski.
Na koniec, podsumowuj\k{e} w Rozdziale 4.}

\end{document}