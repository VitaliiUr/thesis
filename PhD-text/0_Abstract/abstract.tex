\begin{abstract}

%%%%%%%%%%%%%%%%%%%%%%%%%%%%%%%%%%%%%%%%%%%%%%%%%%%%%
% Enter jaw droping abstract below:                 % 
%%%%%%%%%%%%%%%%%%%%%%%%%%%%%%%%%%%%%%%%%%%%%%%%%%%%%

This Ph.D. thesis presents a comprehensive investigation into the application of the chiral potential to understand two types of processes with two- and three-nucleons. The primary focus is on the application of 
the state-of-the-art \gls{sms} potential to photodisintegration processes of $^2$H, $^3$H, and $^3$He,
as well as to pion absorption by the same nuclei.
The \gls{sms} potential is taken up to the fifth order of chiral expansion, N$^4$LO+, and augmented by the consistently regularized three-nucleon force at N$^2$LO.

The study employs the momentum space formalism, solving the standard Lippmann-Schwinger equation to obtain the $t$-operator and the corresponding two-nucleon scattering state. For three-nucleon processes, the Faddeev equations are utilized to include both initial and final state interactions, allowing a thorough examination of observables such as total cross sections, capture rates, differential cross sections, and polarization observables.

The main goal is to assess the quality and convergence of predictions based on the chiral \gls{sms} potential, particularly in comparison to the semi-phenomenological AV18 potential. The research explores the convergence of predictions with respect to the chiral order, revealing that predictions based on the \gls{sms} interaction are generally well-converged, indicating satisfactory model performance. Additionally, in this work I study the dependence of predictions on the intrinsic cut-off parameter $\Lambda$, providing valuable insights into the sensitivity of results to this parameter.

Furthermore, the study investigates the role of various dynamical components, including final state interactions and two-nucleon current contributions. The analysis highlights the significance of these components in influencing predicted values, emphasizing the importance of a fully consistent model incorporating 2N forces, 3N forces, and one-, two-, and three-body currents.

In conclusion, this work contributes to our understanding of electromagnetic and strong processes in nuclear physics, demonstrating the high quality and convergence of the chiral \gls{sms} potential.
For all studied processes I point out especially interesting observables and kinematical configurations, in which the role of individual components of the Hamiltonian is highlighted.
% The findings provide valuable insights for future developments in theoretical frameworks and emphasize the importance of comprehensive models for accurate predictions in nuclear physics.

This work is organized as follows. Chapter 1 provides an introduction to the theoretical framework and the motivation for the study. Chapter 2 presents the formalism and methodology used in the study. Chapter 3 presents the results of numerical calculations for corresponding processes, and Chapter 4 provides a summary and conclusions.

\tmp{zrobic rozrzerzoną wersję dla rady.}

\end{abstract}
