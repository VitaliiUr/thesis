% \chapter{Introduction}
% \label{introduction}


% \begin{itemize}
    %     \item Chiral review \cite{epelbaum_frontiers}
    
    %     \item Weinberg \cite{WEINBERG1990}\\
    %         Weinberg \cite{WEINBERG1990} suggested using a most general Lagrangian
    %         satisfying spontaneously broken chiral symmetry and other symmetries and
    %         evolving pions together with low-energy nucleons.
    
    %         \item Epelbaum \cite{epelglockle98, epelglockle2000, epelglockle2003}
    %     \item The newest chiral force \cite{reinkrebs2018} 
    
    %     \item Arenhovel \cite{ArenhovelPhotodisint1991}
    %     \begin{itemize}
        %         \item He used different currents (`names, types')
        %         \item Approaches
        %     \end{itemize}
        % \end{itemize}
% \section{Historical overview}
        
% In the second half of XX century physical society faced
% a problem of describing low-energetic nuclear reactions.
% \gls{qcd} is hardly applicable here as it is nonperturbative 
% at low energies what complicates a lot search for the solutions \cite{Machleidt2011}. 


% \printglossary[type=\acronymtype]
% \printnoidxglossary[type=acronym]

\chapter{Introduction}

% \begin{itemize}
%     \item Why we study few nucleon systems
%     \begin{itemize}
%         \item Strong interactions (2N and 3N force investigation; QCD, relativistic effects)
%         \item Electro-magnetic processes (electrons-, photons-induced reactions) (Arenhovel did ...)
%         \item Weak interactions (neutrons)
%     \end{itemize}

%     \item Nuclear forces used in the thesis
%     \begin{itemize}
%         \item AV18
%         \item Chiral (scs, sms; difference between chiral models; regularization problem)
%     \end{itemize}

%     \item Currents used in the thesis (regularization of currents to be done)
    
%     \item Formalism \& numerical methods
%     \begin{itemize}
%         \item Lippman-Schwinger eq
%         \item Schrodinger eq for deuteron; wave functions (sms) for deuteron - figures, binding energy
%         \item Three body: Fadeev eq. for bound (He3, H3) and scattering states
%         \item Siegert theorem ?
%         \item Partial wave decomposition, states ($pq\alpha$), Jakobi momenta;
%         operators in PW decomp. (current); Mathematica for PW
%         \item Theoretical uncertainties: truncation error, cut-off dependency, chiral order dependency
%     \end{itemize}

%     \item Results (\textbf{find everything what I have calculated: all processes and energies} )
%     \begin{itemize}
%         \item H2 photodisintegration
%         \item He3 and H3 photodisintegration
%         \item Pion capture
%     \end{itemize}

%     \item Summary
    
%     \item References
% \end{itemize}

\subsection*{Why we study few nucleon systems}

The study of light nuclei and their reactions has been serving as an easy way to investigate particles in nuclei and
the forces between them for decades. A convenient way to proceed may be to study the interaction of a nucleus with
other nuclei, particles, or electromagnetic probes such as electrons, photons, muons, pions, neutrinos, and hyperons.
In most cases, the study of elastic or inelastic scattering is possible. This can be done either theoretically or by
performing relevant experiments to test if the theory works. It should be taken into account that interactions may be
caused by different forces and therefore described in different ways, including strong, weak, or electromagnetic
interactions. This depends on the type of particle being scattered and the target of the reaction.

In the past many experimental efforts have been undertaken and
experimentalists have been interested in electromagnetic reactions in light nuclei for decades.
There are experimental data from the second half of 20th century (e.g. \cite{Skopik1974}, \cite{Liuexp68}, \cite{Kose1969MeasurementsOT}, \cite{Kamae}) which are still 
useful for comparing theoretical predictions with experimental measurements.    
There are several facilities that provide sources of gamma rays (both low- and high-energy)
and other particles that have been operating for decades and still enable experiments to be conducted.
Other groups and collaborations include TUNL \cite{TUNL}, MAMI \cite{MAMI}, 
NIKHEF \cite{NIKHEF}, HI$\gamma$S \cite{TONCHEV2005170} and others.
To accurately describe nuclear reactions, multiple components must be considered.
The most important are nucleon interactions and nuclear current.

Different forces may act on
the participants.
The strong nuclear force acts inside the nuclei and, among others, bound neutrons 
and protons together. The description of strong interactions is extremely
difficult as it deals not only with the nucleons, but with their constituents: quarks
and gluons. \gls{qcd} is a modern theory
describing strong interactions, but, at the moment,
it is not applicable at low energies ($Q^2 \lesssim 1 GeV^2$).
In our situation various approaches are emerging.
The most advanced are the
chiral effective theory, lattice calculation and others \cite{IOFFE2006232}, \cite{BEANElaticce}, \cite{Machleidt2011}.

Starting the study of three- (and more) nucleon systems, it was found that
the strong 2N force is not enough to describe
the system, and a 3N force was introduced. The first applications of such
a force showed that it brings a
sufficient contribution and cannot be ignored \cite{GLOCKLE1982343}.
The contribution of the 3NF can be examined by
comparing binding energies of light nuclei calculated with
and without this part with respect to experimental values.

For example, the binding energy for $^3$H calculated with the \gls{av18} potential without 3NF gives
$E_b(^3\mathrm{H}) = \SI{-7.628}{\mev}$ \cite{NoggaAV18}. There are different models that might add a 3NF
contribution to \gls{av18} (or other potentials). Using the Tucson-Melbourne (TM) model \cite{Tucson-Melbourne}
results in $E_b(^3\mathrm{H}) = \SI{-8.478}{\mev}$, and Urbana IX \cite{Urbana3NF} 3NF provides us with
$E_b(^3\mathrm{H}) = \SI{-8.484}{\mev}$. Looking at the experimental value $E_b(^3\mathrm{H}) = \SI{-8.482}{\mev}$,
it is clear that the 3NF contribution makes the prediction much closer to the measurement. Nevertheless, the UrbanaIX
3NF in this case was fitted by the experimental value for $^3\mathrm{H}$, so there is no surprise in good agreement.

One can check the binding energy for other atoms, which were not used for fitting. The 2NF binding energy for $^3$He
(calculated with \gls{av18}) is $E_b(^3\mathrm{He}) = \SI{-6.917}{\mev}$. TM contribution makes it
$E_b(^3\mathrm{He}) = \SI{-7.706}{\mev}$, Urbana IX gives $E_b(^3\mathrm{He}) = \SI{-7.739}{\mev}$, while the
experimental value is $E_b(^3\mathrm{He}) = \SI{-7.718}{\mev}$. Once more, we can see the importance of 3NF
contribution on the $\alpha$-particle's ($^4\mathrm{He}$) binding energy: pure \gls{av18} gives
$E_b(^4\mathrm{He}) = \SI{-24.25}{\mev}$, \gls{av18} + TM gives \SI{-28.84}{\mev}, \gls{av18} + Urbana IX gives
\SI{-28.50}{\mev}, and the experimental value is \SI{-28.30}{\mev}.

Whereas the first applications included only early simplified "realistic" 3N potential, the latter
investigations, based on more advanced models, only confirmed this statements \cite{StoksPhysRevC49, AV18Wiringa}.
It was also used to construct four-nucleon (4N) bound state \cite{NoggaPhysRevLett}.

Electromagnetic force appears between charged particles like protons, electrons or pions.
Also, the force is transferred between charged particles and a photon, so 
in photon- and electron- scatterings on the nuclei an electromagnetic
force is necessary component of a description.

Arenh"{o}vel's study \cite{ArenhovelPhotodisint1991} used different theoretical models, including a nonrelativistic potential model,
a relativistic impulse approximation, and a relativistic meson-exchange model, to describe the deuteron photodisintegration process.
These models were used to calculate the differential cross section, which describes the probability
of the process occurring at different scattering angles and photon energies.

The calculated cross sections were then compared to experimental data, and it was
found that the relativistic meson-exchange model provided
the best agreement with the data. This model includes the exchange of virtual mesons
between the interacting particles, which accounts for
the strong and electromagnetic forces between them.

Overall, Arenh"{o}vel's study demonstrated the importance of including both strong
and electromagnetic forces in a description of the deuteron photodisintegration process,
and highlighted the need for accurate theoretical models to interpret experimental data.

% Arenh\"{o}vel \cite{ArenhovelPhotodisint1991} 
% studied electromagnetic process - deuteron photodisintegration,
% applying different approaches and comparing the results with
% experimental data.

The weak force is of great importance in the study of nuclear processes. One of the main roles of the weak force is to mediate beta decay,
which is a process in which a neutron in a nucleus is converted into a proton, emitting an electron and an antineutrino. This process plays
a crucial role in the formation of elements in the universe, as it allows for the conversion of neutron-rich isotopes into more stable,
proton-rich isotopes. Additionally, the weak force plays a role in neutrino interactions with matter, which are of great interest in both
astrophysics and particle physics. In nuclear physics, weak interactions can also play a role in the decay of unstable nuclei, the
production of neutrinos in nuclear reactions, and the scattering of neutrinos off nuclei. The study of weak interactions is therefore an
essential component of the overall understanding of nuclear physics and the behavior of matter on the subatomic scale.



\subsection*{Models of strong interaction used in the thesis}

In order to model the nuclear potential, physicists often use phenomenological
or semi-phenomenological approaches. It allows them to combine
theoretical knowledge about processes and experimental data.

One of such models, which was used in current thesis is the Argonne V18 (AV18) \cite{AV18Wiringa}.
In order to construct \gls{nn} force, authors combine
% analytical electromagnetic and 
one-pion-exchange parts
with phenomenological one and supplement them by electromagnetic corrections.
Free parameters were fitted to
the Nijmegen partial-wave analysis of $pp$ and $np$ data \cite{NijmegenPhysRevC.48.792}. 
Authors showed, that AV18 potential delivers good results
in the description of nucleon scattering data ($\chi ^2/data = 1.08$ for around \num{4000} $pp$ and $np$ scattering datasets) 
as well as deuteron properties (estimated binding energy is \SI{2.2247(35)}{\mev} vs experimental \SI{ 2.224 575(9)}{\mev}).

Weinberg's idea of using chiral symmetry to describe nuclear interactions at low
energies was first introduced in his papers published in 1990 and 1991 \cite{WEINBERG1990,WEINBERG1991}.
In these papers, Weinberg argued that the low-energy dynamics of nucleons
could be described using a chiral Lagrangian, which is the most general
Lagrangian consistent with chiral symmetry and its spontaneous breaking.
This Lagrangian is expressed in terms of nucleon and pion fields,
which are the degrees of freedom that become relevant at low energies.

The chiral Lagrangian is the starting point for the development of
the \gls{ceft}, which has become one of the
most advanced approaches to low-energy nuclear physics.
The use of \gls{ceft} allows for the calculation of nuclear properties
and reactions in a model-independent way, making it possible to
quantify the uncertainties associated with the calculation. One of the
key features of \gls{ceft} is that it allows for the construction of
a nuclear potential, which can then be used to solve the Schr"odinger
equation to obtain bound state properties. The accuracy of the
potential can be systematically improved by including higher-order
terms in the chiral expansion, which leads to a better description of
experimental data.

% In the early 1990-ies Weinberg \cite{WEINBERG1990,WEINBERG1991} introduced 
% an idea of using the most general Lagrangian
% satisfying assumed symmetry principles and in particular
% spontaneously broken chiral symmetry to 
% describe nuclear interactions at low energies.
% This idea together with \gls{eft} of \gls{qcd} 
% led to the development of the \gls{ceft}
% % a Chiral effective field theory ($\chi$EFT)
% which nowadays has become one of the most advanced approach to
% low-energy nuclear physics. While in \gls{ceft} it is possible to study processesor bound states directly,
% it is also possible to describe nuclear potential, which can be next used in quantum equations,
% e.g. Schr\"odinger equation.
 
% For the \gls{eft} it is very important to 
% define a quantity, which powers will determine a perturbation order.
% The basic idea
% of \gls{ceft} is to construct a systematic expansion of the nuclear
% interaction in powers of the small parameter $Q/\Lambda_\mathrm{ch}$,
% where $Q$ is a typical momentum or energy scale of the process under
% consideration and $\Lambda_\mathrm{ch}$ is the chiral symmetry breaking
% scale, which is around \SI{1}{\giga\electronvolt}.
In the \gls{ceft} there are two natural scales: so-called soft scale $Q \sim M_\pi$  -
the mass of pion and the hard scale -
$\Lambda_\chi \sim \SI{0.7}{\gev}$ - chiral symmetry breaking scale.
The ratio between these two scales $Q/\Lambda_\chi$
is being used as an expansion parameter in  \gls{ceft} with power
$\nu$: $\left(Q/\Lambda_\chi\right)^\nu$.
\footnote{Note that exact values of some parameters are still under discussion \cite{Epelbaum2004}. We follow here approach described by E.Epelbaum and collaborators, see e.g. \cite{reinkrebs2018}}

Possibility of deriving nuclear potential is an important feature of \gls{ceft}.
The potential, as occurs in an Lagrangian, is a perturbation expression of the same parameter $Q/\Lambda_\chi$.
Considering so-called irreducible diagrams (which cannot be split
by cutting nucleon lines), Weinberg \cite{WEINBERG1990,WEINBERG1991}
came to the expression for the powers $\nu_W$ of such diagrams

\begin{equation}
    \nu_W = 4 - A - 2C + 2L + \sum_i \Delta_i,
    \label{powers}
\end{equation}
where $i$ specifies a vertex number and

\begin{equation}
    \Delta_i \equiv d_i + \frac{n_i}{2} - 2.
    \label{Delta}
\end{equation}

In \eq{powers}, $C$ is a number of pieces which are connected, $L$ - the number of loops in the graph
and $A$ is a number of nucleons in the diagram.
In \eq{Delta}, $n_i$ is a number of nucleon field operators, $d_i$ - the number of insertions
(or derivatives) of  $M_\pi$.

The further analysis of \eq{powers} revealed some problems which occur 
for particular values of parameters in the equation, namely negative values of $\nu_W$ 
are possible while the order has to take integer values from 0 to infinity.
In order to deal with that, \eq{powers} 
was slightly modified adding $3A - 6$ to it  \cite{Machleidt2011, EPELBAUM2006_PROGRESS}:

\begin{equation}
    \nu = \nu_W + 3A  - 6 = -2 + 2A - 2C + 2L + \sum_i \Delta_i
    \label{powers_corrected}
\end{equation}


In \gls{ceft} the first order, "leading order" (LO, $\nu=0$) is followed 
by next-to-leading order (NLO, $\nu=2$)
\footnote{The $\nu=1$ order is completely vanished due to parity and time-reversal invariance,
so next-to-leading order stands for the second order of expansion.},
 next-to-next-to-leading order (\gls{n2lo}, $\nu=3$) and so on.
 At each chiral order, new interaction diagrams complete the potential.
 At LO there is a diagram which consists of 2 contact terms and the diagram
 implying one-pion exchange. Both diagrams reflect only 2NF as well
 as diagrams at NLO, where more contact terms are introduced together with two-pion 
 exchange topologies. Each subsequent order includes more and more sophisticated diagrams
 describing nucleons interaction
 via multiple  pion exchanges and various contact verteces.
 3NF appears at \gls{n2lo} while 4NF contributions are occuring at \gls{n3lo}.
 This establishes for the first time a systematic
way to include all the forces from the simplest diagrams at LO and gradually
adding more and more terms. 
It is also beneficial in the way that 
one can obtain results using chiral potential at different
orders and track which one gives larger or smaller contribution to the final prediction.
The highest order for which there is a derived term in potential
is \gls{n4lo} at the moment. Nevertheless leading F-wave contact interactions from N$^5$LO are included in the \gls{n4lo+} potential,
which is currently regarded as a best possible potential within the model.
The progression of the chiral orders is reflected in a $\chi^2/datum$.
Leading order has only $\chi^2/datum = 73$ (with neutron-proton data with $E_{lab} = 0-100 \unit{\mev}$).
Each subsequent order has better and better results: NLO has $\chi^2/datum = 2.2$, \gls{n2lo} - $\chi^2/datum = 2.2$
and the final \gls{n4lo+} has $\chi^2/datum = 1.08$ \cite{reinkrebs2018}.
Similar progression is observed for larger energies, e.g for $E_{lab} = 0-300 \unit{\mev}$)
$\chi^2/datum$ is 75, 14, 4.2, 2.01, 1.16 and 1.06 at LO, NLO, \gls{n2lo}, \gls{n3lo}, \gls{n4lo} and \gls{n4lo+} respectively.
The proton-proton data description has similar trend, so $\chi^2/datum$ is 1380, 91, 41, 3.43, 1.67, 1.00 
for the same energy bin and chiral orders. At \gls{n4lo+} $\chi^2/datum$ for proton-proton data
stands similar value (close to 1) as for neutron-proton, but the convergence comes a bit later and 
leading order has way worse description.
In my work I will use chiral potentials from LO to \gls{n4lo+}.

\begin{figure}[h]
    \begin{center}
    \includegraphics[width=0.6\textwidth]{Figures/chiral.png}
    \end{center}
    \caption{\tmp{Make own diagrams, e.g. with JaxoDraw or PyFeyn \cite{Entem2017}}}
    \label{proton_rad}
\end{figure}

The Argonne V18 potential \cite{AV18Wiringa}, mentioned earlier, has 18 free parameters,
while \gls{ceft} NN potential at \gls{n4lo} \cite{Machleidt2011} has only three low-energy constants (LECs) fitted to deuteron properties.
The reduction of the number of free parameters of the \gls{ceft}-based potentials
has not only a theoretical but also a practical advantage in the studies of the many-body nuclear systems.

The general scheme outlined above was developed mainly by the Bochum-Bonn and Moscow-Idaho groups.
Both groups have similar approaches and were independently and almost simultaneously
developing own chiral potentials. In 1998 Epelbaum and colleagues from Bochum-Bonn group 
presented a first version of their \gls{nn} chiral potential \cite{EPELBAOUM1998107, epelbaum2000two}.
Developing a more and more sophisticated versions with higher chiral orders, authors presented
\gls{n3lo} potential in 2005 \cite{epelbaum2005two} which included a 3NF contributions.
Authors were further developng their model, taking into account more Feynmann diagrams
coming to a higher chiral orders. Another problem Bochum-Bonn goup faced with was a potential regularization. An important point was when authors started using a semi-local regularisation 
in the coordinate space (\gls{scs} potential) \cite{Epelbaum2014SCS} and later similar regularisation, but in the momentum space, resulting in a most advanced chiral potential at the moment up to  
\gls{n4lo+} chiral order \cite{reinkrebs2018} (the \gls{sms} potential).

On the other side of the planet, in Idaho, Machleidt and his group from Moscow were also developing 
a chiral potential. Their results from 2003 \cite{Entem2003}, following with later investigations \cite{Machleidt2005, Machleidt2010, Entem2017} were introducing very similar to Bochum-Bonn
model with minor technical differences.

There are a number of another approaches within \gls{ceft} utilized.
The group of Piarulli is using quite similar approach, including
the same chiral potentials with minor differences \cite{Piarulli2012,Piarulli2015}.

There are several other approaches within the framework of \gls{ceft}
that have been utilized in nuclear physics.
One of them is the work of Ekström et al. who developed a chiral
effective theory for nuclear forces and currents that includes both
nucleon and pion degrees of freedom \cite{?}. Their approach
is based on a power counting scheme that separates long-range and
short-range contributions and allows for systematic improvements in the
calculations. Another approach is the pionless effective field
theory, which neglects the pion degrees of freedom and
focuses on the interactions between nucleons only \cite{?}.

Another important approach is the lattice effective field theory (LEFT), which
is based on lattice QCD simulations of the strong
interaction. Meissner and collaborators have developed a chiral effective
theory for nuclear forces based on the results of lattice QCD
calculations \cite{?}. This approach has the advantage of being able
to calculate the nuclear force directly from first
principles, without the need for phenomenological input. However, it is limited
to small systems and low energies due to the computational
resources required for lattice QCD calculations.


{\color{red} Machleidt, Ekstr\"om, pion-less EFT, Lattice EFT(Mesissner), Girlanda, Piarulli}


Technically the chiral potential may be derived both in coordinate and momentum spaces.
Nevertheless in both cases it requires regularization which cuts 
low coordinate values in order to avoid infinities 
(or high momentum values - in momentum space). 
The \gls{sms} potential is being regularized semilocally. 
It means that  in its pion propagators both local and nonlocal regularizations
are being applied for different parts of the potential.
In \cite{Entem2003, epelbaum2005two} and later in \cite{Entem2017, Epelbaum2014SCS} the non-local regularisator was applied to both short- and long-range parts of the potential. 
This regularization is applied directly to the potential matrix elements 
in the coordinate space:

\begin{equation}
    V_\pi(\vec{r}) \rightarrow V_{\pi,R} (\vec{r}) = V_\pi (\vec{r}) \left(1 - \exp(-r^2/R^2 )\right),
    \label{scs_regulator} 
\end{equation}
where the cutoff R was chosen in the range of R = 0.8, ..., 1.2 \unit{fm}.

In the current work the potential from \cite{reinkrebs2018} is mainly regarded. In this motential,
regularization is being applied in the momentum space:

\begin{equation}
    V_\pi(\pvec{p}, \vec{p}) \rightarrow V_{\Lambda} (\pvec{p}, \vec{p}) = 
    V_\pi (\pvec{p}, \vec{p}) 
    \exp\left[-(p^\prime/\Lambda)^{2n} -(p/\Lambda)^{2n} \right],
    \label{sms_regulator} 
\end{equation}
with $\Lambda = \frac{2}{R}$ and $n$ being adjusted with respect to the considered chiral order.

The other part of regularization, local, is applied to the propagator operator, during the derivation of potential. Namely, the Gaussian form factor $F(\vec{l}^2)$ is being used
:

\begin{equation}
    \int_{-\infty}^{\infty} \frac{\rho(\mu^2)}{\vec{l}^2 + \mu^2} d\mu^2 \rightarrow 
    \frac{F(\vec{l}^2)}{\vec{l}^2 + \mu^2}
\end{equation}
with

\begin{equation}
    F(\vec{l}^2) = e^{-\frac{\vec{l}^2 + M_\pi^2}{\Lambda^2}},
    \label{regulator}
\end{equation}
where $M_\pi$ is an effective pion mass and $\Lambda$ - is a cutoff parameter and $l$ is a four-momentum of the exchanged pion.
The form factor from \eq{regulator}, being used together with Feynman propagator,
ensures that long-range part of the forces has no singularities. 

The cut-off parameter $\Lambda$ is not fixed and usually calculations
are being performed with different values. The comparison
of such results may reveal stronger or weaker dependance and in perfect
case, which is expected at $\nu >> 1$, one will come up with such a potential, were the cut will
not affect results at all. One of aims of my thesis is to test how big cut-off dependency
of predictions is
observed for currently available forces. In \fig{potential_cutoff} 
I show values of the matrix elements for 2N potential $\matrixel{\vec{p}}{V}{\pvec{p}}$
as a function on the momentum $|\vec{p}|$ with fixed value $|\pvec{p}|$=\SI{0.054}{fm^{-1}}.



\begin{figure}[htb]
    \begin{center}
    \includegraphics[width=0.95\textwidth]{PlotData/Deuteron/WAVEFUNC/potential_pp1.798.pdf}
    \end{center}
    \caption{Potential components in case of coupled partial waves $3S_1 - 3D_1$ as a function on the momentum $p$ with fixed
    value of the momentum $p'=\SI{1.798}{fm^{-1}}$.
    }
    \label{potential_cutoff}
\end{figure}



The potential may be transformed from coordinate to momentum space (or vice versa),
but it is important at which frame the regularization was performed
and what was a regularization function. That's why there are different 
versions of chiral potential. One is the \gls{scs} potential \cite{Epelbaum2014SCS}
and another one is similar, but with regularization applied in momentum space (\gls{sms} potential) \cite{reinkrebs2018}.

YEt another regularisation function  is used by R.~Machleidt, D.R.~Entem and A.~Nogga, 
when regulating matrix elements of the potential in momentum space with non-local regulator only.

\subsection*{Currents}
 
There are several types of nuclear currents used for studying scattering processes. One of the most common types is the one-body current, which describes the motion of individual nucleons within the nucleus. This type of current can be further divided into two categories: the convection current and the spin current. The convection current is associated with the motion of the center of mass of the nucleus, while the spin current is associated with the intrinsic spin of the nucleons.

Another important type of nuclear current is the two-body current, which describes the interaction between two nucleons within the nucleus. The two-body current is typically associated with meson exchange between the nucleons, and it plays a crucial role in scattering processes at low energies.

A third type of nuclear current is the three-body current, which describes the interaction between three nucleons within the nucleus. This type of current is particularly important for studying scattering processes involving light nuclei, such as helium-3 or tritium.

The use of nuclear currents in scattering experiments is essential for understanding the structure of the nucleus and the interactions between its constituent nucleons. Advances in theoretical and experimental techniques have allowed for more precise measurements of these currents and have provided insights into the fundamental properties of the nucleus.

