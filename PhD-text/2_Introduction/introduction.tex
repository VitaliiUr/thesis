\chapter{Introduction}
\label{introduction}


% \begin{itemize}
    %     \item Chiral review \cite{epelbaum_frontiers}
    
    %     \item Weinberg \cite{WEINBERG1990}\\
    %         Weinberg \cite{WEINBERG1990} suggested using a most general Lagrangian
    %         satisfying spontaneously broken chiral symmetry and other symmetries and
    %         evolving pions together with low-energy nucleons.
    
    %         \item Epelbaum \cite{epelglockle98, epelglockle2000, epelglockle2003}
    %     \item The newest chiral force \cite{reinkrebs2018} 
    
    %     \item Arenhovel \cite{ArenhovelPhotodisint1991}
    %     \begin{itemize}
        %         \item He used different currents (`names, types')
        %         \item Approaches
        %     \end{itemize}
        % \end{itemize}
% \section{Historical overview}
        
In the second half of XX century physical society faced
a problem of describing low-energetic nuclear reactions.
QCD is hardly applicable here as it is nonperturbative 
at low energies what complicates a lot search for the solutions \cite{Machleidt2011}. 

In the early 1990-ies Weinberg \cite{WEINBERG1990,WEINBERG1991} introduced 
an idea of using a most general Lagrangian
satisfying assumed symmetry principles and in particular
spontaneously broken chiral symmetry to 
describe nuclear interactions at low energies.
This idea together with effective field theory(EFT) of QCD 
led to the development of 
a Chiral effective field theory ($\chi$EFT)
which nowadays has become one of the most advanced approach to
describing nuclear reactions at low energies.
 
For the effective field theory (EFT) it is very important to 
define a quantity, which powers will determine a perturbation order.
In the $\chi$EFT there are two natural scales: so-called soft scale -
the mass of Pion $Q \sim M_\pi$ and hard scale -
$\Lambda_\chi \sim 1~GeV$ (chiral symmetry breaking scale).
The ratio between these two scales $(Q/\Lambda_\chi)^\nu$
is being used as an expansion parameter in  $\chi$EFT with power
$\nu$.

Considering so-called irreducible (the diagrams that cannot be split
by cutting nucleon lines), Weinberg \cite{WEINBERG1990,WEINBERG1991}
came to the identity for the powers of such diagrams\cite{Machleidt2011}:

\begin{equation}
    \nu_W = 4 - A - 2C + 2L + \sum_i \Delta_i,
    \label{powers}
\end{equation}
where

\begin{equation}
    \Delta_i \equiv d_i + \frac{n_i}{2} - 2
    \label{Delta}
\end{equation}

In \ref*{powers}, $C$ is a number of pieces which are connected, $L$ - the number of loops in the graph.
In \ref*{Delta}, $n_i$ is a number of nucleon field operators, $d_i$ - the number of insertions
(or derivatives) of  $M_\pi$.