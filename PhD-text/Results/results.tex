\chapter{Results}\label{chap:results}

    In this chapter I will show results of my calculations. I start from the deuteron photodisintegraion process
    (Section~\ref{sec:de_results}),
    presenting predictions and discussing the cross (subsection~\ref{sec:cross_results}) 
    and polarization observables  (subsection~\ref{sec:polarisation_results}).Next, in 
    Section~\ref{sec:hel_results} I will present my predictions of the observables
    for $^3$He photodisintegration.
    \tmp{...}
    Finaly in Section~\ref{sec:pion_results} I will discuss results of the calculations
    for pion absorption from the lowest atomic orbital.

\section{Deuteron photodisintegraion}
\label{sec:de_results}
    \subsection{Cross section}
    \label{sec:cross_results}

    In this section I will show the results of my calculation starting from the
    deuteron photodisintegration process. One of the most
    studying observable is obviously the cross section. There is
    a number of papers which present 
    measurement results for both differential and total cross section
    \cite{BOSMAN1979,ARENDS1984,Skopik1974, Moreh1989, Birenbaum1985, Bernabei1986, rachek2007,Ying_Experiment_Deut, DeSanctis_Experiment_Deut} 
    and it seems interesting to compare 
    our predictions with experimental results.

    In Fig.~\ref{TOTAL_CROSS_small} and Fig.~\ref{TOTAL_CROSS} 
    I present predictions for the
    total cross section $\sigma_{tot}~[\mu\text{b}]$ which I obtained
    using the chiral \gls{sms} potential at the \gls{n4lo+} order and with 
    the cutoff parameter $\Lambda=\SI{450}{\mev}$.
    From Fig.~\ref{TOTAL_CROSS_small}, we see that at low photon energies
    (below 50 MeV) predictions which include 2N contributions
    to the electromagnetic current
    via the Siegert approach, describe experimental results quite well.
    We can suppose that the difference with experimental data may come from 
    the statistical uncertainty of  the data itself, as my predictions
    are often in between the data from different experiments.
    Moreover, even at such low energies the 1N current is clearly not enough
    to describe cross section - dashed pink line has much lower values and
    the difference becomes even larger with increasing photon energies.
    At 5~MeV the difference between 1N predictions
    and 1N + Siegert is 297.54~$\mu$b (10.8\%), increasing energy to 10 MeV
    it is 304.28~$\mu$b (20.4\%) and at \SI{20}{\mev} it is 229.50~$\mu$b (39.2\%).
    % Even with energy increasing from 5~MeV to \SI{20}{\mev} -
    % the difference between predictions has changed from 10.8\% to 39.2\% and
    From \fig{TOTAL_CROSS_small} we see that the gap between these predictions
     continues increasing even more with larger energies.
    This tells us that Siegert approach works quite well as no additional 2N contributions are taken into account.
    The cross section predictions corrected via Siegert 2N contribution  reproduce 
    experimental data to some extend.

    Here and later the relative difference between set of predictions ($x_1$, $x_2$, ..., $x_N$) is calculated
    using the formula:

    \begin{equation}
        \Delta = \frac{\max(x) - \min(x)}{\frac{1}{N}\sum_{i=1}^N x_i} \cdot 100\%,
        \label{eq:relative_diff}
    \end{equation}
    so in the specific case of comparison the date with 1N current and "1N + Siegert" we
    calculate relative difference as 
    $\Delta = \frac{|\sigma^{1N+Siegert} - \sigma^{1N}|}{0.5(\sigma^{1N+Siegert} + \sigma^{1N})}$.

    While my main goal is to describe deuteron photodisintegration at low energies, 
    where predictions seem to be well describing experimental data at
    $\text{E}_\gamma \lesssim \SI{50}{\mev}$, it is also interesting to check how 
    present theory works at higher energies.
    At the higher energies (above $\text{E}_\gamma=\SI{50}{\mev}$, Fig.~\ref{TOTAL_CROSS})
    we can notice that the discrepancy with experimental data is not only 
    quantitative, but also qualitative.  
    This starts already above $\text{E}_\gamma=\SI{50}{\mev}$ and is especially pronounced at peak around \SI{300}{\mev}
    seen in the experimental data from \cite{Bernabei1986} which is not
    reflected in predictions. The reason of such discrepancy 
    is most likely coming from the relativistic effects
    which we do not take into account within this work.
    It is also confirmed by the calculations in \cite{ArenhovelPhotodisint1991}
    where authors discuss different potentials applying to the deuteron photodisintegration.
    Despite authors use a much simpler potentials that used in this thesis, their predictions obtained with including
    some relativistic effects show, that such a peak appears in predictions. 
    
    
     The higher energies region is presented in order
    to investigate how far the predictions are from experimental results and 
    what should be improved in the future (e.g. include relativistic part). 
    
    \begin{figure}[h]
        \begin{center}
        \includegraphics[width=0.75\textwidth]{Figures_De/TOTAL_CROSSSECTION_SMALL_REGION.pdf}
        \end{center}
        \caption{Total cross section $\sigma_{tot}$ for the deuteron photodisintegration process
        as a function of the photon energy E$_\gamma$.
        Solid blue line presents results obtained with SN+Siegert 
        and dashed pink line - predictions based on the SN current.
        In both cases the \gls{sms} \gls{n4lo+} $\Lambda=\SI{450}{\mev}$ force is used.
        The experimental data are from \cite{Bernabei1986} (black filled circles),
        \cite{BOSMAN1979} (empty circles),
        % \cite{ARENDS1984} (squares),
        \cite{Skopik1974} (triangles),
        \cite{Moreh1989} (bold cross "X") and
        \cite{Birenbaum1985} (diamonds).
        }
        \label{TOTAL_CROSS_small}
    \end{figure}

    
    \begin{figure}[htb!]
        \begin{center}
        \includegraphics[width=0.75\textwidth]{Figures_De/TOTAL_CROSSSECTION.pdf}
        \end{center}
        \caption{The same as in \fig{TOTAL_CROSS} but for the energy range 2.5 - 400 MeV.
        The experimental data are the same plus data above $\text{E}_\gamma=\SI{200}{\mev}$ from
        \cite{ARENDS1984} (crosses).
        }
        \label{TOTAL_CROSS}
    \end{figure}

    \begin{figure}[htb!]
        \begin{center}
            \includegraphics[width=0.95\textwidth]{Figures_De/TOTAL_CROSSSECTION_Truncation.pdf}
        \end{center}
        \caption{Total cross section of the deuteron photonisintegration
        process as a dependence on the chiral order for three photon energy E$_\gamma$ values: \SIlist[list-units = single]{30;100;140}{\mev}.
        Error bands show an estimated truncation error at each order.}
        \label{Trunc_100}
    \end{figure}
    
    In Fig. \ref{Trunc_100} I present the
    total cross-section for the deuteron photodisintegration 
    at three photon energy values: \SIlist[list-units = single]{30;100;140}{\mev} as a function of the chiral order.
    Error bands show truncation errors calculated using Eq.~\ref{trunc2}~-~\ref{trunc5}.
    One can see that truncation errors are being reduced with each consecutive chiral order. 
    At LO it is the biggest: \SI{29.46}{\percent} at $\text{E}_\gamma = \SI{30}{\mev}$,
    \SI{29.46}{\percent} at $\text{E}_\gamma = \SI{100}{\mev}$ and
    \SI{41.82}{\percent} at $\text{E}_\gamma = \SI{140}{\mev}$.
    At N4LO+ it is hardly visible at presented scale and amounts up to 
    \SI{1.3}{\percent} for each energy.
    For each energy the prediction is within the uncertainty range of lower orders.
    We see that at lower energy $\sigma_{tot}$ already at NLO reaches valve which remains
    practically unchanged at higher orders.
    Contrary, at two higher energies, contributions from higher orders are necessary to obtain 
    stable predictions 

        
    Figures \ref{Diff_cross_order_pw} and \ref{Diff_cross_err} show my predictions 
    for the differential cross section
    $\frac{d\sigma}{d\Omega}$.
    In both figures the top, middle and bottom row shows predictions at 
    $\text{E}_\gamma = \SIlist[list-units = single]{30;100;140}{\mev}$, respectively.
    with 2N current's contributions taken into account via Siegert theorem.
    If not stated otherwise I use th \gls{sms} \gls{n4lo+} potential.
    % They all are organized in a similar way: the left panel
    % presents predictions obtained using \gls*{sms} potential at different chiral orders (from LO to \gls{n4lo+})
    % with cutoff parameter $L = \SI{450}{\mev}$,
    % the middle panel includes the truncation error's bands (described in Sec. \ref{sec:deu\text{T}_bound})
    % for each chiral order starting
    % from NLO. And the right panel shows predictions obtained with different values of the
    % cutoff parameter at the chiral order \gls{n4lo+}.
    The left column of \fig{Diff_cross_order_pw} shows the predictions obtained at 
    different chiral orders (from LO to \gls{n4lo+}) and with $\Lambda=\SI{450}{\mev}$.
    Looking at the best predictions (\gls{n4lo+}, $\Lambda=$~\SI{450}{\mev}) for each
    energy, I
    conclude that the higher photon energy is, the larger is 
    difference between the theoretical predictions and experimental 
    data. At $\text{E}_\gamma = \SI{30}{\mev}$ (top panel) my predictions
    almost perfectly match the data and the difference is almost always
    within the experimental uncertainties. 
    Moving to  $\text{E}_\gamma=\SI{100}{\mev}$ (middle row)
    the description of the data is deteriorating: theoretical
    predictions still match the data qualitatively, but
    the gap for proton emission angle $\theta_p$ in range ($\ang{60} < \theta_p < \ang{130}$) 
    is up to \SI{32}{\percent} (of the predicted value) and relative difference is up to
    \SI{7}{\percent} (calculated with \eq{eq:relative_diff}).
    At the highest energy(bottom figure), it is even hard to say about 
    good qualitative description: the general trend of the
    angular dependence is presented, but the predictions are 
    far from the experimental points.
    We observe improvements introduced by each subsequent chiral order, but 
    stabilization shows that some ingredients are missing.

    Obtained results at each energy confirm the convergence 
    of the predictions with respect to the chiral order.
    We see that the cross section at LO is far from both experimental 
    data and the best potential's predictions (\gls{n4lo+}) and
    the higher photon energy, the larger is this
    difference. With each subsequent chiral order, the 
    curves are more closer to each other and the difference
    between \gls{n4lo} and \gls{n4lo+} is hardly visible at scale used in \fig{Trunc_100}.
    The relative difference between these two predictions at $\text{E}_\gamma=\SI{30}{\mev}$ around the point of maximum 
    ($\theta_p = \ang{80}$) is \SI{0.05}{\percent} which is \SI{0.02}{\micro \barn \per \steradian};
    at \SI{100}{\mev} and $\theta_p = \ang{107}$ it is \SI{0.79}{\percent} (\SI{0.025}{\micro \barn \per \steradian});
    and at \SI{140}{\mev} (same angle) it is \SI{1.8}{\percent} (\SI{0.043}{\micro \barn \per \steradian}).
    Having such a small differences between predictions from two highest chiral orders,
    I can conclude that predictions are converged and 
    using NN potential at subsequent chiral orders would rather not bring large contribution 
    to the cross section values. 
    The difference with experimental data is rather systematic 
    and is independent on the chiral order. 
    The relative difference between experimental data and predictions obtained with $N^4LO+$ and $\Lambda=$~\SI{450}{\mev} at \SI{30}{\mev} is less then \SI{13}{\percent}
    and absulute difference is < \SI{3.07}{\micro \barn \per \steradian}.
    At \SI{100}{\mev} descripancy is larger and relative difference reaches 46\% with absolute difference up to \SI{1.39}{\micro \barn \per \steradian}.
    Coming to \SI{140}{\mev} the relative difference 
    increases up to \SI{48.6}{\percent} and absolute - \SI{1.93}{\micro \barn \per \steradian}.
    What may be helpful
    for a better data description is a 2N current 
    and relativistic correction, mentioned earlier.

    Predictions obtained with the \gls{av18} potential (dashed-dotted purple line In the \fig{Diff_cross_order_pw} left)
    are very similar to these from the \gls{sms} force at lower energies (relative difference at $\text{E}_\gamma=\SI{30}{\mev}$ is \SI{0.06}{\percent}
    at the point of maximum - $\theta_p = \ang{80}$) and with increasing energy to \SI{140}{\mev}
    it growes to \SI{3.1}{\percent} at the same angle. 
    It can be connected with our potential's quality loss, but \gls{av18} can
    be struggling with high energies as well. Then other 
    components of Hamiltonian become important at that energies.

    
    In the right column of the \fig{Diff_cross_order_pw} I compare predictions based on various assumptions on nuclear current and dynamical mechanism. I again use \gls{sms} N$4$LO+, $\Lambda=\SI{450}{\mev}$ force.
    At the lowest energy predictions with plane-wave component only (without rescattering part)
    has relatively small deviation from the full predictions, but the difference increases at larger energies.
    With $\text{E}_\gamma=\SI{30}{\mev}$ the relative difference is \SI{10}{\percent} (\SI{4.03}{\micro \barn \per \steradian})
    at $\theta_p = \ang{80}$. Difference at \SI{100}{\mev} and the same angle is \SI{4}{\percent} (\SI{0.21}{\micro \barn \per \steradian})
    and at \SI{140}{\mev} it is \SI{7}{\percent} (\SI{0.21}{\micro \barn \per \steradian}).
    In contrast, predictions without two-body current component (1NC) have much larger gap with full prediction:
    the difference is \SI{46.5}{\percent} (\SI{13.67}{\micro \barn \per \steradian}) at \SI{30}{\mev},
    \SI{78.6}{\percent} (\SI{2.88}{\micro \barn \per \steradian}) at \SI{100}{\mev} and
    \SI{77.8}{\percent} (\SI{1.68}{\micro \barn \per \steradian}) at \SI{140}{\mev}
    at the same $\theta_p=\ang{80}$.
    Obviously 2NC contributions are extremely important in this case, the difference connected with 2NC
    contributions is much higher then theoretical errors or even rescattering contribution.
    For other scattering angles, especially $\theta_p=\ang{0}$ or $\theta_p=\ang{180}$,
    the role of two-body current or FSI is even more pronound.


    The \fig{Diff_cross_err} (left)
    presents theoretical truncation uncertainties.
    That confirms our expectations
    that for the regarded photo reaction chiral order
    \gls{n4lo+} is able to produce converged predictions: 
    the black band is hardly visible for the $\text{E}_\gamma=$~\SI{30}{\mev}
    (the relative error for \gls{n4lo+} at \ang{80} is only \SI{0.12}{\percent})
    and is also quite narrow for larger energies (at \SI{140}{\mev} 
    the error at the same angle is \SI{1.46}{\percent}).
    At the lower chiral orders, this band is obviously much wider:
    at \gls{n2lo} it is \SI{1.25}{\percent} at $\text{E}_\gamma=$~\SI{30}{\mev}
    and \SI{15.0}{\percent} at $\text{E}_\gamma=$~\SI{140}{\mev}. 
 
    Finally, \fig{Diff_cross_err} (right) presents a cutoff dependence
    of the differential cross section. The ideal case is when the dependency is so weak that
    the choice of the parameter $\Lambda$ would not make significant 
    changes. In practice the choice of this parameter can be 
    important as it makes a noticeable difference in prediction at higher energies.
    Namely, while on
    the top of \fig{Diff_cross_err} ($\text{E}_\gamma=$~\SI{30}{\mev}) the cutoff dependence is so tiny,
    that, in fact, all the lines (for different $\Lambda$ values)
    overlap each other and we cannot distinguish them with the naked eye:
    the relative difference at maximum is \SI{0.08}{\percent}.
    This is approximately $\frac{2}{3}$ smaller than truncation error discussed above.
    hov, with increasing photon energy up to \SIlist{100; 140}{\mev} 
    (middle and bottom rows of the right column of \fig{Diff_cross_err}) the spread becomes bigger:
    the  uncertainty related to the 
    $\Lambda$-dependence is  \SI{3.35}{\percent} at \SI{100}{\mev}
    and \SI{5.66}{\percent} at \SI{140}{\mev} (the same $\theta_p$).
    Thus at two higher energies the cut-off dependence becomes more important
    than truncation errors.
    That shows, that proper choice of the $\Lambda$ is important.
    However, if I restrict myself to $\Lambda=\SIlist{450;500}{\mev}$,
    the dependence drops to \SI{1.98}{\percent} at $\text{E}_\gamma=\SI{140}{\mev}$. 

    On the Fig.~\ref{Cutoff_dep} we saw that the total
    cross section for the same energies has the cutoff spread
    around \SI{4.5}{\percent} for \SI{100}{\mev} and \SI{8}{\percent} for \SI{140}{\mev}
     For $\text{E}_\gamma=\SI{30}{\mev}$ it is below  \SI{1}{\percent}.


    \begin{figure}[h]
        \centering
        \begin{subfigure}[t]{0.46\textwidth}
            \caption{}
            \includegraphics[width=\textwidth]{Figures_De/CROSS2_order_vert.pdf}
            \label{Diff_cross_order}
        \end{subfigure}
        \begin{subfigure}[t]{0.46\textwidth}
            \caption{}
            \includegraphics[width=\textwidth]{Figures_De/Diff_cross_pw_1nc.pdf}
            \label{Diff_cross_pw_1nc}
        \end{subfigure}
        \caption{Differential cross section $\frac{d^2\sigma}{d\Omega}$
        as a function of the outgoing proton momentum polar angle $\theta_p$ in the center of mass frame 
        for the photon energy \SI{30}{\mev} (top), \SI{100}{\mev} (middle) and \SI{140}{\mev} (bottom).
        {\bf (a)} Results obtained using the \gls{sms} potential
        at different chiral orders (from LO to \gls{n4lo+}) with the cutoff parameter $\Lambda=\SI{450}{\mev}$ and 
        2NC contributions taking via the Siegert theorem.
        For the sake of comparison, predictions obtained with the \gls*{av18} potential are shown
        by dashed-dotted purple line.
        Data points (filled and empty circles) are from \cite{Ying_Experiment_Deut}
        for $\text{E}_\gamma=\SIlist[list-units = single]{30; 100}{\mev}$
        and \cite{DeSanctis_Experiment_Deut} for $\text{E}_\gamma=\SI{140}{\mev}$.
        {\bf (b)} Predictions obtained with the chiral \gls{n4lo+} potential and $\Lambda=\SI{450}{\mev}$
        with various models of nuclear current(1N and 2N) and scattering state.
        The blue solid curve is plane-wave plus rescattering parts, 1NC + Siegert(the same as \gls{n4lo+} line in (a)).
        THe pink dashed curve shows predictions obtained with
        the single-nucleon current only (without applying the Siegert theorem) and the green dashed-dotted
        curve represents predictions with the full current(1N + Siegert) but plane-wave part only.
        }
        % Results in {\bf (a)} are obtained using \gls*{sms} potential
        % at different chiral orders (from LO to \gls{n4lo+}) with the cutoff parameter $\Lambda=\SI{450}{\mev}$ and 
        % 2NC contributions taking via Siegert theorem.
        % Data points (filled and empty circles) are from \cite{Ying_Experiment_Deut}
        % for (\SIlist[list-units = single]{30; 100}{\mev})
        % and \cite{DeSanctis_Experiment_Deut} (for energy \SI{140}{\mev}).
        % Predictions obtained with chiral \gls{n4lo+} potential and $\Lambda=\SI{450}{\mev}$ are on {\bf (b)}.
        % Blue solid line is a best predictions we have (plane-wave plus rescattering parts, 1NC + Siegert), pink dashed line shows predictions obtained with
        % single-nucleon current only (without Siegert contributions) and green dashed-dotted line
        % is a prediction with plane-wave part only - without rescattering.}
        \label{Diff_cross_order_pw}
    \end{figure}


        
    \begin{figure}[h]
        \centering
        \begin{subfigure}[t]{0.46\textwidth}
            \caption{Truncation error bands.}
            \includegraphics[width=\textwidth]{Figures_De/CROSS2_truncation_vert.pdf}
            \label{Diff_cross_truncation}
        \end{subfigure}
        \begin{subfigure}[t]{0.46\textwidth}
            \caption{Cutoff dependence.}
            \includegraphics[width=\textwidth]{Figures_De/CROSS2_cutoff_vert.pdf}
            \label{Diff_cross_cutoff}
        \end{subfigure}
        \caption{Theoretical uncertainies 
        for the differential cross section $\frac{d^2\sigma}{d\Omega}$
        as a function of the outgoing proton's momentum polar angle $\theta_p$ in the center of mass frame 
        for the photon energy is \SI{30}{\mev} (top row), \SI{100}{\mev} (middle row) and \SI{140}{\mev} (bottom row).
        {\bf(a)} The truncation error bands for each energy in a corresponding row. 
        Results are obtained using the \gls{sms} potential at different chiral orders (from NLO to \gls{n4lo+}) 
        with the cutoff parameter $\Lambda=\SI{450}{\mev}$ and 2NC contributions taking via the Siegert approach.
        {\bf (b)} Predictions obtained using different values of the cutoff parameter $\Lambda$.
        The double-dotted-dashed red curve, the solid black line, the dashed green line
        and the dotted blue line represent predictions obtained 
        with $\Lambda=\SIlist[list-units = single]{400;450;500;550}{\mev}$ respectively
        and the chiral potential \gls{n4lo+}. 
        Data points are the same as in \fig{Diff_cross_order}}
        \label{Diff_cross_err}
    \end{figure}

    \clearpage

    \subsection{Polarisation observables}
    \label{sec:polarisation_results}

    In this subsection I will present my predictions for 
    selected polarisation observables.
    I start with deuteron vector $i\text{T}_{11}$ and tensor $\text{T}_{20}$, $\text{T}_{21}$ and $\text{T}_{22}$ analyzing power,
    which are defined as \cite{ArenhovelPhotodisint1991}:
    
    \begin{eqnarray}
        i\text{T}_{11} (\theta) &=& -2 \frac{\Im V_{11}}{V_{00}},\\
        \text{T}_{2i} (\theta) &=& \frac{(2 - \delta_{i0}) \Re V_{2i}}{V_{00}}, i=0,1,2,
    \end{eqnarray}
    where $V_{ij}$ is \tmp{??? all derivation?}.

    in the Figures \ref{T20_T21_30} (a, b) and \ref{T22_T11_30}(a,b)
    I show my predictions at $\text{E}_\gamma = \SI{30}{\mev}$, for the
    $\text{T}_{20}$, $\text{T}_{21}$, $\text{T}_{22}$  and $i\text{T}_{11}$ respectively as functions 
    of the outgoing proton angle $\theta_p$ in the \gls{cm} frame. Each of them
    is organized in the similar way: the top
    panel shows a dependence of the predictions on the 
    chiral order of the potential. The middle subfigure is
    showing a correspondent truncation error for each of the 
    predictions from a top row (excepting LO, because its uncertainty is
    too large and will make the readability worse). The last (bottom)
    panel shows the cutoff dependence at the chiral
    order \gls{n4lo+}. 

    All the analyzing powers presented here show excellent convergence 
    upon a chiral order as it is hard to distinguish the predictions
    from each subsequent order starting from the \gls{n2lo}.
    The relative width of \gls{n4lo+} truncation band 
    for T$_{20}$, T$_{21}$ and T$_{22}$
    are \SIlist{0.06; 0.05; 0.19}{\percent} respectively (at $\theta_p=$ \ang{90}, \ang{60} and \ang{90} respectively).
    The slowest convergence is observed for $i\text{T}_{11}$ (Fig.~\ref{T11_30_vert})
    where we can recognize \gls{n2lo} band.
    Still at \gls{n4lo+} it has only a \SI{0.2}{\percent}
    width truncation band at $\theta_p = \ang{20}$ (maximum point).
    The cutoff dependency for all regarded observables is weak and 
    predictions for each value of the $\Lambda$ are hardly separable 
    with the naked eye.
    The relative spread of the predictions based on various $\Lambda$ at the same angles as above 
    are \SIlist{0.87; 0.94; 3.42; 0.68}{\percent} for T$_{20}$, T$_{21}$, T$_{22}$ and $i\text{T}_{11}$, respectively.


    Predictions for the photon energy $\text{E}_\gamma = \SI{100}{\mev}$
    (Figs. \ref{T20_T21_100} and \ref{T22_T11_100}) preserve similar
    trends for each observable. 
    Generally, predictions are being converged starting
    even from the \gls{n2lo} while for $i\text{T}_{11}$
    we can see 
    that truncation error's bands are noticeably wide
    even at \gls{n4lo} and \gls{n4lo+}.
    Cutoff dependence at this energy is a bit stronger
    compared to those at $\text{E}_\gamma = \SI{30}{\mev}$, especially
    for 
    $\text{T}_{22}$ and $i\text{T}_{11}$ analyzing powers (Fig.~\ref{T22_T11_30}),
    where one can see 
    slightly stronger discrepancy at the stationary points.
    
    Looking at the predictions for the deuteron tensor analyzing powers,
    we can conclude that cutoff dependence is generally weak
    and the choice of $\Lambda$ does not affect predictions much
    even at that higher energy:
    the relative spread among all cutoffs for T$_{20}$ is \SI{1.54}{\percent}
    around the point of maximum ($\theta_p = \ang{90}$).
    For other components of the tensor analyzing power spread is:
    \SI{0.14}{\percent} at $\theta_p = \ang{60}$ for T$_{21}$;
    \SI{3.68}{\percent} at $\theta_p = \ang{90}$ for T$_{22}$;
    and for $i\text{T}_{11}$: \SI{4.91}{\percent} at $\theta_p = \ang{75}$.
    We see again that $i\text{T}_{11}$ has much larger spread in the maximum and
    is more sensitive to the cutoff choice.

    Turning our attraction to the chiral order convergence, we can
    observe that predictions are mostly converged after \gls{n2lo} or \gls{n3lo}.
    The relative width of \gls{n4lo+} truncation band 
    for T$_{20}$, T$_{21}$ and T$_{22}$
    are \SIlist{0.7; 0.7; 0.04}{\percent} respectively (at the same angles as used above).
    Another case is $i\text{T}_{11}$, for which this width is much larger: \SI{6.8}{\percent} at $\theta_p = \ang{20}$.
    The truncation uncertainty for all analyzing powers is much lower than one,
    connected to the choice of cutoff parameter.
    Although, $i\text{T}_{11}$ seems to be more sensitive both
    to the choice of the cutoff parameter and to the chiral order than other regarded observables.
    But even for $i\text{T}_{11}$ cutoff spread is almost twice larger than truncation error's band.
    Standing out of other tensor components, $i\text{T}_{11}$ can be useful for the investigation of 
    the cutoff dependence of the model.
    Of course, we can repeat once more that 
    our model is less accurate at higher energies which is reflected
    in a stronger cutoff dependence and slower chiral convergence.
    However we conclude that even at these energies tensor analyzing powers are
    converged with respect to the chiral order.

    Comparison of the predictions obtained using the chiral \gls{n4lo+} potential
    $\Lambda = \SI{450}{\mev}$ with ones obtained taking \gls{av18} potential
    show a very good aggreement: the relative difference at $\text{E}_\gamma=\SI{30}{\mev}$
    is below \SI{6}{\percent} for regarded tensor analyzing powers at specified angles.
    And at  $\text{E}_\gamma=\SI{100}{\mev}$ it is even better: the difference does not exceed 
    \SI{4}{\percent}.


    \begin{figure}[htb]
        \centering
        \begin{subfigure}[b]{0.46\textwidth}
            \caption{T$_{20}$}
            \includegraphics[width=\textwidth]{Figures_De/T20D2_30mev.pdf}
            \label{T20_30_vert}
        \end{subfigure}
        \begin{subfigure}[b]{0.46\textwidth}
            \caption{T$_{21}$}
            \includegraphics[width=\textwidth]{Figures_De/T21D2_30mev.pdf}
            \label{T21_30_vert}
        \end{subfigure}
        \caption{Tensor analyzing powers T$_{20}$  {\bf (a)}
        and T$_{21}$ {\bf (b)}
        as a function of the outgoing proton angle $\theta_p$ in \gls{cm} frame 
        for the photon energy $\text{E}_\gamma = \SI{30}{\mev}$.
        Top row presents results obtained using potential
        with different chiral orders (from LO to \gls{n4lo+}) with cutoff parameter $\Lambda=\SI{450}{\mev}$.
        The middle row shows truncation errors for each 
        chiral order starting from NLO and
        the bottom row presents a cutoff dependency at \gls{n4lo+}.
        For the sake of comparison, predictions obtained with the \gls{av18} potential are given as well.}
        \label{T20_T21_30}
    \end{figure}

    \begin{figure}[htb]
        \centering
        \begin{subfigure}[b]{0.46\textwidth}
            \caption{T$_{22}$}
            \includegraphics[width=\textwidth]{Figures_De/T22D2_30mev.pdf}
            \label{T22_30_vert}
        \end{subfigure}
        \begin{subfigure}[b]{0.46\textwidth}
            \caption{$i\text{T}_{11}$}
            \includegraphics[width=\textwidth]{Figures_De/T11D2_30mev.pdf}
            \label{T11_30_vert}
        \end{subfigure}
        \caption{The same as in \fig{T20_T21_30} but for the deuteron tensor analyzing power
        T$_{22}$ (subfigure {\bf (a)}) and the deuteron vector analyzing power $i\text{T}_{11}$ (subfigure {\bf (b)}).}
        \label{T22_T11_30}
    \end{figure}

    \begin{figure}[htb]
        \centering
        \begin{subfigure}[b]{0.46\textwidth}
            \caption{T$_{20}$}
            \includegraphics[width=\textwidth]{Figures_De/T20D2_100mev.pdf}
            \label{T20_100_vert}
        \end{subfigure}
        \begin{subfigure}[b]{0.46\textwidth}
            \caption{T$_{21}$}
            \includegraphics[width=\textwidth]{Figures_De/T21D2_100mev.pdf}
            \label{T21_100_vert}
        \end{subfigure}
        \caption{The deuteron tensor analyzing powers T$_{20}$  {\bf (a)}
        and T$_{21}$ {\bf (b)}
        as a function of the outgoing proton angle $\theta_p$ in the center of mass frame 
        for the photon energy E$_\gamma=\SI{100}{\mev}$.
        Top row presents results obtained using potential
        with different chiral orders (from \gls{lo} to \gls{n4lo+}) with cutoff parameter $\Lambda=\SI{450}{\mev}$.
        The middle row shows truncation errors for each 
        chiral order starting from \gls{nlo} and the
        bottom row presents a cutoff dependency at \gls{n4lo+}.
        For the sake of comparison, predictions obtained with the \gls{av18} potential are given as well.}
        \label{T20_T21_100}
    \end{figure}

    \begin{figure}[htb]
        \centering
        \begin{subfigure}[b]{0.46\textwidth}
            \caption{T$_{22}$}
            \includegraphics[width=\textwidth]{Figures_De/T22D2_100mev.pdf}
            \label{T22_100_vert}
        \end{subfigure}
        \begin{subfigure}[b]{0.46\textwidth}
            \caption{$i\text{T}_{11}$}
            \includegraphics[width=\textwidth]{Figures_De/T11D2_100mev.pdf}
            \label{T11_100_vert}
        \end{subfigure}
        \caption{The same as in \fig{T20_T21_100} but for the deuteron tensor analyzing power
        T$_{22}$ (subfigures {\bf (a)}) and deuteron vector analyzing power $i\text{T}_{11}$ (subfigures {\bf (b)}).}
        \label{T22_T11_100}
    \end{figure}

    In \fig{tensor_pw_1nc} together with our most advanced "Full" predictions 
    at $\text{E}_\gamma = \SI{30}{\mev}$ (\gls{n4lo+}, $\Lambda=\SI{450}{\mev}$, the Siegert theorem),
    I show predictions obtained with 1NC only and 
    the Siegert predictions with plane-wave contribution without rescattering part.
    In the case of deuteron's tensor analyzing power components, the contribution of rescattering part is  
    imortant for T$_{20}$, T$_{21}$ and T$_{22}$ (the relative difference is up to \SI{20}{\percent} in extremum points).
    and crucial for
    $i\text{T}_{11}$ where the PW part equals to zero.
    The 2NC component taken into account via Siegert theorem has a dominant contribution here. We see that 
    1NC predictions are absolutely away from the "Full" predictions and in case of $i\text{T}_{11}$
    does not even reflect complete prediction qualitatively.

    \begin{figure}[h]
        \begin{center}
        \includegraphics[width=0.9\textwidth]{Figures_De/TensorPowers_pw_1nc.pdf}
        \end{center}
        \caption{The deuteron tensor analyzing powers T$_{20}$, T$_{21}$, T$_{22}$ and 
        the vector analyzing power $i\text{T}_{11}$ as a function of the
        outgoing proton angle $\theta_p$ in the \gls{cm} frame at $\text{E}_\gamma = \SI{30}{\mev}$.
        Similarly to \fig{Diff_cross_pw_1nc} predictions obtained with chiral \gls{n4lo+} potential
        and $\Lambda=\SI{450}{\mev}$ are presented for three models.
        Blue solid line is the most complete prediction we have (plane-wave plus rescattering parts, 1NC + Siegert), pink dashed line shows predictions obtained with
        single-nucleon current only (without Siegert contributions) and green dashed-dotted line
        is a prediction in which we neglect the rescattering part
        and stick to the plane-wave part only.}
        \label{tensor_pw_1nc}
    \end{figure}

    \fig{tensor_pw_1nc_100mev} presents similar results but for E$_\gamma = \SI{100}{\mev}$
    and it is intersting that difference between Full and 1NC prediction becomes smaller
    and it is especially visible for $\text{T}_{22}$. At this energy the relative difference 
    between "Full" and 1NC predictions
    at $\theta_p = \ang{90}$ is \SI{43.6}{\percent} comparing to \SI{122.8}{\percent}
    at E$_\gamma = \SI{30}{\mev}$. Similarly, the difference for T$_{20}$
    at E$_\gamma = \SI{30}{\mev}$($\theta_p = \ang{90}$) is \SI{91.4}{\percent}
    and at E$_\gamma = \SI{100}{\mev}$ it drops to \SI{28.8}{\percent}.
    This trend is noticeable looking also on results presented below. 

    \begin{figure}[h]
        \begin{center}
        \includegraphics[width=0.9\textwidth]{Figures_De/TensorPowers_100mev_pw_1nc.pdf}
        \end{center}
        \caption{The same as in \fig{tensor_pw_1nc} but for E$_\gamma=\SI{100}{\mev}$}
        \label{tensor_pw_1nc_100mev}
    \end{figure}


    
    In the next figures, I will compare my predictions with experimental data.
    I will keep a similar way as it was done
    in \cite{rachek2007} where due to experimental conditions results are given not at single photon energy,
    but for a specific ranges of $\text{E}_\gamma$.
    In the Figures \ref{tensor_angular_25-45} - \ref{tensor_angular_230-330}
    I show an angular dependence of the $\text{T}_{2i}$ ($i=0,1,2$) for a specific energy bands:
    \SIrange[range-phrase=--]{25}{45}{\mev}, \SIrange[range-phrase=--]{45}{70}{\mev},
    \SIrange[range-phrase=--]{70}{100}{\mev}, and \SIrange[range-phrase=--]{230}{330}{\mev}.
    Solid blue line shows an average value of the observable in the specified energy intervals:
    obtained at \gls{n4lo+} with $\Lambda=\SI{450}{\mev}$, while the pink dashed line is a prediction
    obtained with a same setup but without using a contributions from Siegert approach
    (single nucleon current only). Bands for each of prediction specify the spread of
    predictions due to the energy band.
    
    One clearly sees that the data description is better for the predictions with Siegert contributions 
    and SN current alone is not able to describe experiment properly. With increasing energy 
    (above \SI{100}{\mev}),
    the difference between predicted values and experimental data becomes larger
    (especially for $\text{T}_{22}$), 
    which shows a neccesity of improving theoretical model befor applying it to 
    higher energies. Nevertheless, even with approximations used,
    the data description remains reasonable. 
    We that quite often and in particular in \fig{tensor_angular_180-230} and \fig{tensor_angular_230-330}
    the data description is worse for smaller angles. Especially for the tensor analyzing power T$_{22}$
    the group of data point lying closer to $\theta_p = \ang{30}$ are farther 
    from the theoretical prediction than second group.
    In \fig{tensor_angular_230-330} the description of data points for T$_{20}$ seem to be better
    with SN current (dashed line), but in this case it is only accidental match as we do not observe similar 
    trend for any other angular span or for different observable. 
    




    \begin{figure}[h]
        \begin{center}
        \includegraphics[width=1\textwidth]{Figures_De/Tensor_analyzing_power_angular_E25-45.pdf}
        \end{center}
        \caption{Tensor analyzing powers T$_{20}$, T$_{21}$ and T$_{22}$ as a function of the
        outgoing proton angle $\theta_p$ in the \gls{cm} frame.
        Solid blue line is a mean value of my predictions obtained 
        at energy values from \SIrange[range-phrase=\text{ to }]{25}{45}{\mev} with the
        \gls{sms} potential at \gls{n4lo+} chiral order and with $\Lambda$~=~450~MeV
         and with SN current used together with the Siegert approach. 
        Pink dashed line is similar prediction but with SN only. 
        The corresponding bands show the limits of predictions in the regarded
        energy region.
        % Filled bands show maximal spread of my predictions obtained with a 
        % \gls*{sms} potential at \gls{n4lo+} chiral order and with $\Lambda$~=~450~MeV
        % for the energy span from 25 to 45 MeV. Blue bands correspond to the
        % case where SN current was used together with Siegert approach and 
        % pink bands - to the SN currentonly. 
        Filled circles are experimental data
        from \cite{rachek2007} for the same energy span.}
        \label{tensor_angular_25-45}
    \end{figure}

    \begin{figure}[h]
        \begin{center}
        \includegraphics[width=0.95\textwidth]{Figures_De/Tensor_analyzing_power_angular_E45-70.pdf}
        \end{center}
        \caption{The same as in \fig{tensor_angular_25-45} but for energy bin \SIrange{45}{70}{\mev}.}
        \label{tensor_angular_45-70}
    \end{figure}

    \begin{figure}[h]
        \begin{center}
        \includegraphics[width=0.95\textwidth]{Figures_De/Tensor_analyzing_power_angular_E70-100.pdf}
        \end{center}
        \caption{The same as in \fig{tensor_angular_25-45} but for energy bin \SIrange{70}{100}{\mev}.}
        \label{tensor_angular_70-100}
    \end{figure}        

    \begin{figure}[h]
        \begin{center}
        \includegraphics[width=0.95\textwidth]{Figures_De/Tensor_analyzing_power_angular_E100-140.pdf}
        \end{center}
        \caption{The same as in \fig{tensor_angular_25-45} but for energy bin \SIrange{100}{140}{\mev}.}
        \label{tensor_angular_100-140}
    \end{figure}
        
        

    \begin{figure}[h]
        \begin{center}
        \includegraphics[width=0.95\textwidth]{Figures_De/Tensor_analyzing_power_angular_E140-180.pdf}
        \end{center}
        \caption{The same as in \fig{tensor_angular_25-45} but for energy bin \SIrange{140}{180}{\mev}}
        \label{tensor_angular_140-180}
    \end{figure}
        

    \begin{figure}[h]
        \begin{center}
        \includegraphics[width=0.95\textwidth]{Figures_De/Tensor_analyzing_power_angular_E180-230.pdf}
        \end{center}
        \caption{The same as in \fig{tensor_angular_25-45} but for energy bin \SIrange{180}{230}{\mev}}
        \label{tensor_angular_180-230}
    \end{figure}

    \begin{figure}[h]
        \begin{center}
        \includegraphics[width=0.95\textwidth]{Figures_De/Tensor_analyzing_power_angular_E230-330.pdf}
        \end{center}
        \caption{The same as in \fig{tensor_angular_25-45} but for energy bin \SIrange{230}{330}{\mev}}
        \label{tensor_angular_230-330}
    \end{figure}
        


    In the Figure \ref{T20_vs_en} the energy dependence of $\text{T}_{20}$ and $\text{T}_{22}$
    is presented for the energy range 0-400~MeV. Beside my predictions I also demonstrate the experimental data from
    \cite{rachek2007} and \cite{mishev1993} as well as theoretical calculations from \cite{Schmitt1989}.
    For $\text{T}_{20}$ all models are able to describe experimental data well even for
    high energies. On the other hand, $\text{T}_{22}$ is not described so well: for the 
    energies below \SI{140}{\mev} the predictions are within uncertainties of experimental data,
    but further the difference with the data increases. Above \SI{140}{\mev}
    my "Full" predictions do not  
    reflect the qualitative nature of the data. Namely, I observe that
    data points start ascending which is not represented in my predictions.
    Theoretical predictions from \cite{Schmitt1989} (brown dashed curve) are also not able
    to describe data quantitatively for $\text{T}_{22}$, but increasing of T$_{22}$
    towords data is present. The calculation predictions in \cite{Schmitt1989}
    are obtained with a one-body current using the Bonn OBEPR
    NN potential with the major part of meson exchange
    currents (MEC) included implicitly via the Siegert opera-
    tors plus explicit
    pion exchange currents (MEC), isobar configurations
    (IC) and the leading order relativistic corrections
    (RC). So authors use a different potential, but
    probably the main difference of predictions is coming from the RC included there.

    Similar picture is seen in the Figures \ref{tensor_energy_24-48} and \ref{tensor_energy_70-102}
    where I show an energy dependence of the mean of deuteron analyzing powers over 
    specific angular ranges (following the data from \cite{rachek2007}).
    In \fig{tensor_energy_24-48} we see that only predictions for $\text{T}_{20}$
    are able to reflect the experimental results,
    while for $\text{T}_{21}$ and $\text{T}_{22}$ predictions are reasonable (quantitative-wise) 
    only for lower energies and difference with data becomes larger
    when energy increases. Predictions for $\text{T}_{21}$ and $\text{T}_{22}$ once more 
    confirm an insufficiency of SN and an importance of
    2-nucleon current contributions.
    The description is better for larger angles \fig{tensor_energy_70-102}:
    at lower energies (below \SI{140}{\mev}) the correspondance to
    experimental data is good for all three observables, but above that 
    threshold, all predictions (especially for $\text{T}_{22}$)
    move away from the measurements data.
    

    \begin{figure}[h]
        \begin{center}
        \includegraphics[width=0.9\textwidth]{Figures_De/T20_T22_vs_en.pdf}
        \end{center}
        \caption{The tensor analyzing powers T$_{20}$ and T$_{22}$ as a function of the photon energy E$_\gamma$
        at fixed outgoing proton angle $\theta_p = \ang{88}$ in the center of mass frame.
        My predictions (blue solid line) are obtained with the \gls{sms} potential at the chiral order \gls{n4lo+},
        with the cutoff parameter $\Lambda = \SI{450}{\mev}$ and with 2NC contributions included via the Siegert theorem.
        The dashed pink curve shows predictions obtained with the same interaction, but without 2NC contributions.
        The dashed-dotted brown curve presents theoretical results from \cite{Schmitt1989}.
        Experimental data is taken from \cite{rachek2007} (filled circles)
        and \cite{mishev1993} (empty circles).}
        \label{T20_vs_en}
    \end{figure}

    \begin{figure}[h]
        \begin{center}
        \includegraphics[width=0.95\textwidth]{Figures_De/TensorPower_Th24-48.pdf}
        \end{center}
        \caption{The tensor analyzing powers T$_{20}$, T$_{21}$ and T$_{22}$ as a function of the
        photon energy within the outgoing proton momentum polar angle $\theta_p$ range $\ang{24} - \ang{48}$
        in the center of mass frame.
        The solid blue curve is a mean value of my predictions at energy values from 25 to \SI{45}{\mev} obtained with
        the \gls{sms} potential at \gls{n4lo+} chiral order and with $\Lambda=\SI{450}{\mev}$
        and with SN current used together with the Siegert approach. 
        The pink dashed curve represents similar predictions but with
        nuclear current reduced to th SN current only. 
        The corresponding bands show predictions at border energies 25 and \SI{45}{\mev}.
        The filled circles are experimental data
        from \cite{rachek2007} for the same energy span.}
        \label{tensor_energy_24-48}
    \end{figure}

    \begin{figure}[h]
        \begin{center}
        \includegraphics[width=0.95\textwidth]{Figures_De/TensorPower_Th70-102.pdf}
        \end{center}
        \caption{The same as in the Fig.~\ref*{tensor_energy_24-48} but
        for the $\theta_p$ in range $\ang{70} - \ang{102}$.}
        \label{tensor_energy_70-102}
    \end{figure}
    
    In the \fig{assymetry} I demonstrate predictions
    for the photon asymmetry $\Sigma_\gamma$ for the 
    deuteron photodisintegration with $\text{E}_\gamma=\SI{20}{\mev}$ (a)
    and \SI{60}{\mev}(b) together with experimental data of 
    \cite{KRAUSE1992_asymetry, depascale_asymmetry, Barannik_asymetry, Vnukov_asymmetry}.
    Both (a) and (b) figures are organized similarly to the 
    figures I showed above for the tensor analyzing powers (e.g. \fig{T22_T11_30}).
    That is the top panel is aimed to demonstrate predictions obtained
    with the chiral \gls{sms} potential at different orders of the chiral expansion,
    the middle one is showing a truncation errors and the bottom one shows 
    the cutoff dependence. For that observable we see a  good 
    convergence with respect to the chiral order. For both regarded 
    energies predictions at different orders are very close to each other
    except the \gls{lo} and \gls{nlo} curves. Nevertheless, at E$_\gamma = \SI{60}{\mev}$
    the truncation error bands reveal some uncertainty connected 
    with the chiral order and it is expected that even some higher chiral 
    orders would still contribute to the predictions at this energy.
    The relative width of the truncation band are 
    \SIlist{0.26;5.04;5.05;5.73;14.96}{\percent} for \gls{n4lo+}, \gls{n4lo}, \gls{n3lo},
    \gls{n2lo} and \gls{nlo} respectively and at $\theta_p=\ang{90}$.

    The cutoff dependence is also much stronger at \SI{60}{\mev}. One clearly sees
    that predictions are different for various values of the $\Lambda$.
    The relative difference betweeen predictions to the cutoff parameter at \SI{20}{\mev}
    is \SI{0.26}{\percent} while at the photon energy \SI{60}{\mev}
    it is  \SI{4.41}{\percent} (both calculated at $\theta_p=\ang{90}$).

     Regarding the correspondence to experimental data, we observe that
     for the lower energy predictions are almost perfectly overlaping
     with exparimental points within the error bars. 
     For far data point our predictions are
     outside data error bars, but they are still within $3\sigma$ range.
     For \SI{60}{\mev}, experimental data points are systematically below theoretical
     curves, especially in the middle of angles range. It seems that some systematic 
     uncertainty is presented in predictions and ad hoc multiplication by some factor
     (around 0.8)
     could help predictions be more similar to experimental data. But, of course, the observed descripancy
     points to simplified character of the model used here.

     In the \fig{asymmetry_90deg} I present a dependence of the photon asymmetry
     $\Sigma_\gamma$ on the photon energy at a fixed value
     of the outgoing proton poalr direction $\theta_p = \ang{90}$ 
     (following the data given at \cite{delbianco_1981} and \cite{depascale_asymmetry}).
     It is noticeable that with increasing energy, the predictions
     are above the experimental data and the discrepancy growths with energy.
     This trend
     was also observed in the angular dependence of the asymmetry at \SI{60}{\mev}
     so I conclude that within our framework, 
     $\Sigma_\gamma$ is sensitive to the initial photon energy and some theoretical
     contributions are missing in order to get satisfactory predictions
     at higher energies. From the \fig{asymmetry_90deg} we can say that
     large descripancy with data starts already above $\text{E}_\gamma = \SI{35}{\mev}$.
     \footnote{It is worth to note that data of \cite{depascale_asymmetry} are
     above these from \cite{delbianco_1981} (however still inside $3\sigma$ experimental
     error bands) and remain in agreemant with my predictions even at E$_\gamma=\SI{60}{\mev}$.}

    \begin{figure}[h]
        \centering
        \begin{subfigure}[b]{0.46\textwidth}
            \caption{\small E$_\gamma = \SI{20}{\mev}$}
            \includegraphics[width=\textwidth]{Figures_De/AX2_20mev.pdf}
            \label{AX_20_vert}
        \end{subfigure}
        \begin{subfigure}[b]{0.46\textwidth}
            \caption{\small E$_\gamma = \SI{60}{\mev}$}
            \includegraphics[width=\textwidth]{Figures_De/AX2_60mev.pdf}
            \label{AX_60_vert}
        \end{subfigure}
        \caption{The photon asymmetry $\Sigma_\gamma$ 
        as a function of the outgoing proton angle in the center of mass frame 
        for the photon energy \SI{20}{\mev}(a) and \SI{60}{\mev}(b).
        Top row presents results obtained using the \gls{sms} potential
        with chiral orders from \gls{lo} to \gls{n4lo+} and with the cutoff parameter $\Lambda=\SI{450}{\mev}$.
        The middle row shows truncation errors for each 
        chiral order starting from NLO and the
        bottom row presents a cutoff dependency for the chiral potential \gls{n4lo+}.
        Filled circles are experimental data from \cite{KRAUSE1992_asymetry},
        empty circles - from \cite{depascale_asymmetry}, filled squares
        - from \cite{Barannik_asymetry} and triangles are from \cite{Vnukov_asymmetry}.
        For the sake of comparison, predictions obtained with the \gls{av18} potential are
        given by the dashed-dotted curve as well.}
        \label{assymetry}
    \end{figure}
     
    \begin{figure}[h]
        \begin{center}
        \includegraphics[width=0.75\textwidth]{Figures_De/AX2_90deg.pdf}
        \end{center}
        \caption{The photon asymmetry $\Sigma_\gamma$ 
        as a function of the photon energy  
        at the fixed outgoing proton's momentum polar angle $\theta_p=90^\circ$.
        Each curve corresponds to the particular value of the cutoff parameter
        and chiral potential used here is the \gls{n4lo+} one.
        Filled circles are experimental data from \cite{delbianco_1981},
        empty circles - from \cite{depascale_asymmetry}.}
        \label{asymmetry_90deg}
    \end{figure}
    

    The proton polarization is demonstrated in \fig{PY_30_100_vert} for the 
    photon energy \SI{30}{\mev}(a) and \SI{100}{\mev}(b). In this case even higher energy
    such as \SI{100}{\mev} does not reveal neither
    slower convergence with respect to the chiral order no
    stronger cutoff dependence. Figures for both energies show
    that only next-to-leading order brings relatively high contribution
    while taking into account other subsequent orders does not change predictions
    significantly. In the case of cutoff dependence, we see that curves for each
    value of $\Lambda$ are very close to each other. 
    The relative difference of the predictions with respect to the cutoff parameter
    is \SI{4.04}{\percent} at the minimum point $\theta_p=\ang{130}$ of $\text{E}_\gamma = \SI{30}{\mev}$
    and \SI{5.62}{\percent} at $\theta_p=\ang{160}$ and $\text{E}_\gamma = \SI{100}{\mev}$.
    The dependence is slightly stronger for higher energy, but both values are comparable.


    \begin{figure}[h]
        \centering
        \begin{subfigure}[b]{0.46\textwidth}
            \caption{\small E$_\gamma = 30$~MeV}
            \includegraphics[width=\textwidth]{Figures_De/POLNOUT2(y)_30mev.pdf}
            \label{PY_30_vert}
        \end{subfigure}
        \begin{subfigure}[b]{0.46\textwidth}
            \caption{\small E$_\gamma = 100$~MeV}
            \includegraphics[width=\textwidth]{Figures_De/POLNOUT2(y)_100mev.pdf}
            \label{PY_100_vert}
        \end{subfigure}
        \caption{Proton polarisation $P_y(p)$ 
        \label{PY_30_100_vert}
        as a function of the outgoing proton's momentum polar angle in the center of mass frame 
        with the photon energy \SI{30}{\mev} (a) and \SI{100}{\mev} (b).
        Top figures present results obtained using potential
        with different chiral orders (from \gls{lo} to \gls{n4lo+}) with cutoff parameter $\Lambda=\SI{450}{\mev}$.
        The middle panel show truncation errors for each 
        chiral order starting from \gls{nlo} and
        bottom figures present a cutoff dependency for predictions
        based on the \gls{sms} \gls{n4lo+} potential.
        For the sake of comparison, predictions obtained with the \gls*{av18} potential 
        are shown as well.}
    \end{figure}


    Predictions for the neuteron polarization at the energies \SI{2.75}{\mev} and \SI{100}{\mev} are shown in the
    \fig{Pn_2p75_100}. The choice of energy is conditioned by the availability of experimental data,
    which were taken in 1965 at Livermore \cite{Jewell_neuteronpolarization} and 
    in 1986 at TRIUMF facility \cite{CAMERON_neuteronpolarization}.
    In the case of E$_\gamma = \SI{2.75}{\mev}$ (\fig{Pn_2p75_vert}), we see that predictions reflect
    the behavior of experimental data points qualitatively,
    having more o less a constant offset of the values. Similar offset was obtained
    also in \cite{ArenhovelPhotodisint1991}, where various approaches  were presented.
    Authors compare different models which have a very similar theoretical results
    eventhough different potentials are used with and without relativistic correction.
    One of the theoretical predictions is included even in the experimental papers
    \cite{Jewell_neuteronpolarization} and authors state that there might be 
    systematic error in the calibration of the analyzing power
    of the neutron polarimeter which could affect experimental results precision.
    Interesting is that predictions clearly show symmetrical form of the curve, while in the experimental data
    have some deviations from symmetrical form. It can be a sign that some problem with data can be
    in this case (taking into account also that experiment had been done in 1965) as well 
    as approximations in theoretical models.
    Even in \cite{Jewell_neuteronpolarization} authors make a plot of theoretical curve
    multiplied by the factor 0.879 wich is almost perfectly overlaps with experimental data afterwards. 

    At the energy E$_\gamma = \SI{100}{\mev}$ (\fig{Pn_100_vert}), the difference with experimental
    data does not look like a systematic shift and deviations look like random.
    For most of data points, predicted values are within error bars and only some
    of points (e.g. around \ang{50}) have prediction in distance more than one standard deviation.
    Nevertheless, these data points look like being out of general trend and may be
    a result of unpresice measurement.


    \begin{figure}[h]
        % \centering
        % \includegraphics[width=0.5\textwidth]{Figures_De/POLNOUT2(y)_2.75mev_neuteron.pdf}
        \centering
        \begin{subfigure}[b]{0.46\textwidth}
            \caption{\small E$_\gamma = 2.75$~MeV}
            \includegraphics[width=\textwidth]{Figures_De/POLNOUT2(y)_2.75mev_neuteron.pdf}
            \label{Pn_2p75_vert}
        \end{subfigure}
        \begin{subfigure}[b]{0.46\textwidth}
            \caption{\small E$_\gamma = 100$~MeV}
            \includegraphics[width=\textwidth]{Figures_De/POLNOUT2(y)_100mev_neuteron.pdf}
            \label{Pn_100_vert}
        \end{subfigure}
        \caption{The same as in \fig{PY_30_100_vert} but for the neuteron polarisation
        $P_y(n)$ and at photon energies \SI{2.75}{\mev} {\bf (a)} and \SI{100}{\mev} {\bf (b)}.
        Data are from \cite{Jewell_neuteronpolarization} (empty circles)
        and \cite{CAMERON_neuteronpolarization} (filled circles).}
        \label{Pn_2p75_100}
    \end{figure}


\clearpage

\section{Helium photodisintegration}
\label{sec:hel_results}

\subsection{3N photodisintegration}

    In this section I will discuss results
    for $^3\text{He} \rightarrow p + p + n$ process.
    
    In the \fig{CROSS_HE_EXCL_30} I demonstrate a differential cross section 
    $\frac{d^5\sigma}{d\Omega_1d\Omega_2dS}$ as a function of the S arc length.
    The photon energy is  E$_\gamma=\SI{30}{\mev}$ and the kinematic configuration
    $\theta_1 = \ang{15}$, $\phi_1 = \ang{0}$,
    $\theta_2 = \ang{15}$, $\phi_2 = \ang{180}$; predictions have been obtained without 3NF.
    We see that only \gls{nlo} and \gls{n2lo} introduce relatively large truncation error.
    The maximal width of a band for NLO is \SI{37.6}{\percent} at $S=\SI{10}{\mev}$,
    for \gls{n2lo} it is \SI{12.4}{\percent} at the same point and it is gradually decreasing
    coming to \SI{0.13}{\percent} at \gls{n4lo+}.
    The cutoff spread around maxima values is less than \SI{3}{\percent} and it is
    \SI{0.78}{\percent} at the minimum point ($S=\SI{10}{\mev}$).
    

    \begin{figure}[h]
        \begin{center}
            \includegraphics[width=0.9\textwidth]{Figures_HE/CROSS_excl_trunc_30mev.pdf}
            \end{center}
            \caption{The five-fold differential cross section for the photon 
            energy E$_\gamma=\SI{30}{\mev}$ for the kinematic configuration
            $\theta_1 = \ang{15}$, $\phi_1 = \ang{0}$,
            $\theta_2 = \ang{15}$, $\phi_2 = \ang{180}$.
            The left figure presents truncation error bands obtained using potential
            with chiral orders from \gls{nlo} to \gls{n4lo+}, and with
            cutoff parameter $\Lambda=\SI{450}{\mev}$.
            The right figure presents a cutoff dependency at \gls{n4lo+}.
            Results are obtained with two-nucleon force only.}
            \label{CROSS_HE_EXCL_30}
        \end{figure}

    With larger energy E$_\gamma=\SI{100}{\mev}$ demonstrated In the \fig{CROSS_HE_EXCL_100},
    both truncation error and cutoff spread become larger.
    The truncation band at the maximum point $S=\SI{10}{\mev}$ for NLO is \SI{55.0}{\percent}
    decreasing to \SI{2.2}{\percent} at \gls{n4lo+} which is around 3 times larger than
    it was in predictions with E$_\gamma=\SI{30}{\mev}$.
    The cutoff spread also becomes larger with increasing energy value: \SI{9.0}{\percent}
    at the same (maximum) point which is also $\sim 3$ times bigger than the one we observed
    for the lower energy.

        \begin{figure}[h]
            \begin{center}
            \includegraphics[width=0.9\textwidth]{Figures_HE/CROSS_excl_trunc_100mev.pdf}
            \end{center}
            \caption{The same as in \fig{CROSS_HE_EXCL_30} but 
            for the photon energy E$_\gamma=100$~MeV}
            \label{CROSS_HE_EXCL_100}
        \end{figure}

    Reults for other angular configurations at 
    $\theta_1 = \ang{75}$, $\phi_1 = \ang{75}$,
    $\theta_2 = \ang{75}$, $\phi_2 = \ang{105}$ are
    demonstrated in \fig{CROSS_HE_EXCL_75_75_75_105}.
    The top row shows results obtained with 2NF only, while predictions obtained with 3NF are shown
    on the bottom row.
    It seems that 3NF does not change much the convergence with respect to the chiral order:
    truncation error band at the point of maximum $S=\SI{35}{\mev}$ (\gls{n4lo+})
    is \SI{1.11}{\percent} and \SI{1.16}{\percent} with and without 3NF, respectively.
    So it is almost the same, meaning that 3NF contribution arising starting from \gls{n2lo} order
    does not affect chiral order convergence much.

    The cutoff dependence, in turn, is affected by the 3NF presence. Predictions with 2NF only have
    \SI{13.7}{\percent} spread at the same maximum point $S=\SI{35}{\mev}$, while predictions with 3NF
    have only \SI{1.23}{\percent} relative spread, so the difference is tremendous.

        \begin{figure}[h]
            \begin{center}
                \includegraphics[width=0.9\textwidth]{Figures_HE/CROSS_excl_trunc_100mev_75_75_75_105_2NF_3NF.pdf}
                \end{center}
                \caption{The same as in the \fig{CROSS_HE_EXCL_100} but for the different kinematic
                configuration
                $\theta_1 = \ang{75}$, $\phi_1 = \ang{75}$,
                $\theta_2 = \ang{75}$, $\phi_2 = \ang{105}$.
                Results obtained with 2NF are presented on the top row. The same, but
                with 3NF is presented on the bottom row (starting from \gls{n2lo} - where 3NF appears.)}
                \label{CROSS_HE_EXCL_75_75_75_105}
        \end{figure}


    Similar trends are present also in other configurations, demonstrated for the comparison:
    Figs.\ref{CROSS_HE_EXCL_15_105_15_75},
    Figs.\ref{CROSS_HE_EXCL_45_75_45_105} and
    Figs.\ref{CROSS_HE_EXCL_165_15_15_165}.

    
        \begin{figure}[h]
            \begin{center}
                \includegraphics[width=0.9\textwidth]{Figures_HE/CROSS_excl_trunc_100mev_15_105_15_75_2NF_3NF.pdf}
                \end{center}
                \caption{The same as in the \fig{CROSS_HE_EXCL_75_75_75_105} but for the different kinematic
                configuration.}
                \label{CROSS_HE_EXCL_15_105_15_75}
        \end{figure}




        \begin{figure}[h]
            \begin{center}
                \includegraphics[width=0.9\textwidth]{Figures_HE/CROSS_excl_trunc_100mev_45_75_45_105_2NF_3NF.pdf}
                \end{center}
                \caption{The same as in the \fig{CROSS_HE_EXCL_15_105_15_75} but for the different kinematic
                configuration.}
                \label{CROSS_HE_EXCL_45_75_45_105}
        \end{figure}


        \begin{figure}[h]
            \begin{center}
                \includegraphics[width=0.9\textwidth]{Figures_HE/CROSS_excl_trunc_100mev_165_15_15_165_2NF_3NF.pdf}
                \end{center}
                \caption{The same as in the \fig{CROSS_HE_EXCL_45_75_45_105} but for the different kinematic
                configuration.}
                \label{CROSS_HE_EXCL_165_15_15_165}
        \end{figure}


        The semi-inclusive differential cross section $\frac{d^3\sigma}{d\Omega_p d\text{E}_p}$
        as a function of the outgoing protons energy $E_p$ is demonstrated on the
        \fig{CROSS_HE_INCL_30MEV_2NF} (for E$_\gamma = \SI{30}{\mev}$) and
        \fig{CROSS_HE_INCL_100MEV_2NF} (for E$_\gamma = \SI{100}{\mev}$).
        Each figure consists of subfigures where each row presents results
        for a proton angles $\theta_p = \ang{10}, \ang{50}, \ang{90}, \ang{130}$ and \ang{170}.
        The left part of each subfigure shows a chiral order dependence while the right - cutoff dependence.
        
        At the photon energy \SI{30}{\mev} the chiral dependence is relatively weak: at the maximum point
        ($E_p \simeq \SI{3.8}{\mev}$) the relative difference varies between \SI{12}{\percent} and 
        \SI{28}{\percent} at LO for different angles. This difference decreases with each subsequent order
        resulting in \SI{0.15}{\percent} at $N^4LO+$. At the energy $\text{E}_\gamma = \SI{100}{\mev}$ truncation errors
        are larger: at the $E_p \simeq \SI{1.46}{\mev}$ the discrepancy is around \SI{40}{\percent} (NLO),
        \SI{15}{\percent} (N2LO), coming to \SI{1.5}{\percent} at \gls{n4lo+}.

        The cutoff uncertainty at $\text{E}_\gamma = \SI{30}{\mev}$ is around \SI{2}{\percent}
        and at $\text{E}_\gamma = \SI{100}{\mev}$ is around \SI{8}{\percent} for all angles and 
        at the same values of $E_p$ as regarded above.


        \begin{figure}[h]
            \begin{center}
            \includegraphics[width=0.8\textwidth]{Figures_HE/CROSS_incl_trunc_30mev_all.pdf}
            \end{center}
            \caption{The semi-inclusive differential cross section $\frac{d^3\sigma}{d\Omega_p d\text{E}_p}$
            at E$_\gamma = \SI{30}{\mev}$ as a function of outgoing proton energy E$_p$. Each row represents 
            predictions for different values of the outgoing proton's momentum polar angle $\theta_p$: 
            \ang{10}, \ang{50}, \ang{90}, \ang{130} and \ang{170}. Each column has similar 
            curves and bands definitions as it was for exclusive cross section in \fig{CROSS_HE_EXCL_30}.  
            Predictions have been obtained with the \gls{sms} NN potential but neglecting 3NF.}
            \label{CROSS_HE_INCL_30MEV_2NF}
        \end{figure}

        \begin{figure}[h]
            \begin{center}
            \includegraphics[width=0.8\textwidth]{Figures_HE/CROSS_incl_trunc_100mev_all.pdf}
            \end{center}
            \caption{The same as in \fig{CROSS_HE_INCL_30MEV_2NF} but for E$_\gamma = \SI{100}{\mev}$}
            \label{CROSS_HE_INCL_100MEV_2NF}
        \end{figure}


\clearpage


\subsection{D-p photodisintegration}

    The differential cross section $d\sigma/d\Omega_d$ for the $^3\text{He} + \gamma \rightarrow d + p$ reaction
    is presented In the \fig{CROSS_nd_30} (for the photon energy $\text{E}_\gamma = \SI{30}{\mev}$)
    and In the \fig{CROSS_nd_100} (for the photon energy $\text{E}_\gamma = \SI{100}{\mev}$).
    We see that both truncation and cutoff uncertainties are larger with increasing photon energy.
    The relative spread of the truncation error at the maximum point ($\theta_p = \ang{105}$)
    for the lower energy is \SI{0.05}{\percent} at \gls{n4lo+}, while for the larger energy
    similar spread is \SI{0.45}{\percent} (at \gls{n4lo+}, $\theta_p = \ang{120}$).

    The cutoff dependence is also stronger for the larger energy:
    it is  \SI{1.45}{\percent} at \SI{30}{\mev} and \SI{4.01}{\percent} at \SI{100}{\mev}
    (at the points of maximum). 

\begin{figure}[h]
    \begin{center}
        \includegraphics[width=0.9\textwidth]{Figures_HE/CROSS_nd_trunc_30mev.pdf}
        \end{center}
        \caption{Differential cross section for the D-$p$ 
        two-body photodisintegraion of $^3$He as a function of the d$\gamma$ angle.
        The initial photon energy $\text{E}_\gamma=\SI{30}{\mev}$.}
        \label{CROSS_nd_30}
    \end{figure}


    \begin{figure}[h]
        \begin{center}
        \includegraphics[width=0.9\textwidth]{Figures_HE/CROSS_nd_trunc_100mev.pdf}
        \end{center}
        \caption{The same as on Fig.~\ref{CROSS_nd_30} but 
        for the photon energy E$_\gamma=\SI{100}{\mev}$}
        \label{CROSS_nd_100}
    \end{figure}

    \clearpage
\section{Triton photodisintegration}
    \label{sec:triton_results}
    
    
    In this section I will discuss results
    for $^3\text{H} \rightarrow p + n + n$ process.

    \begin{figure}[h]
        \begin{center}
            \includegraphics[width=0.9\textwidth]{Figures_Triton/CROSS_excl_trunc_30mev_15_0_15_180_2NF.pdf}
            \end{center}
            \caption{The five-fold differential cross section for the photon 
            energy E$_\gamma=\SI{30}{\mev}$ for the kinematic configuration
            $\theta_1 = \ang{15}$, $\phi_1 = \ang{0}$,
            $\theta_2 = \ang{15}$, $\phi_2 = \ang{180}$.
            The left figure presents truncation error bands obtained using potential
            with chiral orders from \gls{nlo} to \gls{n4lo+}, and with
            cutoff parameter $\Lambda=\SI{450}{\mev}$.
            The right figure presents a cutoff dependency at \gls{n4lo+}.
            Results are obtained with two-nucleon force only.}
            \label{CROSS_Triton_EXCL_30_15_0_15_180}
    \end{figure}

    \begin{figure}[h]
        \begin{center}
            \includegraphics[width=0.9\textwidth]{Figures_Triton/CROSS_excl_trunc_100mev_15_0_15_180_2NF.pdf}
            \end{center}
            \caption{The same as in the \fig{CROSS_Triton_EXCL_30_15_0_15_180} but for the photon energy
            E$_\gamma = \SI{100}{\mev}$.}
            \label{CROSS_Triton_EXCL_100mev_15_0_15_180}
    \end{figure}

    \begin{figure}[h]
        \begin{center}
            \includegraphics[width=0.9\textwidth]{Figures_Triton/CROSS_excl_trunc_30mev_75_75_75_105_2NF.pdf}
            \end{center}
            \caption{The same as in the \fig{CROSS_Triton_EXCL_30_15_0_15_180} but for the different kinematic
            configuration
            $\theta_1 = \ang{75}$, $\phi_1 = \ang{75}$,
            $\theta_2 = \ang{75}$, $\phi_2 = \ang{105}$.
            Results obtained with 2NF only.}
            \label{CROSS_Triton_EXCL_75_75_75_105}
    \end{figure}


    \begin{figure}[h]
        \begin{center}
            \includegraphics[width=0.9\textwidth]{Figures_Triton/CROSS_excl_trunc_100mev_75_75_75_105_2NF.pdf}
            \end{center}
            \caption{The same as in the \fig{CROSS_Triton_EXCL_75_75_75_105} but for the photon energy
            E$_\gamma = \SI{100}{\mev}$.}
            \label{CROSS_Triton_EXCL_100mev_75_75_75_105}
    \end{figure}



    \begin{figure}[h]
        \begin{center}
            \includegraphics[width=0.9\textwidth]{Figures_Triton/CROSS_excl_trunc_30mev_15_105_15_75_2NF.pdf}
            \end{center}
            \caption{The same as in the \fig{CROSS_Triton_EXCL_75_75_75_105} but for the different kinematic
            configuration
            $\theta_1 = \ang{15}$, $\phi_1 = \ang{105}$,
            $\theta_2 = \ang{15}$, $\phi_2 = \ang{75}$.
            Results obtained with 2NF only.}
            \label{CROSS_Triton_EXCL_15_105_15_75}
    \end{figure}


    \begin{figure}[h]
        \begin{center}
            \includegraphics[width=0.9\textwidth]{Figures_Triton/CROSS_excl_trunc_100mev_15_105_15_75_2NF.pdf}
            \end{center}
            \caption{The same as in the \fig{CROSS_Triton_EXCL_15_105_15_75} but for the photon energy
            E$_\gamma = \SI{100}{\mev}$.}
            \label{CROSS_Triton_EXCL_100mev_15_105_15_75}
    \end{figure}

    \begin{figure}[h]
        \begin{center}
            \includegraphics[width=0.9\textwidth]{Figures_Triton/CROSS_excl_trunc_30mev_45_75_45_105_2NF.pdf}
            \end{center}
            \caption{The same as in the \fig{CROSS_Triton_EXCL_15_105_15_75} but for the different kinematic
            configuration
            $\theta_1 = \ang{45}$, $\phi_1 = \ang{75}$,
            $\theta_2 = \ang{45}$, $\phi_2 = \ang{105}$.
            Results obtained with 2NF only.}
            \label{CROSS_Triton_EXCL_45_75_45_105}
    \end{figure}


    \begin{figure}[h]
        \begin{center}
            \includegraphics[width=0.9\textwidth]{Figures_Triton/CROSS_excl_trunc_100mev_45_75_45_105_2NF.pdf}
            \end{center}
            \caption{The same as in the \fig{CROSS_Triton_EXCL_45_75_45_105} but for the photon energy
            E$_\gamma = \SI{100}{\mev}$.}
            \label{CROSS_Triton_EXCL_100mev_45_75_45_105}
    \end{figure}

    \begin{figure}[h]
        \begin{center}
            \includegraphics[width=0.9\textwidth]{Figures_Triton/CROSS_excl_trunc_30mev_165_15_15_165_2NF.pdf}
            \end{center}
            \caption{The same as in the \fig{CROSS_Triton_EXCL_45_75_45_105} but for the different kinematic
            configuration
            $\theta_1 = \ang{165}$, $\phi_1 = \ang{15}$,
            $\theta_2 = \ang{15}$, $\phi_2 = \ang{165}$.
            Results obtained with 2NF only.}
            \label{CROSS_Triton_EXCL_165_15_15_165}
    \end{figure}


    \begin{figure}[h]
        \begin{center}
            \includegraphics[width=0.9\textwidth]{Figures_Triton/CROSS_excl_trunc_100mev_165_15_15_165_2NF.pdf}
            \end{center}
            \caption{The same as in the \fig{CROSS_Triton_EXCL_165_15_15_165} but for the photon energy
            E$_\gamma = \SI{100}{\mev}$.}
            \label{CROSS_Triton_EXCL_100mev_165_15_15_165}
    \end{figure}

    % In the \fig{CROSS_HE_EXCL_30} I demonstrate a differential cross section 
    % $\frac{d^5\sigma}{d\Omega_1d\Omega_2dS}$ as a function of the S arc length.
    % The photon energy is  E$_\gamma=\SI{30}{\mev}$ and the kinematic configuration
    % $\theta_1 = \ang{15}$, $\phi_1 = \ang{0}$,
    %     $\theta_2 = \ang{15}$, $\phi_2 = \ang{180}$; predictions have been obtained without 3NF.
    %     We see that only \gls{nlo} and \gls{n2lo} introduce relatively large truncation error.
    %     The maximal width of a band for NLO is \SI{37.6}{\percent} at $S=\SI{10}{\mev}$,
    %     for \gls{n2lo} it is \SI{12.4}{\percent} at the same point and it is gradually decreasing
    %     coming to \SI{0.13}{\percent} at \gls{n4lo+}.
    %     The cutoff spread around maxima values is less than \SI{3}{\percent} and it is
    %     \SI{0.78}{\percent} at the minimum point ($S=\SI{10}{\mev}$).
        
        
        
    \clearpage
\section{Pion absorption from the lowest atomic orbital}
\label{sec:pion_results}

    \subsection{Pion absorption in $^3$He}

    in \fig{Gamma_pnn} and \ref{Gamma_nd} the pion absorption rates are presented as a function
    of the chiral order with different values of the cutoff parameter
    (for $\pi^- + ^3\text{He} \rightarrow p + n + n$ and $\pi^- + ^3\text{He} \rightarrow n + d$ reactions, respectively).
    Both figures show that with fixed chiral order the arrangement of values with respect of the cutoff parameter
    remains the same, namely with increasing $\Lambda$, absorption rate decreases. The only exception in both cases 
    appears at \gls{n3lo} where prediction with $\Lambda = \SI{550}{\mev}$ goes above other predictions.
    At the next order, \gls{n4lo}, it corrects to the normal arrangement.
    This behavior may be connected to the 3NF used for the calculation and in order to check that I show
    a similar figure for a proton radius $r_p$ in \fig{proton_rad} calculated with 
    and without 3NF (left and right panels respectively). Results obtained with 3NF show
    similar deviation at \gls{n3lo} while data obtained without 3NF does not have that.
    Nevertheless, the spread of predictions with respect to the cutoff values is much smaller
    with 3NF and deviation seems to be not crucial as total difference
    between predictions in this case is very small.




    \begin{figure}[h]
        \begin{center}
        \includegraphics[width=0.6\textwidth]{PlotData/PION/Dalitz_maps/figures/Gamma_pnn.pdf}
        \end{center}
        \caption{Absorption rate for $\pi^- + ^3\text{He} \rightarrow p + n + n$ reaction as a function
        of the chiral order with different values of the cutoff parameter $\Lambda$.
        Predictions were obtained with 3NF.}
        \label{Gamma_pnn}
    \end{figure}

    \begin{figure}[h]
        \begin{center}
        \includegraphics[width=0.6\textwidth]{PlotData/PION/Dalitz_maps/figures/Gamma_nd.pdf}
        \end{center}
        \caption{The same as in \fig{Gamma_pnn}, but for $\pi^- + ^3\text{He} \rightarrow n + d$ reaction.}
        \label{Gamma_nd}
    \end{figure}

    \begin{figure}[h]
        \begin{center}
        \includegraphics[width=0.99\textwidth]{PlotData/PION/Dalitz_maps/figures/proton_radius_mt31_3NF.pdf}
        \end{center}
        \caption{\tmp{check mt3} Proton radius $r_p$ as a function of the chiral order calculated with
        different values of the cutoff parameter $\Lambda$. The radius was calculated with 2NF and 3NF (left panel)
        and with 2NF only (right panel).}
        \label{proton_rad}
    \end{figure}

    In Figs.~\ref{pion_map_E1E2_cutoff} and \ref{pion_map_xy_cutoff} I show 
    intensity plots for the double differential absorption rates
    $d^2 \Gamma_{pnn}/dE_1dE_2$ for the $\pi^- + ^3\text{He} \rightarrow p + n + n$
    process as functions of the nucleons energies (first nucleon is proton) and 
    of \tmp{correct naming} Dalitz coordinates($x$ and $y$) respectively.

    In \fig{pion_map_xy_cutoff} coordinates $x$ and $y$ are defined as:

    \begin{align}
        x &= 3 (E_1 + 2E_2 - E)/E, \nonumber\\
        y &= (3E_1 - E)/E,
        \label{dalitz_xy}
    \end{align}
    taking the region where $r^2 \equiv x^2 + y^2 \leq 1$.

    Each of two figures consists of four panels representing predictions obtained with different 
    values of the cutoff parameter $\Lambda$. The difference between predictions which can be 
    noticed with the naked eye - is that area of the central region (corresponding to smallest values)
    becomes larger with increasing $\Lambda$. It coheres to what we saw in \fig{Gamma_pnn}
    where total absorption rate was inversly correlated with cutoff parameter. The dominant contribution
    comes from the region with lowest proton energy values of $E_1 \rightarrow 0$ where both neutrons have similar large values.
    This is a situation when proton is a spectator while both neutrons share all energy
    - quasi-free scattering(QFS). 

    Another region with high absorption rate is neutron-neutron final state interaction (FSI(nn)).
    It is located at high $E_1$ when proton gets one third part of total energy while neutrons both get
    one sixth.
    
        \begin{figure}[h]
            \begin{center}
            \includegraphics[width=0.7\textwidth]{PlotData/PION/Dalitz_maps/figures/Dalitz_map_pnn_E1E2_cutofs.pdf}
            \end{center}
            \caption{Intensity plots for the double differential absorption rates
            $d^2 \Gamma_{pnn}/dE_1dE_2$ for the $\pi^- + ^3\text{He} \rightarrow p + n + n$
            process, obtained using the SMS potential at \gls{n4lo+}
            with all contributions possible: plane wave + rescattering, SN + 2N, 2NF+3NF.
            Each panel present predictions obtained with different values of the cutoff parameter $\Lambda$:
            from \SI{400}{\mev} (upper left) to \SI{550}{\mev} (lower right). Nucleon 1 is a proton.}
            \label{pion_map_E1E2_cutoff}
        \end{figure}

    \begin{figure}[h]
        \begin{center}
        \includegraphics[width=0.7\textwidth]{PlotData/PION/Dalitz_maps/figures/Dalitz_map_pnn_xy_cutofs.pdf}
        \end{center}
        \caption{The same as in \fig{pion_map_E1E2_cutoff} but for the double differential absorption rates
        $d^2 \Gamma_{pnn}/dxdy$.}
        \label{pion_map_xy_cutoff}
    \end{figure}

    Next I show similar colormaps but for the predictions obtained with plane wave component only (without rescattering part)
    in Figs.~\ref{pion_map_E1E2_cutoff_PW} and \ref{pion_map_xy_cutoff_PW}. Presented plots show that
    the difference of predictions obtained without rescattering part with full is very large. 
    Predicted values 
    are few times larger and the distribution is completely different.
    The FSI(nn) region is not presented here in a sense that there is no peak with respect to other 
    values. The QFS region is, on the contrary, 
    which obviously tells us that one has to take into account rescattering pat in order to obtain relevant results.   

    \begin{figure}[h]
        \begin{center}
        \includegraphics[width=0.7\textwidth]{PlotData/PION/Dalitz_maps/figures/Dalitz_map_pnn_E1E2_cutofs_PWIAS.pdf}
        \end{center}
        \caption{Intensity plots for the double differential absorption rates
        $d^2 \Gamma_{pnn}/dE_1dE_2$ for the $\pi^- + ^3\text{He} \rightarrow p + n + n$
        process, obtained using the SMS potential at \gls{n4lo+}
        with plane wave part only (without rescattering).
        All other contributions are the same as in \fig{pion_map_E1E2_cutoff}: SN + 2N and 2NF+3NF.
        Each panel present predictions obtained with different values of the cutoff parameter $\Lambda$:
        from \SI{400}{\mev} (upper left) to \SI{550}{\mev} (lower right). Nucleon 1 is a proton.}
        \label{pion_map_E1E2_cutoff_PW}
    \end{figure}

    \begin{figure}[h]
        \begin{center}
        \includegraphics[width=0.7\textwidth]{PlotData/PION/Dalitz_maps/figures/Dalitz_map_pnn_xy_cutofs_PWIAS.pdf}
        \end{center}
        \caption{The same as in \fig{pion_map_E1E2_cutoff_PW} but for the double differential absorption rates
        $d^2 \Gamma_{pnn}/dxdy$.}
        \label{pion_map_xy_cutoff_PW}
    \end{figure}

    Results obtained with Plane wave plus recattering part and with single nucleon current only
    are presented in Figs.~\ref{pion_map_E1E2_cutoff_1NC} and \ref{pion_map_xy_cutoff_1NC}.
    As previously, each panel in figures presents predictions obtained with different values of the cutoff parameter $\Lambda$.
    In contrast to the configurations shown above, the change of the cutoff value has larger impact here:
    we see that there is a different pattern in each panel. I does not change dramatically, but 
    two peaks inside the figure become more or less clear. On the other hand, the distribution of 
    the absorption rate is very different from the one, obtained with more complete components setup 
    (Figs.~\ref{pion_map_E1E2_cutoff} and \ref{pion_map_xy_cutoff}).

    \begin{figure}[h]
        \begin{center}
        \includegraphics[width=0.7\textwidth]{PlotData/PION/Dalitz_maps/figures/Dalitz_map_pnn_E1E2_cutofs_1NC.pdf}
        \end{center}
        \caption{Intensity plots for the double differential absorption rates
        $d^2 \Gamma_{pnn}/dE_1dE_2$ for the $\pi^- + ^3\text{He} \rightarrow p + n + n$
        process, obtained using the SMS potential at \gls{n4lo+}
        with SN current only (without 2N).
        All other contributions are the same as in \fig{pion_map_E1E2_cutoff}: PWIAS+RESC and 2NF+3NF.
        Each panel present predictions obtained with different values of the cutoff parameter $\Lambda$:
        from \SI{400}{\mev} (upper left) to \SI{550}{\mev} (lower right). Nucleon 1 is a proton.}
        \label{pion_map_E1E2_cutoff_1NC}
    \end{figure}

    \begin{figure}[h]
        \begin{center}
        \includegraphics[width=0.7\textwidth]{PlotData/PION/Dalitz_maps/figures/Dalitz_map_pnn_xy_cutofs_1NC.pdf}
        \end{center}
        \caption{The same as in \fig{pion_map_E1E2_cutoff_1NC} but for the double differential absorption rates
        $d^2 \Gamma_{pnn}/dxdy$.}
        \label{pion_map_xy_cutoff_1NC}
    \end{figure}

    Figs.~\ref{pion_map_E1E2_order} and \ref{pion_map_xy_order} show prediction obtained using similar configuration
    as in Figs.~\ref{pion_map_E1E2_cutoff} and \ref{pion_map_xy_cutoff}, but each panel includes
    predictions obtained with different chiral orders of the \gls{sms} potential.
    We see that predictions are not sensitive to the chiral order and even \gls{n2lo} predictions
    are pretty much similar to ones obtained with the most advanced \gls{n4lo} potential. 
    
    \begin{figure}[h]
        \begin{center}
            \includegraphics[width=0.7\textwidth]{PlotData/PION/Dalitz_maps/figures/Dalitz_map_pnn_E1E2_orders.pdf}
        \end{center}
        \caption{Intensity plots for the double differential absorption rates
        $d^2 \Gamma_{pnn}/dE_1dE_2$ for the $\pi^- + ^3\text{He} \rightarrow p + n + n$
        process, obtained using the SMS potential at \gls{n4lo+}
        with all contributions possible: plane wave + rescattering, SN + 2N, 2NF+3NF.
        Each panel present predictions obtained with different chiral orders of the \gls{sms} potential:
        from \gls{n2lo} (upper left) to \gls{n4lo+} (lower right) and with $\Lambda = \SI{450}{\mev}$.
        Nucleon 1 is a proton.}
        \label{pion_map_E1E2_order}
    \end{figure}

    \begin{figure}[h]
        \begin{center}
        \includegraphics[width=0.7\textwidth]{PlotData/PION/Dalitz_maps/figures/Dalitz_map_pnn_xy_orders.pdf}
        \end{center}
        \caption{The same as in \fig{pion_map_E1E2_order} but for the double differential absorption rates
        $d^2 \Gamma_{pnn}/dxdy$.}
        \label{pion_map_xy_order}
    \end{figure}

    Following figures demonstrate the same results but from the different prospective.
    Namely in \fig{pion_GdEp} I show differential absorption rate $d\Gamma_{pnn} /d\text{E}_p$
    that is the same as in e.g. \fig{pion_map_E1E2_order}, but integrated over E$_n$.
    All the results are obtained with the most advanced setup (plane wave + rescattering, SN + 2N, 2NF+3NF).
    The left panel consists of the results with $\Lambda=\SI{450}{\mev}$ and each curve correspond 
    to a particular chiral order. Interesting region here is a maximum point which corresponds
    to a bottom part of the circles from \fig{pion_map_E1E2_order}. At the point of maximum
    \gls{n2lo} outstands from other results as its point of maximum is noticeably higher.
    At $\text{E}_p = \SI{0.92}{\mev}$ (maximum point) the value of \gls{n2lo} is
    \num{1.37} times larger than one from \gls{n4lo+} (\SI{3.44e+17}{fm.\s^{-1}}
    vs \SI{2.52e+17}{fm.\s^{-1}}) - the relative difference is \SI{31.1}{\percent}.
    At the same time, the relative difference between all the predictions except for \gls{n2lo}
    is \SI{8.3}{\percent}

    The right panel of the \fig{pion_GdEp} shows a cutoff dependance of the predictions obtained
    with the \gls{sms} chiral potential at \gls{n4lo+}. In this case maximum point is interesting as well.
    We see that predictions with $\Lambda=\SI{500}{\mev}$ and $\SI{550}{\mev}$ are quite close to each other:
    the relative difference between them at $\text{E}_p = \SI{0.92}{\mev}$ is only \SI{1.5}{\percent}.
    In turn the spread between $\Lambda=\SIlist{400;450;500}{\mev}$ is \SI{40}{\percent} (at the same point).
    This cutoff dependance is hidden looking at the colormaps 
    \fig{pion_map_E1E2_cutoff}, but from this prospective it is clearly presented.

    \begin{figure}[h]
        \begin{center}
        \includegraphics[width=0.9\textwidth]{PlotData/PION/Dalitz_maps/figures/3HE_dGdEp.pdf}
        \end{center}
        \caption{Differential absorption rate $d\Gamma_{pnn} /d\text{E}_p$ 
        as a function of the proton energy E$_p$ for the 
        $\pi^- + 3\text{He} \rightarrow p + n + n$ process.
        Left panel shows results obtained with \gls{n2lo} (green dashed line),
        \gls{n3lo} (blue dotted line), \gls{n4lo} (red dashed-double-dotted line)
        and \gls{n4lo+} (black solid line) chiral orders, and with $\Lambda = \SI{450}{\mev}$.
        The right panel includes results obtained with the \gls{n4lo+} \gls{sms} potential
        with different values of the $\Lambda$: \SI{400}{\mev} (red dashed-double-dotted line),
        $\Lambda$: \SI{450}{\mev} (black solid line),
        $\Lambda$: \SI{500}{\mev} (green dashed line line) and
        $\Lambda$: \SI{550}{\mev} (blue dotted line).
        All predictions were obtained with "FULL-(SN+2N)-(2NF+3NF)" setup.}
        \label{pion_GdEp}
    \end{figure}

    Similarly in \fig{pion_dGdEn} $d\Gamma_{pnn} /d\text{E}_n$ is presented. We observe similar trends
    which are also shown up at the extremum point which is around E$_n = \SI{66.9}{\mev}$ now.
    The difference between \gls{n2lo} and \gls{n4lo+} predictions at this point is \SI{29.0}{\percent}.
    The relative difference between all the predictions except for \gls{n2lo}
    is \SI{7.5}{\percent}.
    The cutoff predictions are also very similar for $\Lambda=\SIlist{500;550}{\mev}$ (the spread is \SI{1.6}{\percent})
    while all the rest predictions are quit distinguished - the spread is \SI{39.1}{\percent}.


    \begin{figure}[h]
        \begin{center}
        \includegraphics[width=0.9\textwidth]{PlotData/PION/Dalitz_maps/figures/3HE_dGdEn.pdf}
        \end{center}
        \caption{The same as in \fig{pion_GdEp} but for the differential absorption rate $d\Gamma_{pnn} /d\text{E}_n$
        as a function of the neuteron energy E$_n$.}
        \label{pion_dGdEn}
    \end{figure}

    Coming to the next figures \fig{pion_dGdEr} and \fig{pion_dGdphi} which show 1D dependance of the
    absorption rate on the Dalitz coordinates $r = \sqrt{x^2 + y^2}$  and $\phi = \arctan \frac{y}{x}$.
    The similar trend is preserved, namely chiral order figures show that \gls{n2lo} predictions
    outstand from all other predictions, and noticeable cutoff dependance is observed.

    In general we can conclude that predictions are converged starting from the \gls{n3lo} chiral order as
    most of the demonstrated results show that the difference between \gls{n3lo}, \gls{n4lo} and 
    \gls{n4lo+} is negligible. At the same time we observe a cutoff dependance where 
    predictions obtained with $\Lambda=\SIlist{500;550}{\mev}$ are very similar, but the spread with
    all the rest values is there. This nature of the cutoff dependance is also reflected in the total absorption 
    rate, presented in \fig{Gamma_pnn}. 

    \begin{figure}[h]
        \begin{center}
        \includegraphics[width=0.9\textwidth]{PlotData/PION/Dalitz_maps/figures/3HE_dGdr.pdf}
        \end{center}
        \caption{The same as in \fig{pion_GdEp} but for the differential absorption rate $d\Gamma_{pnn} /dr$
        as a function of the parameter $r$ of the Dalitz coordinates: $r = \sqrt{x^2 + y^2}$.}
        \label{pion_dGdEr}
    \end{figure}

    \begin{figure}[h]
        \begin{center}
        \includegraphics[width=0.9\textwidth]{PlotData/PION/Dalitz_maps/figures/3HE_dGdphi.pdf}
        \end{center}
        \caption{The same as in \fig{pion_dGdEr} but for the differential absorption rate $d\Gamma_{pnn} /d\phi$
        as a function of the azimuthal angle $\phi$ of the Dalitz coordinates: $\phi = \arctan \frac{y}{x}$.}
        \label{pion_dGdphi}
    \end{figure}


    \clearpage
    \subsection{$\pi^- + ^3\text{H} \rightarrow n + n + n$}

    In this subsection I will show results of calculations for
    the $\pi^- + ^3\text{H} \rightarrow n + n + n$ process.
    In this case we have only a three-body breakup as 
    no two-body configuration can be composed out of three neutrons.
    
    The total absorption rate $\pi^- + ^3\text{H} \rightarrow n + n + n$ is shown in \fig{Gamma_nnn}
    as a function on the chiral order of the \gls{sms} potential while each curve represents
    different cutoff values used to obtain the prediction. 
    THe most advanced configuration was used in this case, namely Plane wave plus rescattering part,
    both single- and two-nucleon currents and two-nucleon plus three-nucleon forces.

    Similarly to the process with $^3$He, we see that with each subsequent chiral order
    predictions become closer to each other, so cutoff dependency gets weaker.
    Also, the prediction with $\Lambda=\SI{550}{\mev}$ at \gls{n3lo} is
    strangely above the prediction with $\Lambda=\SI{500}{\mev}$.
    We can also notice, that at \gls{n4lo}
    predictions with cutoff values \SI{500}{\mev} and \SI{550}{\mev}
    are much closer to each other than to other values.

    \begin{figure}[h]
        \begin{center}
        \includegraphics[width=0.6\textwidth]{PlotData/PION/Dalitz_maps/figures/Gamma_nnn.pdf}
        \end{center}
        \caption{Absorption rate for $\pi^- + ^3\text{H} \rightarrow n + n + n$ reaction as a function
        of the chiral order with different values of the cutoff parameter $\Lambda$.
        Predictions were obtained with 3NF.}
        \label{Gamma_nnn}
    \end{figure}


    \begin{figure}[h]
        \begin{center}
        \includegraphics[width=0.7\textwidth]{PlotData/PION/Dalitz_maps/figures/Dalitz_map_nnn_E1E2_cutofs.pdf}
        \end{center}
        \caption{Intensity plots for the double differential absorption rates
        $d^2 \Gamma_{nnn}/dE_1dE_2$ for the $\pi^- + ^3\text{H} \rightarrow n + n + n$
        process, obtained using the SMS potential at \gls{n4lo+}
        with all contributions possible: plane wave + rescattering, SN + 2N, 2NF+3NF.
        Each panel present predictions obtained with different values of the cutoff parameter $\Lambda$:
        from \SI{400}{\mev} (upper left) to \SI{550}{\mev} (lower right).}
        \label{pion_nnn_E1E2_cutoff}
    \end{figure}

    \begin{figure}[h]
        \begin{center}
        \includegraphics[width=0.7\textwidth]{PlotData/PION/Dalitz_maps/figures/Dalitz_map_nnn_xy_cutofs.pdf}
        \end{center}
        \caption{The same as in \fig{pion_nnn_E1E2_cutoff} but for the double differential absorption rates
        $d^2 \Gamma_{nnn}/dxdy$.}
        \label{pion_nnn_xy_cutoff}
    \end{figure}


    \begin{figure}[h]
        \begin{center}
        \includegraphics[width=0.7\textwidth]{PlotData/PION/Dalitz_maps/figures/Dalitz_map_nnn_E1E2_orders.pdf}
        \end{center}
        \caption{Intensity plots for the double differential absorption rates
        $d^2 \Gamma_{nnn}/dE_1dE_2$ for the $\pi^- + ^3\text{H} \rightarrow n + n + n$
        process, obtained using the SMS potential at \gls{n4lo+}
        with all contributions possible: plane wave + rescattering, SN + 2N, 2NF+3NF.
        Each panel present predictions obtained with different chiral orders of the \gls{sms} potential:
        from \gls{n2lo} (upper left) to \gls{n4lo+} (lower right) and with $\Lambda = \SI{450}{\mev}$.}
        \label{pion_nnn_E1E2_order}
    \end{figure}


    \begin{figure}[h]
        \begin{center}
        \includegraphics[width=0.7\textwidth]{PlotData/PION/Dalitz_maps/figures/Dalitz_map_nnn_xy_orders.pdf}
        \end{center}
        \caption{The same as in \fig{pion_nnn_E1E2_order} but for the double differential absorption rates
        $d^2 \Gamma_{nnn}/dxdy$.}
        \label{pion_nnn_xy_order}
    \end{figure}


    \begin{figure}[h]
        \begin{center}
        \includegraphics[width=0.9\textwidth]{PlotData/PION/Dalitz_maps/figures/3H_dGdEn.pdf}
        \end{center}
        \caption{Differential absorption rate $d\Gamma_{nnn} /d\text{E}_n$ 
        as a function of the neuteron energy E$_n$ for the 
        $\pi^- + 3\text{H} \rightarrow n + n + n$ process.
        Left panel shows results obtained with \gls{n2lo} (green dashed line),
        \gls{n3lo} (blue dotted line), \gls{n4lo} (red dashed-double-dotted line)
        and \gls{n4lo+} (black solid line) chiral orders, and with $\Lambda = \SI{450}{\mev}$.
        The right panel includes results obtained with the \gls{n4lo+} \gls{sms} potential
        with different values of the $\Lambda$: \SI{400}{\mev} (red dashed-double-dotted line),
        $\Lambda$: \SI{450}{\mev} (black solid line),
        $\Lambda$: \SI{500}{\mev} (green dashed line line) and
        $\Lambda$: \SI{550}{\mev} (blue dotted line).
        All predictions were obtained with "FULL-(SN+2N)-(2NF+3NF)" setup.}
        \label{pion_dGdEn_3H}
    \end{figure}

    \begin{figure}[h]
        \begin{center}
        \includegraphics[width=0.9\textwidth]{PlotData/PION/Dalitz_maps/figures/3H_dGdr.pdf}
        \end{center}
        \caption{The same as in \fig{pion_dGdEn_3H} but for the differential absorption rate $d\Gamma_{nnn} /dr$
        as a function of the parameter $r$ of the Dalitz coordinates: $r = \sqrt{x^2 + y^2}$.}
        \label{pion_dGdr_3H}
    \end{figure}


    \begin{figure}[h]
        \begin{center}
        \includegraphics[width=0.9\textwidth]{PlotData/PION/Dalitz_maps/figures/3H_dGdphi.pdf}
        \end{center}
        \caption{The same as in \fig{pion_dGdr_3H} but for the differential absorption rate $d\Gamma_{nnn} /d\phi$
        as a function of the azimuthal angle $\phi$ of the Dalitz coordinates: $\phi = \arctan \frac{y}{x}$.}
        \label{pion_dGdphi_3H}
    \end{figure}
