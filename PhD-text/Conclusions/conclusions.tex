\chapter{Summary}

In this thesis I investigated rections of $^2$H, $^3$H and $^3$He photodisintegrations
as well as pion absorption by the same nuclei. In order to analyze these reactions
and calculate predictions for observables I used a chiral model of interaction
namely the most advanced \gls{sms} chiral potential up to \gls{n4lo+} order.
Results prepared with \gls{av18} potential have been shown as 
a reference point.

The first goal of this work was to investigate if the \gls{sms}
potential is converged with respect to the chiral order. It would then be
a hint whether the development of higher orders is required. In most of the results
we observed a very converged predictions which was confirmed by a small difference
between the last two chiral orders: \gls{n4lo} and \gls{n4lo+}. This
difference in most of regarded situations did not exceed a few percent.
Another evidence is a width of the truncation band for \gls{n4lo+} predictions.
For the deuteron photodisintegration process, observables have a maximum
width always below \SI{1}{\percent} for the regarded photon energies
($E_\gamma=\SI{30}{\mev}$ and $E_\gamma=\SI{100}{\mev}$).
The same trend is presented also for other regarded reactions as well.
This hints us towards a conclusion that further chiral orders would not
improve the predictions much and current potential is converged already.
This is also confirmed by the \gls{av18} predictions which are
always very close to \gls{n4lo+} (see e.g. \fig{T20_T21_100}, \fig{PY_30_100_vert} etc.).

The other interesting point in the investigated chiral potential is its dependance
on the value of the cutoff parameter $\Lambda$, the four values of which (\SIlist[list-units = single]{400;450;500;550}{\mev}) 
were investigated. We have observed that the relative spread of predictions 
with respect to the cutoff value is higher for the larger energies.
For example the spread of the differential cross section for $^3$He photodisintegration
at $E_\gamma=\SI{30}{\mev}$ at the charachteristic point (maximum) is around \SI{3}{\percent}
while at $E_\gamma=\SI{100}{\mev}$ it is three tiems larger - around \SI{3}{\percent}
(see \fig{CROSS_HE_EXCL_30}, \fig{CROSS_HE_EXCL_100} and discussions).
Nevertheless, usually for higher energies the difference betweeen the predictions
obtained with different values of $\Lambda$ is smaller than the difference with experimental
data (when it is available) which is evidently visible in \fig{Diff_cross_err}(b).

