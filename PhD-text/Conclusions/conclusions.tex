\chapter{Summary}

In this thesis, I investigated $^2$H, $^3$H and $^3$He photodisintegration processes
as well as pion absorption by the same nuclei. To analyze these reactions
and to calculate predictions for observables I used a chiral model of interaction
namely the most advanced \gls{sms} nucleon-nucleon chiral potential up to \gls{n4lo+} order
augmented by the consistently regularized three-nucleon force at \gls{n2lo}.
Results prepared with the semi-phenomenological \gls{av18} potential have been shown as 
a reference point. The current operator is restricted to the single-nucleon part only.
Both processes are studied in the momentum space.
The standard Lippmann-Schwinger equation is solved to get the $t$-operator and consequently
2N scattering state. For the three-nucleon processes, the formalism of Faddeev equations
has been applied.
I solved corresponding equations for Faddeev components both for the bound and 3N scattering states.
In that way, I include all initial state as well as final state interactions.
I am also able to test the importance of FSI by restricting components to only plane wave approximation.
The used formalism allows me to study not only the total cross section or capture rates
but also various differential cross sections and polarization
observables. The latter ones are very important to test the model and to compare with experimental data.
That also allows me to conclude on the sensitivity of various observables on studied effects and to 
pick up observables which after measurement could deliver the most valuable information.

The main goal of this work was to investigate the quality of currently available
predictions based on the semi-phenomenological \gls{sms}
interactions if applied to the studied here processes.
Such information is necessary due to expected two- and more- nucleon currents
at higher orders of chiral expansion, consistent with the \gls{sms} potential.
Various features of the model can be studied in that context.
Firstly, I investigated if the predictions based on the \gls{sms} interaction
are converged with respect to the chiral order.
It would then be a hint whether the development of higher-order contributions to the potential is required.
In most of the results, I observe very converged predictions since there is a small difference
between the last two investigated chiral orders: \gls{n4lo} and \gls{n4lo+}. This
difference in most of the regarded cases does not exceed a few percent.
Another piece of evidence is a width of the truncation band for \gls{n4lo+} predictions.
For the deuteron photodisintegration process, observables have a maximum
truncation error below \SI{1}{\percent} for the two studied photon energies:
$E_\gamma=\SI{30}{\mev}$ and $E_\gamma=\SI{100}{\mev}$.
The same trend is presented also for other regarded reactions as well.
This hints\tmp{?} us towards a conclusion that further chiral orders would not
improve the predictions much and the current model shows satisfactory convergence.
This is also confirmed by the \gls{av18} predictions which are
always very close to \gls{n4lo+} (see e.g. \fig{T20_T21_100}, \fig{PY_30_100_vert} etc.).

The other interesting point in the investigation of chiral potential is its dependence
on the value of its intrinsic cut-off parameter $\Lambda$, the four values of which (\SIlist[list-units = single]{400;450;500;550}{\mev}) 
were investigated. I have shown that the relative spread of predictions 
concerning the cutoff value is higher for the higher energies.
For example, the spread of the differential cross section for $^3$He photodisintegration
at $E_\gamma=\SI{30}{\mev}$ at the characteristic point (maximum) is around \SI{3}{\percent},
while at $E_\gamma=\SI{100}{\mev}$ it is three times larger - around \SI{3}{\percent}
(see \fig{CROSS_HE_EXCL_30}, \fig{CROSS_HE_EXCL_100} and discussions).
Nevertheless, usually for higher energies the difference between the predictions
obtained with different values of $\Lambda$ is smaller than the difference with experimental
data (when it is available) which is visible in \fig{Diff_cross_err}(b).

I have also studied the role of the various dynamical components of the model by checking 
how they influence the predicted values.
Namely, I compare predictions obtained with plane wave part only (first term in the \eq{lse_gen}),
with those taking into account the final state interaction,
as well as predictions with and without 2N current contributions (introduced via the Siegert approach).
For example, in the \fig{Diff_cross_pw_1nc} we see predictions for the deuteron photodisintegration 
cross section obtained 
without rescattering part, without 2N contributions and full predictions.
The contribution of rescattering processes is relatively small for the predictions at $E_\gamma = \SI{30}{\mev}$,
but is increasing with larger energies. The analysis of the relative difference at the specific 
$\theta_p$ value does not show this trend (the differences are \SIlist{10;7;4}{\percent} 
for $E_\gamma=\SIlist[list-units = single]{30;100;140}{\mev}$ respectively at $\theta_p=\SI{80}{\percent}$),
but we see that
at the lower energy predicted cross section values are qualitatively very similar and they differ
mainly around the maximum point.
In contrast, for the larger energies, predictions differ qualitatively, and the analyzed point 
depicts the region where the difference is relatively small.
The difference between the full predictions and the ones, where only 1NC was used is much bigger:
even for the lowest energy inclusion of two-body currents changes the cross section 
by around \SI{50}{\percent} and for the larger two it grows up to $\sim\SI{78}{\percent}$ .
Clearly, both rescattering part and 2NC contributions are very important and bring significant contributions.
Similar trend is visible also for other observables(see e.g. \fig{tensor_pw_1nc}) and processes 
(see Figs.\ref{pion_map_E1E2_cutoff}, \ref{pion_map_E1E2_cutoff_PW}, \ref{pion_map_E1E2_cutoff_1NC}
to compare contributions for the pion absorption on $^3$He).

That complex pattern reveals the interplay between interaction and the current operator, and
is one more recommendation after derivation fully consistent model, i.e. with consistent 2N
forces, 3N forces, and one-, two- and three-body currents.
Such a model must be applied only within the readable scheme to compute observables.
My work culminated in the preparation of such a scheme, both analytical and numerical
sides, and now we are ready to study more sophisticated Hamiltonians.

Giving mentioned above results, we can also conclude that the current version of 
the chiral \gls{sms} potential is of very high quality:
it rather does not require any additional development in the sense of adding higher chiral orders or regularization.
Contrary, the 2NC should be completely derived as it is expected to bring 
a significant contribution and thus improvements in our understanding of electromagnetic processes.

Among the studied processes and observables, I would like to point out the following 
main conclusions:

\begin{enumerate}
    \item The most 
\end{enumerate}

\tmp{Our Full model nicely describes data up to $\text{E}_\gamma = \SI{70}{\mev}$.}