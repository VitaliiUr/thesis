\chapter{Summary}

In this thesis I investigated $^2$H, $^3$H and $^3$He photodisintegrations
as well as pion absorption by the same nuclei. In order to analyze these reactions
and calculate predictions for observables I used a chiral model of interaction
namely the most advanced \gls{sms} chiral potential up to \gls{n4lo+} order.
Results prepared with \gls{av18} potential have been shown as 
a reference point.

The first goal of this work was to investigate if the \gls{sms}
potential is converged with respect to the chiral order. It would then be
a hint whether the development of higher orders is required. In most of the results
we observed a very converged predictions which was confirmed by a small difference
between the last two chiral orders: \gls{n4lo} and \gls{n4lo+}. This
difference in most of regarded situations did not exceed a few percent.
Another evidence is a width of the truncation band for \gls{n4lo+} predictions.
For the deuteron photodisintegration process, observables have a maximum
width always below \SI{1}{\percent} for the regarded photon energies
($E_\gamma=\SI{30}{\mev}$ and $E_\gamma=\SI{100}{\mev}$).
The same trend is presented also for other regarded reactions as well.
This hints us towards a conclusion that further chiral orders would not
improve the predictions much and current potential is converged already.
This is also confirmed by the \gls{av18} predictions which are
always very close to \gls{n4lo+} (see e.g. \fig{T20_T21_100}, \fig{PY_30_100_vert} etc.).

The other interesting point in the investigated chiral potential is its dependance
on the value of the cutoff parameter $\Lambda$, the four values of which (\SIlist[list-units = single]{400;450;500;550}{\mev}) 
were investigated. We have observed that the relative spread of predictions 
with respect to the cutoff value is higher for the larger energies.
For example the spread of the differential cross section for $^3$He photodisintegration
at $E_\gamma=\SI{30}{\mev}$ at the characteristic point (maximum) is around \SI{3}{\percent}
while at $E_\gamma=\SI{100}{\mev}$ it is three times larger - around \SI{3}{\percent}
(see \fig{CROSS_HE_EXCL_30}, \fig{CROSS_HE_EXCL_100} and discussions).
Nevertheless, usually for higher energies the difference between the predictions
obtained with different values of $\Lambda$ is smaller than the difference with experimental
data (when it is available) which is evidently visible in \fig{Diff_cross_err}(b).

We have also checked contributions from the various model components to the predicted values.
Namely, predictions obtained with plane wave part only (first term in the \eq{lse_gen}),
as well as predictions with and without 2N current contributions (obtained via the Siegert approach).
For example, in the \fig{Diff_cross_pw_1nc} we see predictions for the deuteron photodisintegration obtained 
without rescattering part(pink line), without 2N contributions (green line) and full predictions (blue line).
Evidently the rescattering part contribution is relatively small for the $E_\gamma = \SI{30}{\mev}$,
but is increasing with larger energies. The analysis of the relative difference at the specific 
$\theta_p$ value does not show this trend (the differences are \SIlist{10;7;4}{\percent} 
for $E_\gamma=\SIlist[list-units = single]{30;100;140}{\mev}$ respectively at $\theta_p=\SI{80}{\percent}$), but we see that
at the lower energy curves are qualitatively very similar and the difference is mainly around the maximum point(which is close to the analyzed point).
In contrast, for the larger energies, curves differ qualitatively and analyzed point is rather depicting the region
were the difference is minimal.
The difference between the full predictions and the ones, where only 1NC was used is much higher:
even for the lowest energy it is around \SI{50}{\percent} and for the larger two it grows
up to $\sim\SI{78}{\percent}$ (at the same angle value).
Clearly, both rescattering part and 2NC contributions are very important and bring significant contribution.
The latter one is taken into account not completely, since there is no proper 2NC current 
available yet, but even using the Siegert approach proves that this part is very important in described process.
Similar trend is visible at other observables(see e.g. \fig{tensor_pw_1nc}) and other processes 
(see Figs.\ref{pion_map_E1E2_cutoff}, \ref{pion_map_E1E2_cutoff_PW}, \ref{pion_map_E1E2_cutoff_1NC}
to compare contributions for the pion absorption at $^3$He).

Giving mentioned above results, we can conclude that current version of the chiral potential is good:
it rather does not require any additional development in a sens of chiral orders or regularization,
but 2NC is expected to bring a large improvements to the predictions.