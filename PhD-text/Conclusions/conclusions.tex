\chapter{Summary}

In this thesis, I investigated the $^2$H, $^3$H and $^3$He photodisintegration processes
as well as the pion absorption by the same nuclei. To analyze these reactions
and to calculate predictions for observables I used a chiral model of interaction
- the \gls{sms} nucleon-nucleon potential up to \gls{n4lo+} order of the chiral expansion
augmented by the consistently regularized three-nucleon force at \gls{n2lo}.
Results prepared with the semi-phenomenological \gls{av18} potential have been shown as 
a reference point. 
In case of the photodisintegration ractions, the electromagnetic current is taken
as a single nucleon current supplemented by the Siegert theorem.
For the absorption processes the leading order two-nucleon currents are included explicitly
on the top of the single nucleon operators.
% The current operator is restricted to the single-nucleon part only.
Both processes are studied in momentum space.
The standard Lippmann-Schwinger equation is solved to get the $t$-operator and consequently
2N scattering state. For the three-nucleon processes, the formalism of Faddeev equations
has been applied.
I solved corresponding equations for Faddeev components both for the bound and 3N scattering states.
In that way, I include all initial state as well as final state interactions.
I am also able to test the importance of FSI by restricting computations to plane wave approximation only.
The used formalism allows me to study not only the total cross section or total absorption rates
but also various differential cross sections and polarization
observables. The latter ones are very important to test the model.
That in turn allows me to conclude on the sensitivity of various observables on studied effects and to 
pick up observables which after measurement could deliver the most valuable information.

The main goal of this work was to investigate the quality of currently available
predictions based on the semi-phenomenological \gls{sms}
interactions if applied to the studied here processes.
Such information is necessary due to expected two- and more- nucleon currents
at higher orders of chiral expansion, consistent with the \gls{sms} potential.
Various features of the model can be studied in that context.
Firstly, I investigated if the predictions based on the \gls{sms} interaction
are converged with respect to the chiral order.
It would then be a hint whether the development of higher-order contributions to the potential is required.
In most of the results, I observe very converged predictions what I conclude from a small difference
between the last two investigated chiral orders: \gls{n4lo} and \gls{n4lo+}. This
difference in most of the regarded cases does not exceed a few percent.
Another piece of evidence is the width of the truncation bands for \gls{n4lo+} predictions.
For the deuteron photodisintegration process, observables have a maximum
truncation error below \SI{1}{\percent} for the two studied photon energies:
$E_\gamma=\SI{30}{\mev}$ and $E_\gamma=\SI{100}{\mev}$.
The same trend is present also for other regarded reactions.
This leads us towards a conclusion that further chiral orders would not
improve the predictions much and the current model shows satisfactory convergence.
This is also confirmed by the \gls{av18} predictions which are
always very close to \gls{n4lo+} (see e.g. \fig{T20_T21_100}, \fig{PY_30_100_vert} etc.).

The other interesting point in the investigation of chiral potential is the dependence of its predictions
on the value of the potential's intrinsic cut-off parameter $\Lambda$, the four values of which (\SIlist[list-units = single]{400;450;500;550}{\mev}) 
were investigated. I have shown that the relative spread of predictions 
concerning the cutoff value is higher for the higher energies.
For example, the spread of the differential cross section for $^3$He photodisintegration
at $E_\gamma=\SI{30}{\mev}$ at the characteristic point (maximum) is around \SI{3}{\percent},
while at $E_\gamma=\SI{100}{\mev}$ it is three times larger - around \SI{9}{\percent}
(see \fig{CROSS_HE_EXCL_30}, \fig{CROSS_HE_EXCL_100} and discussions).
Nevertheless, usually for higher energies the difference between the predictions
obtained with different values of $\Lambda$ is smaller than the difference with experimental
data (when it is available) which is visible e.g. in \fig{Diff_cross_err}(b).

I have also studied the role of the various dynamical components of the model by checking 
how they influence the predicted values.
Namely, I compared predictions obtained with plane wave part only (first term in the \eq{lse_gen}),
with those taking into account the final state interaction.
I also investigated the role of two-body contributions to nuclear current or absorption operator
by performing computations that take or do not take two-body operators into account.
% as well as predictions with and without 2N current contributions (introduced via the Siegert approach).
For example, in the \fig{Diff_cross_pw_1nc} there are predictions for the deuteron photodisintegration 
cross section obtained 
without rescattering part, without 2N contributions to the current and full predictions.
The contribution of rescattering processes is relatively small for the predictions at $E_\gamma = \SI{30}{\mev}$,
but grows with increasing energies. In some cases the analysis of the relative difference at the specific 
$\theta_p$ value does not show this trend (the differences are \SIlist{10;7;4}{\percent} 
for $E_\gamma=\SIlist[list-units = single]{30;100;140}{\mev}$ respectively at $\theta_p=\SI{80}{\percent}$),
but we see that
at the lower energy all predicted cross section values are qualitatively very similar and they differ
mainly around the maximum point.
In contrast, for the larger energies, predictions differ qualitatively, and the analyzed point 
depicts the region where the difference is relatively small.
The discrepancy between the full predictions and the ones, where only 1NC was used is much bigger:
even for the lowest energy inclusion of two-body currents changes the cross section 
by around \SI{50}{\percent} and for the larger two it grows up to $\sim\SI{78}{\percent}$ .
Clearly, both rescattering part and 2NC contributions are very important and bring significant contributions.
Similar trend is visible also for other observables(see e.g. \fig{tensor_pw_1nc}) and processes 
(see Figs.\ref{pion_map_E1E2_cutoff}, \ref{pion_map_E1E2_cutoff_PW}, \ref{pion_map_E1E2_cutoff_1NC}
to compare contributions for the pion absorption on $^3$He).

That complex pattern reveals the interplay between interaction and the current operator, and
is one more recommendation after derivation fully consistent model, i.e. with consistent 2N
forces, 3N forces, and one-, two- and three-body currents.
Such a model must be applied only within the reliable scheme to compute observables.
My work culminated in the preparation of such a scheme, for both analytical and numerical
sides, and now we are ready to study more sophisticated Hamiltonians.

Giving mentioned above results, I conclude that the current version of 
the chiral \gls{sms} potential is of very high quality:
it rather does not require any additional development in the sense of adding higher chiral orders or regularization.
Contrary, for both types of processes the 2NC should be completely derived as it is expected to bring 
a significant contribution and thus improvements in our understanding of processes
with photons or pions.

Among \tmp{various my results}, I would like to highlight the following 
conclusions:

\begin{enumerate}
    \item The chiral \gls{sms} nucleon-nucleon interaction at higher orders of chiral expansion (above \gls{n2lo})
    leads to very stable behavior of predictions for the photodisintegraion and pion absorption processes.
    That confirms previous findings for processes in pure nucleonic systems.
    I have not observed any strange or warning pattern for observables which could be related to
    deficiencies in the nucleon-nucleon interaction.
    In consequence, I conclude, that NN force is known with sufficient accuracy to be used in studies of 
    nuclear processes with external \tmp{sorce}.
    \item 2NC is very important for the regarded electromagnetic processes and observables. Even including it via the Siegert theorem allows seeing significant improvements (e.g. Figs.~ \ref{TOTAL_CROSS}, \ref{T20_T21_30} and \ref{T20_T21_100} for the deuteron photodisintegration process).
    \item The same is true for the pion absorption process. I took into account 2NC at leading order and the difference between
    predictions shown in \fig{pion_map_E1E2_cutoff} and \fig{pion_map_E1E2_cutoff_1NC} (\gls{snc} only) proofs its importance.
    However, the discrepancy with the existing data for the total capture rates calls for more advanced model of two-
    and three-body absorption operator.
    \item The importance of 3NF is less obvious looking at my results.
    For example, for $^3$He photodisintegration in \fig{CROSS_HE_EXCL_75_75_75_105} 3NF makes cutoff dependence weaker,
    but the difference between predictions with and without 3NF is not very big.
    Thus the investigated here processes are not best field to study \tmp{details}
    of the three-nucleon interaction.
    The only exception is pion absorption on $^3$H, see below.
    \item For $\pi + ^3$H there are 3 neutrons in the final state. It is one of very few such situations and it allows us to investigate the neutron-neutron nucleon force. Moreover, it even allows us to investigate the neutron-neutron-neutron force. Such a situation is hardly reproducible in experiments as neutrons are hard to detect, but we can still analyze theoretical predictions and make conclusions. It is very important for the construction of the complete model of nucleon-nucleon 2N and 3N interactions.
    \item \tmp{Write what gives more uncertainty: cut-off or order}
    \item Investigation of the differential cross section is beneficial compared to the total cross sections as it allows us to see smaller details of the reaction mechanisms and the model in a sense of convergence and cutoff dependence. One may observe the reason for particular discrepancies.
    % (e.g. some singularity point which causes computational problems).
    It is also less computationally expensive as the total cross section is obtained via integration of the differential cross section through the whole region.
    Thus, while exclusive or semi-inclusive experiments are much harder to do than the measurement of the total
    cross sections or absorption rates, the experimental efforts should focus on such types of measurements in the future. 
    \item It would be interesting to check experimentally if theoretical uncertainties appearing at some configurations would be also reflected in the data. For example, we see that the kinematical configuration present in \fig{CROSS_HE_EXCL_30} is less sensitive to the model parameters than the one in \fig{CROSS_HE_EXCL_75_75_75_105}. So measuring the data for $^3$He gives more possibilities to test the model as lots of configurations for the differential cross section are possible.
    \item In my opinion, for the pion absorption it would be interesting to have a measurement data for FSI(nn) region. As we do not have a fully consistent 2NC, it would be beneficial to take into account experimental data when analyzing predictions obtained by approximations or (in future) by fully consistent 2NC. Comparing results from  \fig{pion_map_E1E2_cutoff} and \fig{pion_map_E1E2_cutoff_PW} we see a huge difference in the region around $E_1=\SI{85}{\mev}, E_2 = \SI{21}{\mev}$, where the FSI(nn) is.
    \item Our Full model nicely describes data up to $\text{E}_\gamma = \sim\SI{70}{\mev}$. It is ot a strict value, but the general trend is that for higher energies the difference between the model and the data is growing. For example, predictions for the total cross section and the tensor analyzing powers for the deuteron photodisintegraion nicely describe experimental data even up to $E_\gamma = \SI{100}{\mev}$ (\fig{TOTAL_CROSS}, \fig{T20_vs_en}), but photon asymmetry starts deteriorating even after $E_\gamma = \SI{40}{\mev}$ (\fig{asymmetry_90deg}).
    \item \tmp{is it better to do measurement on $^2$H, $^3$He or $^3$H?}
    \item \tmp{The most sensitive to FSI are for (observables, energies)}
\end{enumerate}