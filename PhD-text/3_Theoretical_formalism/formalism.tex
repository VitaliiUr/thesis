\chapter{Formalism and numerical methods}
\label{sec:formalism}

Even though the deuteron problem was solved a long time ago, I will describe it briefly 
to introduce the notation and formalism. 
With that, for more complex 3N cases only a slight 
extension will be needed.

To calculate any observable for the deuteron photodisintegration,
one has to find a nuclear matrix elements:

% \begin{equation}
%     N^\mu = \langle \Psi_f \vec{P_f} \mid J^\mu(0) \mid \Psi_i \vec{P_i} \rangle =
%     \langle p' (l's')j'm_j't'm_t' \vec{P_f} \mid J^\mu \mid \phi_d m_d \vec{P}_i \rangle, 
%     \label{main}
% \end{equation}
\begin{equation}
    N^\mu = \langle \Psi_{final} \mid J^\mu \mid \Psi_{initial} \rangle, 
    \label{main}
\end{equation}
with the two-nucleon wave function of the initial state $\Psi_{initial}  = \Psi_{deuteron}$;
the two-nucleon wave function of the final scattering state $\Psi_{final}$ 
and a four-vector current operator $J^\mu$ which acts between initial and final 
two-nucleon states. 
In case of $^3$He or $^3$H photodisintegration analogous nuclear matrix element
$\langle \Psi_{final} \mid J^\mu \mid \Psi_{initial}$ has to be found, but now 
$\ket{\Psi_{initial}}$ is either $^3$He or $^3$H bound state and $\bra{\Psi_{}final}$
describes a 3N scattering state with all three nucleons unbound after reaction
or Nd scattering state with the pair nucleon-deuteron in the final state.
In the following, I describe how to get these quantities.


% $\vec{P_i}$($\vec{P_f}$) is an initial (final) total 2N m\vec{p}omentum.

% One can introduce relative and total momenta for 2 nucleons:

% \begin{eqnarray}
    %     \vec{p} &=& \frac{1}{2} (\vec{p}_1 - \vec{p}_2)\\
    %     \vec{\mathcal{P}} &=& \vec{p}_1 + \vec{p}_2\\
    %     \pvec{p} &=& \frac{1}{2} (\pvec{p}_1^\prime - \pvec{p}_2^\prime)\\
    %     \pvec{\mathcal{P}}^\prime &=& \pvec{p}_1^\prime + \pvec{p}_2^\prime,
    % \end{eqnarray}
    % where $\vec{p}_1$($\pvec{p}_1^\prime$) and $\vec{p}_2$($\pvec{p}_2^\prime$) are
    % initial(final) momenta of the first and second nucleons.
\section{2N bound state}
    \label{sec:deut_bound}

    Let's find a deuteron bound state wave function $\ket{\phi_d}$. 
    The time-independent Schr\"{o}dinger equation for two particles is expressed as:

    \begin{equation}
        (H_0 + V) \ket{\phi_d}  = E_d \ket{\phi_d},
        \label{shrod_bound}
    \end{equation}
    with a kinetic energy $H_0$ and potential $V$. 
    The kinetic energy $H_0$ can be represented in terms of  relative and total momenta
    of the particles:

    \begin{equation}
        H_0 = \frac{\vec{p}_1^2}{2m_1} + \frac{\vec{p}_2^2}{2m_2} = 
        \frac{\vec{p}^2}{2\mu} + \frac{\vec{\mathcal{P}}^2}{2M}, 
    \end{equation}
    where the relative and total momenta are defined as follows:

    \begin{eqnarray}
        \vec{p} &=& \frac{(m_1\vec{p}_1 - m_2\vec{p}_2)}{m_1 + m_2},\\
        \vec{\mathcal{P}} &=& \vec{p}_1 + \vec{p}_2,
    \end{eqnarray}
    where $M = m_1 + m_2$ is a total mass, $\mu = \frac{m_1m_2}{M}$ is a reduced mass of two nucleons and 
    $\vec{p_i}$ is the momentum of i-th particle.

    Potential $V$ is assumed to depend on the relative degrees of freedom only, so
    \eq{shrod_bound} may be decomposed into two separate equations:

    \begin{eqnarray}
        &\frac{\vec{p}^2}{2\mu} \langle \vec{p} \mid \Psi_{int} \rangle +
        \langle \vec{p} \mid V \mid \Psi_{int} \rangle = 
        (E_d - E)\langle \vec{p} \mid \Psi_{int} \rangle \label{se1}\\
        &\frac{\vec{\mathcal{P}}^2}{2M} \braket{ \vec{\mathcal{P}} }{ \Psi } = 
        E\braket{ \vec{\mathcal{P}} }{ \Psi } \label{se2},
    \end{eqnarray}
    with $\matrixel{\vec{p},\vec{\mathcal{P}}}{H_0}{\phi_d} = \frac{\vec{p}^2}{2\mu} \braket{\vec{p}}{\Psi_{int}} +
    \frac{\vec{\mathcal{P}}^2}{2M} \braket {\mathcal{P}} {\Psi} $. So $\Psi$ is a component 
    of the total wave function, which reflects a deuteron as a single object with momentum $\vec{\mathcal{P}}$
    while $\Psi_{int}$ is an internal wave function describing the interaction between nucleons.
    Basis state $\ket{\vec{p}}$  obeys a completeness
    equation:

    \begin{equation}
        \int d^3\vec{p} \ket{\vec{p}} \bra{\vec{p}}   = \mathbb{1}.
        \label{completness}
    \end{equation}

    \eq{se1} is basically the Schr\"{o}dinger equation for a single particle with mass $\mu$
    in potential $V$ 
    and \eq{se2} can be regarded as a Schr\"{o}dinger equation for particle with mass $M$ in 
    a free motion. Assuming that deuteron is at rest ($E = 0$) we can stick 
    to the Eq.(\ref{se1}) only. Using completeness relation (\ref{completness}) we get:

    \begin{equation}
        \frac{\vec{p}^2}{2\mu} \langle \vec{p} \mid \Psi_{int} \rangle +
        \int d \pvec{p} \langle \vec{p} \mid V \mid \pvec{p} \rangle
        \langle \pvec{p} \mid \Psi_{int} \rangle = 
        E_d \langle \vec{p} \mid \Psi_{int} \rangle
        \label{shrod_old}
    \end{equation}

    Working in 3-dimensional space (i.e. directly with $\vec{p}$ vectors) is difficult, especially numerically,
    so I follow a standard path and introduce the partial-wave decomposed representation (PWD) 
    of the momentum state, adding spin and isospin degrees of freedom in the following form:

    \begin{equation}
        \ket{p \alpha} \equiv \ket{p (ls) j m_j}  \ket{t m_t},
        \label{pwmain}
    \end{equation}
    where quantum numbers l, s, j, t are orbital angular momentum, total spin,
    total angular momentum and total isospin respectively. $m_j$ and $m_t$ are 
    total angular momentum and isospin projections, respectively.


    States $\ket{p (ls) j m_j}$ can be further decomposed to 
    the more basic states than it is in (\ref{pwmain}), separating spin part as 
    
    \begin{equation}
        \ket{p (ls) j m_j} = \sum_{m_l} c(lsj;m_l, m_j\!-\!m_l, m_j) \ket{p l m_l}
        \ket{s~m_j\!-\!m_l}.
        \label{full_decomp}
    \end{equation}

    Spin(isospin) states can be further represented via single-nucleon spin(isospin) states:

    \begin{equation}
        \ket{s m_s} = \sum_{m_1} c(\frac{1}{2}\frac{1}{2}s;m_1, m_s\!-\!m_1, m_s)
        \ket{\frac{1}{2} m_1}
        \ket{\frac{1}{2} m_s\!-\!m_1},
        \label{spin_decomp}
    \end{equation}

    \begin{equation}
        \ket{t m_t} = \sum_{\nu_1} c(\frac{1}{2}\frac{1}{2}t;\nu_1, m_t\!-\!\nu_1, m_t)
        \ket{\frac{1}{2} \nu_1}
        \ket{\frac{1}{2} m_t\!-\!\nu_1}.
        \label{isospin_decomp}
    \end{equation}

    In Eqs.(\ref{full_decomp}) -(\ref{isospin_decomp}),  $c(...)$ are Clebsh-Gordon coefficients.
    Nucleons are spin $\frac{1}{2}$ particles, and we also treat proton and neutron as 
    the same particle in different 
    isospin states, using convention in which isospin $\nu_1 = \frac{1}{2}$ stands for proton and $\nu_1 = -\frac{1}{2}$ is for neutron.

    The states $\mid p l m_l \rangle$ from Eq.(\ref{full_decomp}) are orthogonal
    
    \begin{equation}
        \langle p^\prime l^\prime m_l^\prime \mid p l m_l \rangle = 
        \frac{\delta(p - p^\prime)}{p^2} \delta_{ll^\prime}\delta_{m_l m_l^\prime}
    \end{equation}
    and satisfy the completeness relation:

    \begin{equation}
        \sum_{l=0}^\infty \sum_{m_l=-l}^l \int dp p^2 \mid plm_l \rangle \langle plm_l \mid = \mathbb{1}
    \end{equation}


    Projection of $\bra{\pvec{p}}$ states to $\ket{plm_l}$ leads to

    \begin{equation}
        \braket{\pvec{p}}{plm_l} = 
        \frac{\delta ( \abs{\pvec{p}} - p)}{p^2} Y _{l m_l}(\hat{p}^\prime),
    \end{equation}
    where $Y _{l m_l}(\hat{p}^\prime)$ is a spherical harmonic and 'hat' denotes a unit vector $\hat{X}$ in 
    direction of $\vec{X}$. Thus for the momentum vector:

    \begin{equation}
        \vec{p} \equiv |\vec{p}| \hat{p} \equiv p \hat{p}. 
        \label{hat}
    \end{equation}

    Nucleons are fermions so exchanging them leads to antisymmetry of the
    wave function. In PWD it results in additional requirements on allowed quantum numbers which
    is:

    \begin{equation}
        (-1)^{l+s+t} = -1.
        \label{parity}
    \end{equation}

    In general, nuclear NN force conserves spin, parity, and charge so

    \begin{equation}
        \matrixel*{p^\prime\alpha^\prime}{V}{p\alpha} = \delta_{jj'}\delta_{mm'}\delta_{tt'}\delta_{m_tm_{t'}}
        \delta_{ss'}V^{sjtm_t}_{l'l}(p',p)
        \label{conservation}
    \end{equation}
    which introduces restrictions for particular sets of quantum numbers $\alpha$ and $\alpha^\prime$.
    Strong interaction allows for change of the orbital angular momenta $l = j \pm 1,~l'=j'\pm1$.
    The channels, in which  $l \neq l'$ is allowed,
    are called coupled channels and for the deuteron bound state 
    one can find only one such PWD state combination:
    two coupled channels 
    which are commonly denoted as $^3S_1$ and $^3D_1$ (the naming stands for $^{2s+1}l_j$). They correspond 
    to $l=0$ and $l=2$ respectively (with $s = j = 1$ and $t = m_t = 0$). 
    I will denote wave functions for these channels as $\phi_l(p)$ with $l=0,2$, such that:

    % Taking into account Eq.(\ref{parity}), one can find only one possible PWD state combination for 
    % the deuteron bound state(under ... experimental evidence): 2 coupled channels for l=0,2; s=1; j=1 and $t = m_t = 0$. 
    % These 2 channels are usually denoted as $^3S_1$ and $^3D_1$ and
    % corresponding wave functions are $\phi_0(p)$ and $\phi_2(p)$:
    
    \begin{equation}
        \phi_l(p) = \bra{p (ls) j m_d}\braket{t m_t}{\Psi_{int}} = \bra{p(l1)1m_d} \braket{00}{\Psi_{int}}; l=0,2.
        \label{deut_waves}
    \end{equation}

    In that new basis Eq.(\ref{shrod_old}) takes a form of two coupled equations:

    \begin{equation}
        \frac{\vec{p}^2}{2\mu} \phi_l(p) +
        \sum_{l^\prime =0,2} \int d p^\prime p^{\prime 2} 
        \bra{p(l1)1m_d} \matrixel{00}{V}{00} \ket{p^\prime(l^\prime1)1m_d}
        % \langle plm_l \mid V \mid p^\prime l^\prime m_l^\prime  \rangle
        \phi_{l^\prime}(p^\prime) = 
        E_d \phi_l(p),
        \label{integral}
    \end{equation}
    for $l=0,2$. In case one does not have a matrix elements for the potential 
    $\langle plm_l \mid V \mid p^\prime l^\prime m_l^\prime  \rangle$ in analytical form,
    but only numerical values for some grid of points are given, 
    there is still one complication in the Eq.(\ref{integral}) - integration, which has to be discretized.
    In order to get rid of the integral I use a Gaussian quadrature 
    method of numerical integration \cite{jacobi1826ueber}.
    It allows me to replace an integral by the weighted sum:
        $\int_a^b f(x)dx = \sum_{i=1}^N \omega_i f(x_i)$.
    In current work I used $N=72$ points in the interval from $0$ to \SI{50}{fm^{-1}}. 
    Using this method, Eq.(\ref{integral}) becomes  

    
    \begin{equation}
        \frac{p_i^2}{2\mu} \phi_l(p_i) +
        \sum_{l^\prime =0,2}\sum_{j =0}^{N}  \omega_j p^{2}_j 
        \bra{p_i(l1)1m_d} \matrixel{00}{V}{00} \ket{p_j(l^\prime1)1m_d}
        % \langle p_jlm_l \mid V \mid p^\prime_j l^\prime m_l^\prime  \rangle
        \phi_{l^\prime}(p_j) = 
        E_d \phi_l(p_i).
        \label{integral2}
    \end{equation}

    In practical computations, the same grid points $p_i$ are used for $p$ and $p^\prime$ in order to
    optimize computational time. 
    I solve this equation as an eigenvalue problem $M\Psi = E_d \Psi$ and
    in that way
    find simultaneously wave function values in a grid of $p$ points and the binding energy $E_d$. 
    For example, the binding energy $E_d$ calculated with the \gls{sms} potential at different chiral orders 
    is presented in Fig.~\ref{bind}.

    \begin{figure}[h]
        \begin{center}
            \includegraphics[width=0.65\textwidth]{Figures_De/Binding_energy.pdf}
        \end{center}
        \caption{Deuteron binding energy calculated using the chiral \gls{sms} potential
        at different chiral orders as a mean value over all cutoffs.
        % Error bands represent a spread the calculated binding energy with respect to
        % the cutoff parameter $\Lambda$ (minimal and maximal values).
        Experimental data is taken from \cite{VANDERLEUN1982261}.}
        \label{bind}
    \end{figure}

    An example of such wave functions is demonstrated in \fig{wave_func}. The left panel demonstrates
    a wave function for $l=0$ - $^3S_1$ while the right one - for $l=2$ - $^3D_1$. Both 
    plots consist of the curves for different cutoff values and using the chiral \gls{sms} potential at \gls{n4lo+}.
    The small deviation between lines shows that cutoff dependence is rather weak at this stage
    but further discrepancies connected to the value of $\Lambda$ may appear in other components
    of nuclear matrix elements.  

    \begin{figure}[h]
        \begin{center}
            \includegraphics[width=0.85\textwidth]{Figures_De/Wave_function_cutoff.pdf}
        \end{center}
        \caption{The deuteron wave function $\phi_l$ for l=0 ($^3S_1$ partial wave)(left) and l=2 ($^3D_1$) (right).
        Each curve shows results obtained with different values of the cutoff parameter $\Lambda$. 
        The chiral \gls{sms} potential at \gls{n4lo+} was used.}
        \label{wave_func}
    \end{figure}

\section{2N scattering state}

% \subsection{The Lippmann-Schwinger equation}
    % Let us consider a two-nucleon scattering state $\mid \Psi _f \rangle$, It fulfills
    % a Schr\"{o}dinger equation

    % \begin{equation}
    %     (H_0 + V) \mid \Psi _f \rangle = E \mid \Psi _f \rangle,
    %     \label{schrod_scatt}
    % \end{equation}
    % with $H_0 = \frac{\vec{p}^2}{m}$ and $E > 0$.

    % Solution to the Eq.(\ref{schrod_scatt}) can be presented in the form:

    % \begin{equation}
    %     \mid \Psi _f \rangle = \mid \Psi _0 \rangle + \frac{1}{E - H_0 + i \epsilon} V \mid \Psi _f \rangle
    % \end{equation}

    I work in the time-independent formulation of the scattering process.
    In such a case the Hamiltonian is:

    \begin{equation}
        H = H_0 + V,
    \end{equation}
    where again $H_0$ is a kinetic energy operator, $H_0 = \frac{\vec{p}^2}{2m}$, 
    and $V$ is a nucleon-nucleon interaction.
    For a free particle motion, $V$ will be absent and we will denote a corresponding energy eigenstate as
    a free particle state $\ket{\vec{p}\,}$.
    The scattering state $\ket{\psi}$ fulfills similar Schr\"{o}dinger equation as 
    $\ket{\vec{p}\,}$, with the same energy eigenvalue,
    but with the presence of the potential:
    
    % , the eigenstate will differ from $\ket{\phi_l}$,
    % but in case of elastic scattering (which we are interested in) the energy eigenvalue $E$ should be the same.

    % So these two states fulfill Schr\"{o}dinger equations for such scattering process:

    \begin{equation}
        \begin{cases}
            H_0 \ket{\vec{p}\,} &= E \ket{\vec{p}\,}, \\
            (H_0 + V) \ket{\psi} &= E \ket{\psi}.
        \end{cases}
        \label{system}
    \end{equation}

    We are interested in a solution for \eq{system}, so that 
    $\ket{\psi} \rightarrow \ket{\vec{p}\,}$ as $V \rightarrow 0$
    and both $\ket{\psi}$ and $\ket{\vec{p}\,}$ have the same energy eigenvalues E.
    As we have scattering process, the energy spectra for both operators $H_0$ and $H_0 + V$
    are continuous. 

    From \eq{system} it follows that

    \begin{equation}
        \ket{\psi} = \frac{1}{E - H_0}V \ket{\psi} +  \ket{\vec{p}\,},
        \label{psieq}
    \end{equation}
    % where $\mid \vec{p} \rangle$ is a solution to the
    % free-particle Schr\"{o}dinger equation
    % \begin{equation}
    %     H_0 \ket{\vec{p}} =  E  \ket{\vec{p}},
    % \end{equation}
    % with same energy eigenvalue\cite{Sakurai}.
    % was added artificially in order to satisfy a criterion mentioned above 
    % and following the logic from . 
    % In addition, it 
    which guarantees that
     application of the operator $(E -H_0)$ to the 
    \eq{psieq} results in the second equation from the set (\ref{system}).
    % Also, \eq{psieq} for $V \rightarrow 0$ becomes $\mid \psi \rangle = \mid \vec{p} \rangle$


    In order to deal with a singular operator $\frac{1}{E - H_0}$ in \eq{psieq}, the well-known
    technique is used to make such an operator complex by adding a small imaginary number to the denominator
    so \eq{psieq} becomes
    % making it  $\frac{1}{E + i\epsilon - H_0}$.

    \begin{equation}
        \ket{\psi} = G_0(E \pm i\epsilon)V \mid \psi \rangle +  \mid \vec{p} \rangle,
        \label{lse}
    \end{equation}
    where $G_0$ is a free propagator:

    \begin{equation}
        G_0(z) = \frac{1}{z - H_0}.
        \label{g0}
    \end{equation}

    Solution with $G_0(E - i\epsilon)$ corresponds to the incoming spherical wave,
    while $G_0(E + i\epsilon)$ - to the outgoing one. Since we are interested in the final scattering
    state, only the $(+)$ sign survives.
    
    Eq.~(\ref{lse}) is known as the Lippmann-Schwinger equation (LSE).
    Defining the transition operator $t$:

    \begin{equation}
        t \ket{\vec{p}} = V \ket{\psi}
        \label{t-op}
    \end{equation}
    we can rewrite it as 

    \begin{equation}
        \ket{\psi} = (1 + G_0(E + i\epsilon) t)  \ket {\vec{p}}.
        \label{psi_toper}
    \end{equation}

    With substitution of \eq{lse} into \eq{t-op} we find
    an explicit form of the $t$ operator:

    \begin{eqnarray}
        t \ket{\vec{p}} = V G_0(E + i\epsilon)V \mid \psi \rangle +  V \mid \vec{p} \rangle = \nonumber\\
        = V G_0(E + i\epsilon) t \ket{\vec{p}} +  V \mid \vec{p} \rangle
        \label{top2}
    \end{eqnarray}

    Getting rid off the initial state $\ket{\vec{p}}$ in the \eq{top2} we can get the LSE
    for the transition operator in the iterative operator form:

    \begin{eqnarray}
        t = V + V G_0 t = \nonumber\\
        = V + V G_0 V + V G_0 V G_0V + ...,
        \label{lse_gen}
    \end{eqnarray}
    which constitutes an infinite series of subsequent NN interactions and free propagators of nucleons.

    In the partial-wave representation, the LSE \eq{top2} expresses as:
    \begin{multline}
        \bra{p^\prime (l^\prime s^\prime)j'm_{j'}}\matrixel{t' m_{t'}}{t(E)}{t m_t}\ket{p (l s)jm_{j}} = 
        \bra{p^\prime (l^\prime s^\prime)j'm_{j'}}\matrixel{t' m_{t'}}{V}{t m_t}\ket{p (l s)jm_{j}} + \\
        +\sum_{\alpha^{\prime\prime}} \int_0^\infty dp^{\prime \prime} p^{\prime \prime 2}
        \bra{p^\prime (l^\prime s^\prime)j'm_{j'}}\matrixel{t' m_{t'}}{V}
        {t'' m_{t''}}\ket{p'' (l'' s'')j''m_{j''}} \\
        \cross \frac{1}{E + i\epsilon - p^{\prime \prime 2}/m}
        \bra{p'' (l'' s'')j''m_{j''}}\matrixel{t'' m_{t''}}{t(E)}{t m_t}\ket{p (l s)jm_{j}},
        \label{lse_pwd}
    \end{multline}
    which after using symmetries of potential matrix elements (\ref{conservation}) reduces to
    
    \begin{multline}
        \matrixel{p^\prime (l^\prime s^\prime)jt}{t(E)}{p(ls)jt} = 
        \matrixel{p^\prime (l^\prime s)jt}{V}{p(ls)jt} + \\
        +\sum_{l^{\prime\prime}} \int_0^\infty dp^{\prime \prime} p^{\prime \prime 2}
        \matrixel{p^\prime (l^\prime s)jt}{V}{p^{\prime\prime}(l^{\prime\prime} s)jt} \\
        \cross \frac{1}{E + i\epsilon - p^{\prime \prime 2}/m}
        \matrixel{p^{\prime \prime} (l^{\prime \prime} s)jt}{t(E)}{p(l s)jt}.      
        \label{lse_pwd_reduced}
    \end{multline}

    I solve \eq{lse_pwd_reduced} numerically, which again requires discretisation
    and therefore leads to a set of linear equations.
    Finally, using \eq{psi_toper} and denoting the momentum state of two nucleons
    with spin projections $m_d$ and $m_n$ as $\bra{\phi m_d m_n}$, we can write \eq{main} as
    
    \begin{equation}
        N^\mu = \bra{\phi m_d m_n} (1 + G_0(E + i\epsilon) t) J^\mu \mid \Psi_{deuteron} \rangle
        \label{main_top}
    \end{equation}

    %TODO: rewrite everything in \phi m_d m_n
    
\section{3N bound state}

    The 3N bound state is described by the Schr\"{o}dinger equation for 3N system,
    with Hamiltonian comprising two- and three- nucleon interaction.
    The bound state total wave function $\ket{\Psi}$ obeys the following equation:

    \begin{equation}
        \ket{\Psi} = G_0(E+i\epsilon)\sum_{j=1}^3 (V_j + V_4^j) \ket{\Psi},
        \label{psi3_total}
    \end{equation}
    where $G_0$ is a 3N free propagator as in \eq{g0}, $V_j$ - is a two-body potential
    acting between nucleons $k$ and $l$ (j, k and l - numerate nucleons, $j,k,l \in {1,2,3}$ and $j \neq k \neq l$),
    $V_4^j$ is a component of three-body potential $V_4 = \sum_{j=1}^3 V_4^j$
    symmetrical under exchange of nucleons $k$ and $l$,
    and $E$ - is a 3N binding energy.

    \eq{psi3_total} can be split into three independent equations for
    so-called Faddeev components $\ket{\psi_j}$

    \begin{equation}
        \ket{\Psi} = \sum_{j=1}^3 \ket{\psi_j},
        \label{faddeev_comps}
    \end{equation}
    which fulfills separately

    \begin{equation}
        \ket{\psi_j} = G_0(E+i\epsilon)(V_j + V_4^j) \ket{\Psi}.
        \label{psi3_component}
    \end{equation}

    Next, I introduce a permutation operator P, which is a combination
    of operators $P_{jk}$:
    
    \begin{equation}
        P = P_{12}P_{23} + P_{13}P_{32}.
        \label{permutation}
    \end{equation}

    The operator $P_{jk}$ acting on the state interchange the momenta and  
    quantum numbers of the nucleons $j$ and $k$.

    Using definitions (\ref{permutation}) and (\ref{faddeev_comps}),
    one can rewrite \eq{psi3_component} as:

    \begin{equation}
        \ket{\psi_j} = G_0(E+i\epsilon)t_j P \ket{\psi_j} + 
        (1 + G_0(E+i\epsilon)t_j)G_0(E+i\epsilon)V_4^j(1+P)\ket{\psi_j},
        \label{fadeed_lse}
    \end{equation}
    where $t_j$ is a two-body t-operator which obeys \eq{top2} for corresponding 
    two-body potential $V_j$. I solve \eq{fadeed_lse} numerically to find $\ket{\psi_j}$
    and the binding energy $E$. To do that, I perform PWD again.

The partial wave representation of the \eq{fadeed_lse} is obtained using
following 3N states:

\begin{equation}
    \ket{p,q,\alpha_{J,M_J}} = \ket{p,q,(ls)j, (\lambda, \frac{1}{2})I(jI)JM;(t\frac{1}{2})TM_T}_1,
    \label{3N_PW}
\end{equation}
where index 1 states the choice of the Jacobi momenta, such that $p$ is a relative momentum of the nucleons 2 and 3.
Values l, s, and j are quantum numbers in the two-body subsystem consisting of nucleons 2 and 3. $\lambda$ is the 
orbital angular momentum with respect to the c.m. of the particles 2 and 3,
of the first particle which spin is $\frac{1}{2}$
and $I$ is its total angular momentum.
$J$ and $M_J$ are the total angular momentum of the 3N system and its projection on the z-axis respectively.
$t$ is a total isospin of the 2-3 subsystem whereas $T$ and $M_T$ are the total isospin of the 3N system and its projection on the z-axis, respectively.  
The Jacobi momenta $\vec{p}$ and $\vec{q}$ for three particles with individual momenta $\vec{k_i}, i={1,2,3}$ are defined as [\cite{Glockle1983}]:
    \begin{eqnarray}
        \vec{p_i} &=& \frac{1}{2}(\vec{k_j} - \vec{k_k}),\nonumber\\
        \vec{q_i} &=& \frac{1}{3}(2\vec{k_i} - \vec{k_j} - \vec{k_k}), \quad \{i,j,k\}=\{1,2,3\}, \{2,3,1\}, \{3,1,2\}.
    \end{eqnarray}

% Using states defined in \eq{3N_PW}, we can write a partial-wave representation of the \eq{fadeed_lse}:
% \begin{equation}
%     \begin{aligned}
%         \left\langle p, q, \alpha_{J, M_J}\right. & \left|\psi_i\right\rangle=\frac{1}{E-\frac{p^2}{m}-\frac{3 q^2}{4 m}+i \epsilon}[ \\
%         & \sum_{\alpha^{\prime}, \alpha^{\prime \prime}} \int_0^{\infty} d q^{\prime \prime} q^{\prime \prime 2} \int_{-1}^1 d x t_{\alpha, \alpha^{\prime}}^{+}\left(p, \pi_1\left(q, q^{\prime \prime}, x\right), E-\frac{3 q}{4 m}\right) \\
%         * & \frac{\tilde{G}_{\alpha^{\prime}, \alpha^{\prime \prime}}\left(q^{\prime}, q^{\prime \prime}, x\right)}{\pi_1^{l^{\prime}} \pi_2^{l^{\prime \prime}}}\left\langle\pi_2, q^{\prime \prime}, \alpha_{J^{\prime \prime}, M_{J^{\prime \prime}}^{\prime \prime}} \mid \psi_i\right\rangle \\
%         +\quad & \sum_{\alpha^{\prime \prime}} \int_0^{\infty} d p^{\prime \prime} p^{\prime \prime 2} \int_0^{\infty} d q^{\prime \prime} q^{\prime \prime 2} V_{\alpha, \alpha^{\prime \prime}}^4\left(p, q, p^{\prime \prime}, q^{\prime \prime}\right)\left\langle p^{\prime \prime}, q^{\prime \prime}, \alpha_{J^{\prime \prime}, M_{J^{\prime \prime}}^{\prime \prime}} \mid \psi_i\right\rangle \\
%         +\quad & \sum_{\alpha^{\prime}, \alpha^{\prime \prime}} \int_0^{\infty} d q^{\prime} q^{\prime 2} \int_0^{\infty} d q^{\prime \prime} q^{\prime \prime \prime} \int_{-1}^1 d x V_{\alpha, \alpha^{\prime}}^4\left(p, q, \pi_1\left(q^{\prime}, q^{\prime \prime}, x\right), q^{\prime}\right) \\
%         +\quad & \frac{\tilde{G}_{\alpha^{\prime}, \alpha^{\prime \prime}}\left(q^{\prime}, q^{\prime \prime}, x\right)}{\pi_1^{l^{\prime}} \pi_2^{l^{\prime \prime}}}\left\langle\pi_2, q^{\prime \prime}, \alpha_{J^{\prime \prime}, M_{J^{\prime \prime}}^{\prime \prime}} \mid \psi_i\right\rangle\\
%         & +\sum_{\alpha^{\prime}, \alpha^{\prime \prime}} \int_0^{\infty} d p^{\prime} p^{\prime 2} \int_0^{\infty} d p^{\prime \prime} p^{\prime \prime 2} \int_0^{\infty} d q^{\prime \prime} q^{\prime \prime 2} t_{\alpha, \alpha^{\prime}}^{+}\left(p, p^{\prime}, E-\frac{3 q^2}{4 m}\right) \frac{1}{E-\frac{p^{\prime 2}}{m}-\frac{3 q^2}{4 m}+i \epsilon} \\ & \text { * } \quad V_{\alpha^{\prime}, \alpha^{\prime \prime}}^4\left(p^{\prime}, q, p^{\prime \prime}, q^{\prime \prime}\right)\left\langle p^{\prime \prime}, q^{\prime \prime}, \alpha_{J^{\prime \prime}, M_{J^{\prime \prime}}}^{\prime \prime} \mid \psi_i\right\rangle \\ & +\sum_{\alpha^{\prime}, \alpha^{\prime \prime}, \alpha^{\prime \prime \prime}} \int_0^{\infty} d p^{\prime} p^{\prime 2} \int_0^{\infty} d q^{\prime \prime} q^{\prime \prime 2} \int_0^{\infty} d q^{\prime} q^{\prime 2} \int_{-1}^1 d x t_{\alpha, \alpha^{\prime}}^{+}\left(p, p^{\prime}, E-\frac{3 q^2}{4 m}\right) \\ & \text { * } \frac{1}{E-\frac{p^{\prime 2}}{m}-\frac{3 q^2}{4 m}+i \epsilon} V_{\alpha^{\prime}, \alpha^{\prime \prime \prime}}^4\left(p^{\prime}, q, \pi_1\left(q^{\prime}, q^{\prime \prime}, x\right), q^{\prime}\right) \\ & \left.* \quad \frac{\tilde{G}_{\alpha^{\prime \prime \prime}, \alpha^{\prime \prime}}\left(q^{\prime}, q^{\prime \prime}, x\right)}{\pi_1^{l^{\prime \prime \prime}} \pi_2^{l^{\prime \prime}}}\left\langle\pi_2, q^{\prime \prime}, \alpha_{J^{\prime \prime}, M_{J^{\prime \prime}}}^{\prime \prime} \mid \psi_i\right\rangle\right] \text {, } \\ & 
%     \end{aligned}
% \end{equation}

% where

% \begin{equation}
%     V_{\alpha, \alpha^{\prime}}^4\left(p, q, p^{\prime}, q^{\prime}\right) \equiv\left\langle p, q, \alpha_{J, M_J}\left|V_4^{(1)}\right| p^{\prime}, q^{\prime}, \alpha_{J^{\prime}, M_{J^{\prime}}}^{\prime}\right\rangle
% \end{equation}
    

\section{3N scattering state}

    Let us
    introduce an asymptotic state $\ket{\Phi^{3N}_j}$ describing a free motion
    of three nucleons ${j,k,l}$ with particles $k$ and $l$ 
    having relative momentum $\vec{p}$
    and particle $j$ moving with momentum $\vec{q}$ with respect to the 
    center of mass $(k-l)$ subsystem:
    
    \begin{equation}
        \ket{\Phi^{3N}_j} \equiv \frac{1}{\sqrt{2}}(1 - P_{kl})
        \ket{\vec{p}(kl)\vec{q}(j)}.\\
    \end{equation}

    The Jacobi momenta $\vec{p}$ and $\vec{q}$ build the total energy of three
    nucleons in the 3N-c.m. system:
    
    \begin{equation}
        E_{3N} = \frac{|\vec{p}|^2}{m} + \frac{3 |\vec{q}|^2}{4m}.
    \end{equation}
    
    Using a free propagator $G(E_{3N} - i\epsilon)$ we can come up with
    the total 3N scattering state

    \begin{equation}
        \ket{\Psi^{(-)}}^{3N} = \frac{1}{\sqrt{3}} \sum_j\ket{\Psi^{(-)}_j}^{3N},  
    \end{equation}
    where auxiliary scattering states $\ket{\Psi^{(-)}_j}^{3N}$ are defined as \cite{Glockle1983} 

    \begin{eqnarray}
        \ket{\Psi^{(-)}_j}^{3N}  &\equiv& \lim_{\epsilon \rightarrow 0}
        i \epsilon G(E_{3N} - i\epsilon)\ket{\Phi^{3N}_j}.
    \end{eqnarray}

    The state $\ket{\Psi^{(-)}_j}^{3N}$ together with the state $\ket{\Phi^{3N}_j}$ fulfills an equation \cite{Glockle1983}

    \begin{equation}
        \left|\Psi_j^{(-)}\right\rangle^{3 N}=\left|\Phi_j^{3 N}\right\rangle+
        G_0\left(V_1+V_2+V_3+V_4\right)\left|\Psi_j^{(-)}\right\rangle^{3 N}.
    \end{equation}

    Defining the antysymmetrised Faddeev components, and using index "$i$" instead of "$j$"

    \begin{equation}
        \left|F_i^0\right\rangle \equiv G_0\left(V_i+V_4^{i}\right)(1+P)\left|\Psi_i^{(-)}\right\rangle^{3 N}
        \label{faddeev_3n}
    \end{equation}
    we obtain a 3N scattering wave function as:

    \begin{equation}
        \begin{aligned}
            \left|\Psi^{(-)}\right\rangle^{3 N} & =\frac{1}{\sqrt{3}}\left(\sum_{i=1}^3\left|\Phi_i^{3 N}\right\rangle+G_0\left(V_1+V_2+V_3+V_4\right) \sum_{i=1}^3\left|\Psi_i^{(-)}\right\rangle^{3 N}\right) \\
            & =\frac{1}{\sqrt{3}}\left(\sum_{i=1}^3\left|\Phi_i^{3 N}\right\rangle+\sum_{i=1}^3\left|F_i^0\right\rangle\right)
            =\frac{1}{\sqrt{3}}(1+P)\left(\left|\Phi_1^{3 N}\right\rangle+\left|F_1^0\right\rangle\right) .
        \end{aligned}
        \label{3n_scat_psi}
    \end{equation}

    In this case the nuclear matrix element 
    $N_\mu^{3 N} = \matrixel{\Psi^-}{j^\mu}{\psi_i}$ 
    (with $\ket{\psi_i}$ being the Faddeev component of the 3N bound state (\ref{faddeev_comps})) is
    \cite{GLOCKLE_report_1996, skibinski_prc_2003, Skibiski2005}:

    \begin{equation}
        \begin{aligned}
            N_\mu^{3 N} & =\frac{1}{\sqrt{3}}\left\langle\Phi_1^{3 N}\left|(1+P) j_\mu\right| \Psi_i\right\rangle+\frac{1}{\sqrt{3}}\left\langle\Phi_1^{3 N}\left|t_1^{+} G_0^{+}(1+P) j_\mu\right| \Psi_i\right\rangle \\
            & +\frac{1}{\sqrt{3}}\left\langle\Phi_1^{3 N}\left|t_1^{+} G_0^{+} P\right| U_\mu\right\rangle+\frac{1}{\sqrt{3}}\left\langle\Phi_1^{3 N}|P| U_\mu\right\rangle,
            \end{aligned}
        \label{3n_matrix}
    \end{equation}
    where $ U_\mu$ fulfills the following equation:

    \begin{equation}
        \begin{aligned}
            \left|U_\mu\right\rangle & =\left[t_1^{+} G_0^{+}+\frac{1}{2}(1+P) V_4^{(1)} G_0^{+}\left(1+t_1^{+} G_0^{+}\right)\right](1+P) j_\mu\left|\Psi_i\right\rangle \\
            & +\left[t_1^{+} G_0^{+} P+\frac{1}{2}(1+P) V_4^{(1)} G_0^{+}\left(1+t_1^{+} G_0^{+}\right) P\right]\left|U_\mu\right\rangle.
            \end{aligned}
        \label{u_mu}
    \end{equation}

    \eq{u_mu} is being solved numerically in PWD scheme. The numerical techniques applied are the same as
    those presented in \cite{GLOCKLE_report_1996} for N-d elastic scattering.



\section{Nd scattering state}
\label{nd_state}

    Analogously to the bound and 3N scattering states, one can express a nucleon-deuteron
    scattering state
    using a permutation operator \eq{permutation}.

    \begin{equation}
        \ket{\Psi^{(-)}}^{Nd} = \frac{1}{\sqrt{3}}\sum_j\ket{\Psi^{(-)}_j}^{Nd}    
    \end{equation}

    Further a scattering state $\ket{\Psi^{(-)}_j}^{Nd}$ can be expressed
    in terms of asymptotic state $\ket{\Phi^{Nd}_j}$, in which particles
    k and l form a deuteron and the third particle (nucleon j)
     propagates freely with a relative momentum $\vec{q}_0$ with 
    respect to the deuteron:

    \begin{eqnarray}
        \ket{\Psi^{(-)}_j}^{Nd} &\equiv& \lim_{\epsilon \rightarrow 0} 
        i \epsilon G(E_{Nd} - i\epsilon)\ket{\Phi^{Nd}_j}\\
        \ket{\Phi^{Nd}_j} &\equiv& \ket{\Phi_{d(k,l)}}\ket{\vec{q}_0},
    \end{eqnarray}
    where $\ket{\Phi_{d(k,l)}}$ is a deuteron wave function and 
    $\ket{\vec{q}_0}$ - a free particle state and $G(t)$ is 
    a free propagator of N-d pair. 
    $E_{Nd}$ is the total energy of the N-d system:

    \begin{equation}
        E_{Nd} = E_d + \frac{3 |\vec{q}_0|^2}{4m},
    \end{equation}
    where $E_d$ is the deuteron binding energy and $m$ denotes the nucleon mass. 

    $\ket{\Psi^{(-)}}^{Nd}$ can be expressed by the Faddeev components

    \begin{equation}
        \ket{\Psi^{(-)}}^{Nd} = \frac{1}{\sqrt{3}} (1+P) \ket{F_1},
    \end{equation}
    where 
    
    \begin{equation}
        \ket{F_1} = \ket{\phi_1^{Nd}} G_0 (V_1 + V_2 + V_3 + V_4) 
        \sum_{j=1}^3 \ket{\Psi_j^{(-)}}^{Nd}.
    \end{equation}

    The nuclear matrix element $N_\mu^{Nd} = ^{Nd} \matrixel{\Psi^{-}}{j_\mu}{\Psi_i}$ 
    (with $\Psi_i$ - bound state) is now \cite{GLOCKLE_report_1996, skibinski_prc_2003}:

    \begin{equation}
        N_\mu^{N d}=\frac{1}{\sqrt{3}} \matrixel{\Phi_1^{N d}}{(1+P) j_\mu}{\Psi_i}
        +\frac{1}{\sqrt{3}} \matrixel{\Phi_1^{N d}}{P}{U_\mu}
    \end{equation}
    with $U_\mu$ being a solution of \eq{u_mu}.
    Fact that solving once \eq{u_mu} opens opportunity to compute both  $N_{\mu}^{Nd}$
    and $N_{\mu}^{3N}$ is a great advantage of the presented formulas. 

    % \begin{eqnarray}
    %    \ket{U_\mu} &=& \left[t_1^{+} G_0^{+}+\frac{1}{2}(1+P) V_4^{(1)} G_0^{+}\left(1+t_1^{+} G_0^{+}\right)\right](1+P) j_\mu \ket{\Psi_i} + \nonumber\\
    %    &+& \left[t_1^{+} G_0^{+} P+\frac{1}{2}(1+P) V_4^{(1)} G_0^{+}\left(1+t_1^{+} G_0^{+}\right) P\right]\ket{U_\mu}
    %    \label{u_mu_nd}
    % \end{eqnarray}

    % \eq{u_mu_nd} is being solved numerically in a same way as \eq{u_mu}, namely applying
    % PWD, as it is described in \cite{GLOCKLE_report_1996}.


    \section{Nuclear electromagnetic current}
    \label{sec_current}
    
    % The use of nuclear currents in scattering experiments is essential for understanding the structure of the nucleus and
    % the interactions between its constituent nucleons. Advances in theoretical and experimental techniques have allowed
    % for more precise measurements of these currents and have provided insights into
    % the fundamental properties of the nucleus.

    % The nuclear one-body current is constructed based on the electromagnetic current for a Dirac particle. To account
    % for the extended structure of nucleons, form factors are introduced. Since we use nonrelativistic wave functions, we need to use nonrelativistic currents.    
    % \tmp{????????????????????????????????????????????????????}
    The electromagnetic current operator for 2N (3N) system is constructed 
    from the one- and many-body currents:
    
    \begin{equation}
        j_\mu = j_\mu^1 + j_\mu^2 (+ j_\mu^3).
        \label{j_mu_gen}
    \end{equation}

    Here $j_\mu^n$ is a contribution from the interaction between a photon and 
    $n$ nucleons (obviously $j_\mu^3$ is present in 3N system only).
    Each $j_\mu^n$ describes the interaction of photon with all relevant permutations of $n$ nucleons,
    i.e. $j_\mu^1$ is a sum of photon interactions with each of nucleons in the system separately. 
    % For 2N:

    % \begin{equation}
    %     j_\mu^1 = j_\mu^1(1) + j_\mu^1(2),
    % \end{equation}
    % where $j_\mu^1(i)$ is an interaction of photon with $i$-th nucleon.


    In the case of the photodisintegration process, I will use only a \gls{snc},
    so I stick to its definition here.
    In the Hamiltonian framework, the nucleons are constrained to lie on the mass shell.
    The general current expression for a
    single nucleon at the spacetime point zero, denoted as $j^1_\mu(0)$, is computed between the initial nucleon momentum
    $p \equiv\left(p_0=\sqrt{M_N^2+\vec{p}\,^2}, \vec{p}\right)$
    and the final momentum
    $p^{\prime} \equiv\left(p_0^{\prime}=\sqrt{M_N^2+\pvec{p}^{2}}, \pvec{p}\right)$
    with the latter one constrained by the four-momentum transfer $Q$ from photon to nucleon.
    This computation yields the following matrix elements:

    \begin{equation}
        \begin{aligned}
        \matrixel{\pvec{p}} {j^1_\mu(0)} {\vec{p}} & =& \bar{u}\left(\pvec{p} s^{\prime}\right)\left(\gamma^\mu F_1+i \sigma^{\mu \nu}\left(p^{\prime}-p\right)_\nu F_2\right) u(\vec{p} s) \\
        & =&\bar{u}\left(\pvec{p} s^{\prime}\right)\left(G_M \gamma^\mu-F_2\left(p^{\prime}+p\right)^\mu\right) u(\vec{p} s) .
        \end{aligned}
        \label{current_general}       
    \end{equation}
    
    In the above equation,
    the symbols $u(\vec{p} s)$ represent Dirac spinors
    for particle having momentum $\vec{p}$ and spin $s$, 
    $F_1$ and $F_2$ are the Dirac and Pauli form factors of the nucleon, respectively, and $G_M \equiv F_1+2 M_N F_2$ denotes the magnetic form factor of the nucleon. 
    The proton charge $e$ is extracted from the matrix element. Further $\gamma^\mu$ are Dirac matrices 
    and $\sigma^{\mu\nu} = \frac{i}{2}\left[\gamma^\mu, \gamma^\nu\right]$.
    In this thesis, only the nonrelativistic limit of \eq{current_general} is considered, which leads to well-known expressions for the operators for the single nucleon current \cite{Golak2005}:


    \begin{align}
        % \begin{array}{r}
            \matrixel*{\pvec{p}} {j^1_0} {\vec{p}} &=\left(G_E^p \Pi^p+G_E^n \Pi^n\right), \\
            \matrixel{\vec{p}} {\vec{j^1}} {\vec{p}} &=\frac{\vec{p}+\pvec{p}}{2 M_N}\left(G_E^p \Pi^p+G_E^n \Pi^n\right)+\frac{i}{2 M_N}\left(G_M^p \Pi^p+G_M^n \Pi^n\right) \vec{\sigma} \times\left(\pvec{p}-\vec{p}\right) .
        % \end{array}
        \label{snc_formfact}
    \end{align}

    The electric form factor denoted as $G_E$ is defined as
    $G_E \equiv F_1+\frac{\left(p^{\prime}-p\right)^2}{2 M_N} F_2$ and 
    represents the neutron (n) and proton (p) form factors.
    Both electric and magnetic form factors, $G_E(Q^2)$ and $G_M(Q^2)$, are normalized as:
    \begin{equation}
        \begin{aligned}
        G_E^n(0) & =0 \\
        G_E^p(0) & =1 \\
        G_M^n(0) & =-1.913 \\
        G_M^p(0) & =2.793
        \end{aligned}
    \end{equation}

    The above values correspond to nucleons with point-like characteristics. While all the form factors depend on the squared
    four-momentum transfer $\left(p^{\prime}-p\right)^2$, in the nonrelativistic regime, it is common to use
    as arguments of $G_E$ and $G_M$ the squared
    three-momentum transfer $-\left(\pvec{p}-\vec{p}\right)^2$ or even set it to zero for interactions involving
    real photons. Numerous authors have investigated the properties of electromagnetic nucleon form factors through
    theoretical and experimental approaches, as discussed in \cite{Arrington_2007, JOURDAN1999513c}
    but at low energies discussed here, all models lead to practically the same values of the form factors.
    $\Pi^p$ ($\Pi^n$) in \eq{snc_formfact} is a proton (neutron) isospin projection operator.

    The two-nucleon current contribution is in that work taken into account via Siegert approach \cite{Siegert, GolakKamad2000_ExplDescr, Golak2005}.
    To do that we break down the single nucleon current matrix elements into multipole components and 
    represent some of the electric multipoles using the Coulomb multipoles,
    which arise from the single nucleon charge density operator \cite{Golak2005}.
    This is acceptable because, in low-energy regime, contributions from many nucleons to the nuclear charge density are typically small.
    We then obtain the rest of the electric multipoles and all of the magnetic multipoles exclusively from the single nucleon current operators.

    In 3N system we have following expressions for the \gls{snc}:

    \begin{align}
        % \begin{aligned}
            \matrixel**{\pvec{p}, \pvec{q}}{j_0^1}{\vec{p}, \vec{q}} 
            & =\int d \ppvec{p} d \ppvec{q}
            \matrixel**{\pvec{p}, \pvec{q}}{\frac{1}{2}\left(1+\tau(1)_z\right) F_1^p+\frac{1}{2}\left(1-\tau(1)_z\right) F_1^n}{\ppvec{p}, \ppvec{q}}\nonumber\\
            &\braket{\ppvec{p}, \ppvec{q}-\frac{2}{3} \vec{Q}}{\vec{p}, \vec{q}} \\
            \matrixel**{\pvec{p}, \pvec{q}}{j_{ \pm}^{1, \text { conv }}} {\vec{p}, \vec{q}} 
            & =\frac{1}{m} \int d \ppvec{p} d \ppvec{q}
            \bra{\pvec{p}, \pvec{q}}\left[\frac{1}{2}\left(1+\tau(1)_z\right) F_1^p+\frac{1}{2}\left(1-\tau(1)_z\right) F_1^n \right] q_{ \pm}\nonumber\\
            &\ket{\ppvec{p}, \ppvec{q}}
            \braket{\ppvec{p}, \ppvec{q}-\frac{2}{3} \vec{Q}}{\vec{p}, \vec{q}} \label{3n_conv}\\
            \matrixel**{\pvec{p}, \pvec{q}}{j_{ \pm}^{1, spin}}{\vec{p}, \vec{q}} 
            & =\frac{i}{2 m} \int d \ppvec{p} d \ppvec{q}
            \bra{\pvec{p}, \pvec{q}}\nonumber\\
            & *\left[\frac{1}{2}\left(1+\tau(1)_z\right)\left(F_1^p+2 m F_2^p\right)+\frac{1}{2}\left(1-\tau(1)_z\right)\left(F_1^n+2 m F_2^n\right)\right] \nonumber \\
            & *(\vec{\sigma}(1) \times \vec{Q})_{ \pm}\ket{\ppvec{p}, \ppvec{q}}\braket{\ppvec{p}, \ppvec{q}-\frac{2}{3} \vec{Q}}{\vec{p}, \vec{q}} .
            \label{3n_spin}
        % \end{aligned}
    \end{align}

    In \eq{3n_conv} and \eq{3n_spin} we present convection and spin currents which combined form a spatial component of the 
    single nucleon current $\vec{j^1}(\mu) \equiv \vec{j^1}(\mu, conv) + \vec{j^1}(\mu, spin)$. 
    In my thesis I use a model of M.~Garu and W.~Kr\"umpelmann  \cite{GARI198610} for which:
    \begin{align}
        % \begin{ar
            F_1^p(0)&=1 & 2 m F_2^p(0)&=1.793 \\
            F_1^n(0)&=0 & 2 m F_2^n(0)&=-1.913
        % \end{array}
    \end{align}


    \section{Pion absorption}

    For the pion absorption, I include explicitly \gls{2nc} as well as \gls{snc}.
    Thus the absorption operator is $\rho = \rho(1) + \rho(1, 2)$,
    where absorption on  a single nucleon is included in $\rho(1)$ and
    $\rho(1, 2)$ plays a role of two-body charge current.
    The matrix element of the single nucleon pion absorption operator $\rho(1)$ in momentum-space for nucleon 1 relies on the
    nucleon's incoming momentum ($\vec{p}$) and outgoing momentum ($\pvec{p}$) \cite{BERNARD_1995}:

    \begin{eqnarray}
        \matrixel{\pvec{p}} 
        {\rho(1)} {\vec{p}} = 
        - \frac{g_A M_\pi}{\sqrt{2} F_\pi } \,
            \frac{ \left( \pvec{p} +  \vec{p} \right) \cdot  \vec{\sigma}_1 } { 2 M } \, 
            (\vec{\tau}_1)_- \, ,
    \label{rho1}
    \end{eqnarray}
    where the values of the nucleon axial vector coupling, pion decay constant, and negative pion mass are $g_A =$ 1.29, 
    $F_\pi = \SI{92.4}{\mev}$, and $M_\pi = \SI{139.57}{\mev \per \clight\squared}$, respectively. $\rho (1)$ operates in the spin
    and isospin spaces and involves the Pauli spin (isospin) operator $\vec{\sigma}_1$ ($\tau_1$) for nucleon 1 and the isospin lowering operator
    $(\vec{\tau}_1)_- \equiv((\vec{\tau}_1)_x -{\rm i} (\vec{\tau}_1)_y)/2$.
    As before I use the average 
    "nucleon mass" $M \equiv \frac{1}{2} \left( M_p + M_n , \right)$ where the proton mass is $M_p$ and neutron mass is $M_n$.



    The 2N part of $\rho$ at \gls{lo} has the form \cite{Lensky2006}
    \begin{eqnarray}
    \matrixel{
        \pvec{p}_1, \pvec{p}_2
        } 
    {\rho(1,2)}
    {
        \vec{p}_1 \, 
        \vec{p}_2 \, 
        } = 
        \left(
        v( k_2 )  \vec{k}_2 \cdot \vec{\sigma}_2 \, - \, 
        v( k_1 )  \vec{k}_1 \cdot \vec{\sigma}_1 \,
        \right) \, \nonumber \\ \times \,
        \frac{i}{\sqrt{2}} \, 
        \left[ 
            \left( \vec{\tau}_1 \times \vec{\tau}_2 \, \right)_x 
            - i \left( \vec{\tau}_1 \times \vec{\tau}_2 \, \right)_y \,
        \right] \,,
    \label{rho12}
    \end{eqnarray}
    where 
    $ \vec{k}_1 = \vec{p}_1^{\, \prime} - \vec{p}_1 $,
    $ \vec{k}_2 = \vec{p}_2^{\, \prime} - \vec{p}_2 $
    and the formfactor $v(k)$ reads 
    \begin{eqnarray}
    v (k) = \frac 1{ \left( 2 \pi \, \right)^3 } \,
            \frac{g_A M_\pi}{4 F_\pi^3 } \,
        \frac1{M_\pi^2 + k^2 } \, .
    \label{vk}
    \end{eqnarray}

    In the case of the pion absorption process, we follow a standard procedure
    including partial wave states for both 2N and 3N induced nuclei.
    It occurs that in such a case one needs current matrix elements. 
    % An important difference between these two cases is that
    % after introducing partial wave states, the transition operator is acting between
    % different states. 
    For 2N this is 
    $ \matrixel{\, \vec{p} + \frac12\vec{P}_f } 
    {\rho (1) }{ \vec{p} - \frac12\vec{P}_f + \vec{P}_i \, }$
    and for 3N case
    $\matrixel {\, \vec{q} + \frac13\vec{P}_f}  
    {\rho(1)} {\vec{q} - \frac23\vec{P}_f + \vec{P}_i \,}$.

    For the $\pi^- + {^2{\rm H}} \rightarrow n + n $ reaction,
    the nuclear matrix element for the transition operator is given by:
    \begin{eqnarray}
        N_{nn} (m_1, m_2, m_d \, ) \ = \
   ^{(-)}\matrixel{ \vec{p}_0 \, m_{1} \, m_2 \ \vec{P}_f=0 } 
   {\rho } {\phi_d \, m_d \ \vec{P}_i=0 \, },
   \label{nnn1}
   \end{eqnarray}
   where $ \ket{  \vec{p}_0 \, m_{1} \, m_2 \ \vec{P}_f=0  }^{(-)}  $ denotes the 2N scattering state \cite{Golak2018}.
   The total absorption rate for this reaction expresses as:

   \begin{eqnarray}
        &&\Gamma_{nn} = 
    \frac{ \left( \alpha \, M^\prime_d \, \right)^3 \, c \, M_n \, p_0 }{ 2 M_{\pi^-} }
        \int d \hat{p}_0 \,
        \frac13 \, 
        \sum\limits_{m_1, m_2, m_d} 
        \left| 
        N_{nn} (m_1, m_2, m_d \, ) \, 
        \right|^2  \, .
    \label{gnn1}
    \end{eqnarray} 

    Turning to 3N system I investigate pion absorption in $^3$He or $^3$H
    with various final states.
    For $\pi^- + {^3{\rm He}} \rightarrow n + d $ reaction the most important step in 
    obtaining predictions is calculating
    the matrix element
    of the 3N transition operator $\rho_{3N}$,
     which is the $\rho$ acting between the initial $^3$He and the final Nd scattering state immersed in 3N space:

    \begin{eqnarray}  
        N_{nd} (m_n, m_d , m_{^3{\rm He}}  \, )  \, \equiv \, 
        {}^{(-)}\matrixel{\Psi_{nd}  \, 
            m_n \, m_d \,
            \vec{P}_f=0 
            } {\rho_{3N} }
        {\Psi_{^3{\rm He}} \, m_{^3{\rm He}} \, \vec{P}_i=0 \, }.
        \label{nnd}
    \end{eqnarray}

    Given \eq{nnd} the total absorption 
    rate may be obtained from the following:

    \begin{equation}
        \Gamma_{nd} = 
     {\cal R} \, \frac{ 16 \, \left( \alpha^3 \, M^\prime_{^3{\rm He}} \, \right)^3  \, c \, M q_0 }{ 9 M_{\pi^-}  }
          \int d \hat{q}_0 \,
          \frac12 \, 
         \sum\limits_{m_n, m_d, m_{^3{\rm He}}} 
         \left| 
         N_{nd} (m_n, m_d, m_{^3{\rm He}} \, ) \, 
         \right|^2  \, ,  % with \rho(1) + \rho(2) + \rho(3) + ...
    \label{gnd}
    \end{equation}  
    where
    $ M^\prime_{^3{\rm He}}  = \frac { M_{^3{\rm He}} M_{\pi^-} } { M_{^3{\rm He}} + M_{\pi^-} }$
    is now the reduced mass of the $\pi^- - {^3{\rm He}}$ system.
    The factor ${\cal R} = 0.98 $ appears due to the finite
    volume of the $^3$He charge \cite{marcucci_2011}.
    The final state energy is expressed in terms of the neutron momentum $\vec{q}_0$
    \begin{eqnarray}
    M_\pi + M_{^3{\rm He}} \approx M_n + M_d + \frac{3}{4} \frac{ \vec{q}_0^{\ 2}} {M} \, ,
    \label{eq0}
    \end{eqnarray}  

    The full 3N breakup is calculated in a similar way and the total absorption rate for $\pi^- + {^3{\rm He}} \rightarrow p + n + n $ reaction is defined as follows

    \begin{eqnarray}
        \Gamma_{pnn} = 
       {\cal R} \, \frac{ 16 \, \left( \alpha\, M^\prime_{^3{\rm He}} \, \right)^3 \, c\, M}{ 9 M_{\pi^-} }
          \int d \hat{q} \,
          \int\limits_{0}^{2 \pi} d \phi_{p} \,
              \int\limits_{0}^{\pi} d \theta_{p} \sin \theta_{p} \, \nonumber \\
              \times 
              \int\limits_0^{p_{max}} \, dp p^2  \,
          \sqrt{\frac43 \left( M E_{pq} - p^2  \right)} \,
          \frac12 \, 
         \sum\limits_{m_1, m_2, m_3, m_{^3{\rm He}}} 
         \left| 
         N_{pnn} (m_1, m_2, m_3, m_{^3{\rm He}} \, ) \, 
         \right|^2  \,    % with \rho(1) + \rho(2) + \rho(3) + ...
    \label{gpnn}
    \end{eqnarray}
    with

    \begin{eqnarray}  
        N_{pnn} (m_1, m_2, m_3 , m_{^3{\rm He}}  \, )  \, \equiv \, 
        {{}^{(-)}\bra{\Psi_{pnn}  \, 
            m_1 \, m_2 \, m_3 \,
            \vec{P}_f=0 
            }}\, 
            \rho_{3N}
        \, \ket{\Psi_{^3{\rm He}} \, m_{^3{\rm He}} \, \vec{P}_i=0 \, 
            } 
        \label{npnn}
    \end{eqnarray}

    $E_{pq}$ is the internal energy of the final 3N state and can be expressed in terms of the Jacobi relative momenta $\vec{p}$ and $\vec{q}$ 

    \begin{eqnarray}
        M_\pi + M_{^3{\rm He}} 
        \approx 3 M + \frac{ \vec{p}^{\, 2}} {M} + \frac34 \frac{ \vec{q}^{\, 2}} {M} 
        \equiv 3 M + E_{pq}  \, 
        = 3 M + \frac{ p_{max}^{\, 2}} {M} \,
        = 3 M + \frac34 \frac{ q_{max}^{\, 2}} {M} \, .
        \label{epq}
    \end{eqnarray} 

    In \eq{epq} $p_{max}$ and $q_{max}$ are maximal kinematically allowed values of 
    Jacobi momenta $p$ and $q$, respectively.

    Analogously, the total absorption rate for $\pi^- + {^3{\rm H}} \rightarrow n + n + n $
    reads

    \begin{eqnarray}
        \Gamma_{nnn} = 
    \frac { 2\, \left( \alpha \, M^\prime_{^3{\rm H}} \, \right)^3 \, c \, M}
    { 27 M_{\pi^-}  }
            \int d {\bf\hat q} \,
            \int\limits_{0}^{2 \pi} d \phi_{p} \, 
            \int\limits_{0}^{\pi} d \theta_{p} \sin \theta_{p} \, 
            \nonumber \\
            \times 
            \int\limits_0^{p_{max}} \, dp p^2  \,
            \sqrt{\frac43 \left( M E_{pq} - p^2  \right)} \,
            \frac12 \, 
            \sum\limits_{m_1, m_2, m_3, m_{^3{\rm H}}} 
            \left| 
            N_{nnn} (m_1, m_2, m_3, m_{^3{\rm H}} \, ) \, 
            \right|^2  \,
    \label{gnnn}
    \end{eqnarray}
    with

    \begin{eqnarray}  
        N_{nnn} (m_1, m_2, m_3 , m_{^3{\rm H}}  \, )  \, \equiv \, 
        {{}^{(-)}\bra{\Psi_{nnn}  \, 
                m_1 \, m_2 \, m_3 \,
                \vec{P}_f=0 
                }}\, 
                \rho_{3N}
        \, \ket{\Psi_{^3{\rm H}} \, m_{^3{\rm H}} \, \vec{P}_i=0 
        } \, .
        \label{nnnn}
    \end{eqnarray}




    In the Results section, I also demonstrate (for processes
    with three free nucleons in the final state) predictions of the differential absorption rates.
    The natural domain is defined by energies of outgoing nucleons    
    $(E_1, E_2)$; in such a case the differential absorption rate
     $ {d^2\Gamma_{pnn} }/ \left( {d E_1 \, d E_2} \right) $ expresses as\cite{Golak2018}
    \begin{eqnarray}
        \frac{d^2\Gamma_{pnn} }{ {d E_1 \, d E_2}  }\, = \,
                    {\cal R} \, 
                    \frac { 64\, \pi^2 \, \left( \alpha \, M^\prime_{^3{\rm He}} \, \right)^3 \, c\, M^3 } { 3 M_{\pi^-}   } \,
            \nonumber \\
            \times \,
            \frac12 \, 
            \sum\limits_{m_1, m_2, m_3, m_{^3{\rm He}}} 
            \left| 
            N_{pnn} (m_1, m_2, m_3, m_{^3{\rm He}} \, ) \, 
            \right|^2  \, .   % with \rho(1) + \rho(2) + \rho(3) + ...
    \label{gpnn.4}
    \end{eqnarray}
    
    The kinematically allowed region is restricted to energies fulfilling 
    the condition

    \begin{equation}
         -1 \le \frac{E - 2 E_1 - 2 E_2 }{ 2 \sqrt{ E_1 \, E_2} } \le 1
    \end{equation}
    with $E=E_1 + E_2 + E_3$.

    One can also calculate the differential absorption rate with 
    respect to the dimensionless variables $x$ and $y$
    which are frequently used in the literature, specifically to build
    so-called Dalitz plots \cite{Gotta1995}

    \begin{eqnarray}
        x & = & \sqrt{3} \, ( E_1 + 2 E_2 - E ) / E \, , \nonumber \\
        y & = &  ( 3 E_1 - E ) / E \, .
    \label{xy}
    \end{eqnarray}

    Such definition leads to a simple kinematically allowed region, namely to
    the disk $ r^2 \equiv x^2 + y^2 \le 1 $.
    One can evaluate 
    $ {d^2\Gamma_{pnn} }/ \left( {d x \, d y} \right) $
    or (using polar coordinates)
    $ {d^2\Gamma_{pnn} }/ \left( {d r \, d \phi} \right)$
    and relate it with $\frac{d^2\Gamma_{pnn}}{dE_1dE_2}$.
    The same can be done for $\frac{d^2\Gamma_{nnn}}{dE_1dE_2}$,
    $\frac{d^2\Gamma_{nnn}}{dxdy}$ and $\frac{d^2\Gamma_{nnn}}{drd\phi}$.

\section{Theoretical uncertainties}

    Striving to achieve valuable theoretical results, we cannot omit the estimation of their
    uncertainty. There are various sources of uncertainty appearing in predictions.
    The three most important are: the truncation error, the cutoff dependence and
    the uncertainty related to various models of nuclear interaction.
    The latter can be easily estimated by computing predictions arising from various 
    models, see discussion below.
    The way how to estimate the first two types of uncertainties, together with a short
    discussion of the remaining sources of theoretical errors is given below. 

    \subsection*{Truncation error}
    \label{sec:trunc}

    As mentioned above, each subsequent order of the chiral
    expansion provides us with more and more sophisticated
    potential which is expected to increase the accuracy of data description.
    Starting from the leading order (LO) and coming next to
    \gls{nlo}, \gls{n2lo}, \gls{n3lo}, etc., we take into account more topologies (equivalents of Feynmann diagrams) 
    and resulting potential is expected to provide us with more precise predictions
    for the regarded process and observables. However, the chiral expansion (as any expansion) 
    in principle can be continued up to the infinity, improving the resulting series.
    In practice, we are limited to finite, rather small, orders 
    and we would like to find out
    the uncertainty appearing from cutting off the remaining part of the expansion.
    That type of theoretical uncertainty is called a truncation error. 
    Various methods to estimate its value
    have been proposed \cite{Epelbaum2014SCS, Epelbaum2015_trunc, Binder2015, Epelbaum_pos, Miller_arxiv}.
    Typically predictions at lower orders serve as input information to get truncation error at given order.
    It is worth adding that Bayesian analysis is also used for truncation error estimation.

    I use the method proposed in \cite{Binder2015}.
    Let us regard some prediction $X^i(p)$ for observable $X$ which is calculated
    at $i$-th order of the chiral expansion 
    with the expansion parameter $Q$ ($i = 0,2,3...)$\footnote{As mentioned in Sec.~\ref{sec:intro}, we do not have a first order of expansion
    because this term in the chiral expansion always vanishes
    and NLO  corresponds to the quadratic term ($\nu=2$)}.
    Here $p$ specifies a momentum
    scale of the reaction. In the case of photodisintegration, $p$ is given by a
    photon's momentum.  

    If we define the difference between observables at each subsequent order as:

    \begin{align}
        \Delta X^{(2)} &= |X^{(2)} - X^{(0)}|,& \Delta X^{(i>2)} &= |X^{(i)} - X^{(i-1)}|,
    \end{align}
    than chiral expansion for $X$ can be written as:

    \begin{equation}
        X = X^{(0)} + \Delta X^{(2)} + \Delta X^{(3)} + ... + \Delta X^{(i)}.
        \label{trunc1}
    \end{equation}
        
    The truncation error at $i$-th order, $\delta X^{(i)}$, is estimated using
     values of the observable obtained at lower 
    orders as follows:

    \begin{eqnarray}
        \delta X^{(0)} &=& Q^2 \left| X^{(0)} \right| \label{trunc2},\\ 
        \delta X^{(i)} &=& \max_{2 \leq j \leq i} \left( Q^{i+1} \left| X^{(0)} \right|,
        Q^{i+1-j} \left| \Delta X^{(j)} \right| \right). \label{trunc3} 
    \end{eqnarray}

    Additionally, following \cite{Binder2015} I use the actual high-order predictions 
    (if known) to specify uncertainties at lower orders, so that:

    \begin{equation}
        \delta X^{(i)} \geq \max_{j,k} (|X^{j \geq i} - X^{k \geq i}|)
        \label{trunc4}
    \end{equation}
    and to be conservative I use additional restriction:

    \begin{equation}
        \delta X^{(i)} \geq Q \delta X^{(i-1)}.
        \label{trunc5}
    \end{equation}

    All the conditions above assume that we use the whole available information at hand.
    In \cite{Melendez_BayesTrunc} it was shown that such a method is equivalent
    to the Bayesian approach proposed there.


    \subsection*{Cut-off dependence}



    \begin{figure}[htb]
        \begin{center}
            \includegraphics[width=0.95\textwidth]{Figures_De/TOTAL_CROSSSECTION_cutoff.pdf}
        \end{center}
        \caption{Total cross section of the deuteron photodisintegration
        process (normalized to the maximal cross section among all $\Lambda$)
        as a dependence of the cutoff parameter $\Lambda$ 
        for three photon energy E$_\gamma$ values: 30, 100, and 140 MeV.}
        \label{Cutoff_dep}
        \end{figure}

    Another theoretical uncertainty comes from the choice of the cutoff parameter's value 
    of regulator described in the Chapter \ref{sec:intro}.

    In the case of the \gls{sms} interaction, its free parameters have been obtained from data for 
    % According to \cite{reinkrebs2018}, where the \gls{sms} potential was presented,
   four values of the cutoff parameter $\Lambda$:
   \SIlist[list-units = single]{400;450;500;550}{\mev} \cite{reinkrebs2018}.
   Using each of these values, one obtains different predictions which, of course, can further differ 
   from the actual (experimental) value. Therefore the choice of $\Lambda$ value
   may affect the quality of the prediction.

    To study that I also use the same four values of the $\Lambda$ parameter,
    obtaining in that way a set of four predictions each time.
    That is exemplified in \fig{Cutoff_dep}  for the deuteron photodisintegration cross section for the
    photon's energy $E_\gamma = \SIlist{30;100;140}{\mev}$.
    The \gls{sms} model at \gls{n4lo+} with two-nucleon force is used.
    % Comparison of predictions obtained with different values of the 
    % cutoff parameter $\Lambda$ is presented on the Fig.~\ref{Cutoff_dep}.
    Each subfigure shows predictions for the total cross section as 
    a function of the cutoff parameter,
    normalized to the maximum value among all cross sections
    obtained with various $\Lambda$.
    As we can see, there is almost linear dependence for that observable with a positive linearity coefficient value:
    with higher $\Lambda$ the cross section value increases as well.
    Note that the higher the photon's energy is, the stronger becomes the cutoff dependence: for $E_\gamma=\SI{30}{\mev}$
    the maximal difference between predictions is around \SI{0.5}{\percent} while
    for \SI{140}{\mev} it increases to more than \SI{8}{\percent}.
    These results are generally within our expectations that the chiral model works better at
    smaller energies
    and is therefore less sensitive to the $\Lambda$ value. Let us remind that $\Lambda$ 
    governs the behavior of the potential at small internucleon distances and only higher energy transfer probes that distances. 



    \subsection*{Other theoretical uncertainties}

    There are obviously more sources of theoretical uncertainties. 
    Our model has several either intrinsic limitations in the precision or 
    some simplifications which may be improved with further developments of the model.
    
    \paragraph{Nuclear currents}
    At the moment, our model is limited to a single nucleon current, which may not be sufficient to accurately describe the processes under consideration.  This limitation will be further discussed and tested in
    Chapter~\ref{chap:results}. To address this issue, I utilize the Siegert theorem, which enables me to
    incorporate some contributions from the two-nucleon current, although it does not yet complete a job.
    It is worth noting
    that the incorporation of the two-nucleon current can significantly affect the predicted observables, as it
    includes additional physical effects that are not accounted for in the single-nucleon current. Therefore, the
    ongoing development of a complete chiral two-nucleon current
    is of great importance for our model to improve the accuracy of the predictions.

    \paragraph{Nonrelativistic approach}
    All the results presented here do not include relativistic corrections.
    At the lower energies, the relativistic contribution might not be crucial,
    but at the region with higher energy, we may see a lack of precision.
    This will be also confirmed and discussed on example of the total cross section
    for the deuteron photodisintegration (see \fig{TOTAL_CROSS}).
    
    \paragraph{Uncertainties in the potential free parameters}
    Since not all the chiral potential parameters are given by the theory, the 
    fitting-to-data procedure is applied
    to obtain the potential parameters \cite{reinkrebs2018}. 
    The values of the free parameter of the \gls{sms} NN potential as well as free
    parameters of the  3N  interaction have been obtained from the data by the least square fitting.
    As any fitting procedure, it introduces an uncertainty to the obtained values which depends on the algorithm's precision as
    these values are actually estimators of expectation values only.
    These errors, in principle, are being propagated to the observables as different sets of parameters
    lead to different predictions.
    Indeed, in \cite{reinkrebs2018} the whole correlation matrix for free parameters of the  \gls{sms} NN
    force is given. Using that knowledge, it is possible to study the propagation of the uncertainty of NN
    force parameters to 3N observables. It is done in \cite{skibincki_prc_2018, Volkotrub_2020} for the
    elastic and inelastic nucleon-deuteron scattering.
    Resulting uncertainties have been found to be much smaller (typically one order of magnitude)
    than uncertainties arising from truncation errors or from the cutoff dependence.
    I expect  the same uncertainty level in electromagnetic processes and thus I do not
    intend to study that theoretical error in the presented thesis.
    Moreover, the estimation of such errors is computationally expensive, since it requires 
    significant amount of computer processing power (separate calculations for each set of parameters).
    In the future, it could be interesting to check that type of uncertainty for photodisintegration
    processes but it should be done after completing all pieces of Hamiltonian,
    specifically after including many-body electromagnetic currents.

    \paragraph{Uncertainties from the numerical method}
    All my results are based on numerical calculations,
    so we can come up with a lot of places where numerical methods with limited 
    precision comes into the scene. 
    Our approach is based on partial wave decomposition 
    and in practice only a limited number of partial waves is included 
    (usually for 2N scattering we use all channels up to $j^{max}=4$ which corresponds to 18 partial waves).
    For 3N calculations we use $j^{max}=5$ and $J^{max}=15/2$ which corresponds to 142 partial waves.
    In addition, I work with a grid of points which is used for the calculation of potential, wave function, numerical integration, etc. 
    The choice of grid affects the final results' precision. 
    Usually, we use a grid of 32 values which was proven to make resulting
    uncertainty very small \cite{Glockle1983}.

    \paragraph{Model choice}
    I focus on the \gls{sms} potential, but using another model of interaction 
    in general leads to different predictions. That difference is also a theoretical uncertainty thus
    we use predictions obtained with the semiphenomenological AV18 model to compare the chiral results.
    The AV18 model is a widely used and well-established model of nuclear interaction,
    which has been extensively tested and benchmarked against experimental data. By comparing
    the predictions obtained from the SMS potential with those obtained from the AV18 model,
    we can assess the robustness (compared to the AV18 results) of our results and determine the extent to which they depend
    on the choice of the interaction model. This comparison also helps to identify the strengths
    and weaknesses of each model and provides insights into the underlying physics of nuclear interactions.

    \paragraph{Machine precision}

    Finally, it should be noted that every computer calculation includes limited numerical machine precision which
    can be noticeable for complex calculations. We perform our calculations via CPU machine, where the particular
    choice of the processor and memory card may lead to numerical uncertainty. However, we have found that this
    uncertainty is much smaller than the uncertainties discussed above, and its impact on our results is
    negligible. We have taken great care to ensure that our calculations are performed using appropriate numerical
    methods and sufficient computational resources to minimize any numerical errors that may arise.
    This is done by careful choice of the Fortran compiler and compilation options. 
