\chapter{General description of the dissertation}%Overview of thesis
\label{overview}
One of the main goals of theoretical low-energy nuclear physics is to establish 
the structure of the nuclear Hamiltonian. Up to now, many experimental data has 
been accumulated, both from elastic and inelastic nucleons scattering on nuclei 
over a wide energy range. Especially, for the elastic nucleon-deuteron 
(Nd)~\footnote{the deuteron is a bound state formed by a neutron and a proton 
and is the only bound state of the two-nucleon system} scattering and 
nucleon-induced deuteron breakup processes. During many years of theoretical 
investigations of three-nucleon (3N) systems conducted by the 
Krak{\'{o}}w-Bochum group, it was proved that models of a nucleon-nucleon (NN) 
interaction it is not enough to describe such 3N system. It was also found that 
such systems are important to probe the off-energy-shell properties of the 
nuclear potential to study properties of 3N forces needed to obtain a precise 
description of the 3N data~\cite{Witaa2001}. Therefore one has to supplement an 
NN interaction by a 3N force acting in this system. Nowadays, unfortunately, 
the details of 3N force are still poor and many efforts are lead to establish 
3N force properties. For example, calculations performed in the 
Krak{\'{o}}w-Bochum group enabled experimentalists to prepare measurements 
sensitive to specific features of the nuclear Hamiltonian and study the role of 
particular NN force components, charge independence breaking, the structure of 
the 3N force. Major important results obtained before the mid-1990s for the 3N 
system were summarized in a review paper~\cite{Glockle1996}, which is a crucial 
reference for interested reader in 3N calculations. 

Moreover, in order to know reliable and accurate information about how much 3N 
data differs from predictions based on theoretical models of NN interactions, 
without taking into account the 3N force itself, it is necessary to take into 
account, or at least to estimate, in addition to the uncertainties of 3N data, 
the errors of theoretical predictions. With the increasing accuracy of 
experimental data to a high level in all areas of physics; see, for example, 
Refs.~\cite{Howell1994, Kistryn2005, Przewoski2006, Weisel2014, Sekiguchi2017} 
for examples of state-of-the-art experimental studies in the 3N sector, the 
question of the uncertainty of theoretical predictions became very relevant in 
the last decade and was towards pushed by an Editorial in Physical Review A as 
a guideline~\cite{edit2011}. A discussion of the importance of including 
uncertainty estimates into theoretical calculations of physical quantities was 
recognized as one of the goals of theoretical low-energy nuclear physics, and 
this guideline began to apply to theoretical calculations in this area. As a 
result, ISNET workshops (Information and Statistics in Nuclear Experiment and 
Theory) began to be held, where discuss the issues of application of applied 
mathematics, information theory and statistics in the analysis of experiments, 
and also how it would be possible to calculate the uncertainties of theoretical 
calculations in nuclear physics, which are compared with these experiments. The 
first workshop resulted in a special issue of the \textit{Journal of Physics G: 
Nuclear and Particle Physics} (2015). And in the near future, there will be the 
upcoming another \textit{J Phys G} special issue devoted to this subject.

The main goal of this doctoral dissertation is devoted to the theoretical study 
of 3N observables for the elastic and inelastic Nd scattering and is also 
directed into the establishing of uncertainty quantification (UQ) of that 
observables. It is also supplemented by the study of other selected quantities 
in few-nucleon systems using NN nuclear potentials for which statistical 
properties (the expected values and covariance matrix) of parameters are known.

The knowledge of covariance matrix of the potential parameters opens new 
possibilities in studies of few-nucleon systems one of them is to determine the 
magnitude of uncertainty (a so-called theoretical statistical error which 
arises from the propagation of uncertainty of NN potential parameters) of 
theoretical predictions of the investigated observables. We have studied this 
in~\cite{Skibinski2018} determining for the first time in a quantitative way 
the corresponding theoretical uncertainties for the elastic Nd observables. The 
second is to investigate correlations among various 2N and 3N observables as 
well as between observables and specific potential parameters. The information 
about correlations among such observables is particularly interesting in the 
context of the determination of free strength parameters present in the 3N 
interaction. The values of these parameters are traditionally obtained by 
fixing from 3N data. However, using in such an analysis correlated 3N 
observables that may lead to inaccurate determination of the sought parameters. 
This study is also allowed, for the first time, to claim the correlations among 
3N observables in the statistically correct way, basing on the relatively big 
sample of predictions. 

The manuscript is organized as follows. Chapter~\ref{introduction} gives a 
brief overview of low-energy nuclear physics. In Chapter of this dissertation 
we show the essential elements of our formalism, describe briefly its numerical 
realization Equations for the 2N and 3N bound states
%NN forces, which very accurately described the properties of the two-nucleon 
(2N) system
%
%the nonrelativistic framework for the calculation of nucleon-nucleon 
scattering observables from a
