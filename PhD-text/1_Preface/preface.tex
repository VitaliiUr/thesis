\chapter*{Overview of the thesis}%Overview of thesis
\addcontentsline{toc}{chapter}{Overview of the thesis}
\label{overview}
One of the main goals of theoretical low-energy nuclear physics is to establish 
the structure of the nuclear Hamiltonian. Up to now, a large set of experimental data has 
been accumulated, both from elastic and inelastic nucleons scattering on nuclei 
over a wide range of energy. This is especially true for the elastic nucleon-deuteron 
(Nd) scattering and the nucleon-induced deuteron breakup processes. During many years of theoretical 
investigations of three-nucleon (3N) systems conducted among others, by the 
Krak{\' o}w-Bochum group, it was proved that 
using nucleon-nucleon (NN) force is not sufficient to provide an
accurate description of such systems. An additional 3N force (3NF)
is required to obtain a precise 
description of the 3N data~\cite{Witaa2001}. 
Nowadays, unfortunately, 
the details of 3N force are still poorly known and many efforts are undertaken to establish 
their properties. One example are calculations performed in the 
Krak{\'{o}}w-Bochum group that enabled experimentalists to prepare measurements 
sensitive to specific features of the nuclear Hamiltonians, the role of 
particular NN force components, charge independence breaking, and the structure of 
the 3NF. Major important results obtained before the mid-1990s for the 3N 
system were summarized in a review paper~\cite{Glockle1996}, which is an important 
reference for a reader interested in 3N calculations. 
%~\footnote{The deuteron is a bound state formed by a neutron and a proton 
%and is the only bound state of the two-nucleon (2N) system.} 
%this additional potential is different from a NN
%force in that it can not be written as a sum of pairwise interactions.
%Three nucleon systems are important because they allow the probing of
%the off-energy-shell properties of the 
%nuclear potential to study properties of 3N forces needed 

Moreover, in order to obtain reliable and accurate information from comparing 3N data with rigorous theoretical calculations it is necessary to estimate the uncertainties of theoretical predictions, in addition to the uncertainties of 3N data. Such estimation should start already when working with two-nucleon forces only. This would allow to estimate a contribution of 3NF in the description of various phenomena of nuclear physics. Of course such estimations should be confirmed by calculations which explicitly take into account NN interactions combined with 3NFs. 

In the past, for various reasons, the uncertainty budget for theory calculations in nuclear physics was not available or the estimated uncertainties did not offer a statistical interpretation.
With the increasing accuracy of experimental data in all areas of physics, for example in the three-nucleon sector, see, e.g., Refs.~\cite{Howell1994, Kistryn2005, Przewoski2006, Weisel2014, Sekiguchi2017}, 
the question of the uncertainty of theoretical predictions has became very relevant in the last decade.
For instance, the problem of uncertainty quantification of theoretical calculations was emphasized in an editorial in Physical Review A~\cite{edit2011} journal which covers atomic, molecular, and optical physics. 
This guideline was also taken by the nuclear physicist community.  
%A discussion of the importance of estimating uncertainties theoretical calculations was 
%recognized as one of the goals of theoretical low-energy nuclear physics, and 
%this guideline began to apply to nuclear physics theoretical calculations. 
As a result of intensive discussion, among others, the ISNET workshops (Information and Statistics in Nuclear Experiment and 
Theory) are organized in order to discuss issues related to the application of applied 
mathematics, information theory and statistics in the analysis of experiments,
and possibilities of calculating the uncertainties of relavent theoretical 
calculations. The 
first workshop resulted in a special issue of the \textit{Journal of Physics G: 
Nuclear and Particle Physics} (2015). Additionally, in the near future, the next \textit{J Phys G} 
special issue devoted to this subject will be published. In my studies I would like to contribute to these efforts and an important part of my thesis is devoted to studying some specific theoretical uncertainties.

For a specific case of elastic Nd scattering, the \textit{ab~initio} theoretical studies of 3N observables are possible 
using modern models of nuclear forces. NN force models used in such investigations contain a number 
of free parameters whose values are fixed from the 2N data. 
For my studies, the most important examples of such models are the new generation of the chiral 
interaction derived even beyond to the fifth-order (N$^{4}$LO) of the chiral expansion using the semilocal regularization 
in momentum space (SMS) by the Bochum-Bonn group~\cite{Reinert2018} and the semi-phenomenological 
One-Pion-Exchange-Gaussian (OPE-Gaussian) potential, proposed by the Granada group~\cite{NavarroPerez2014}. 
This choice is dictated by the availability of the covariance matrix for free parameters of these forces. 
The knowledge of the covariance matrix of the potential parameters opens new opportunities in studies of 
few-nucleon systems. 
One of them, realized in this thesis, is to determine the magnitude of uncertainty of the investigated observables (a so-called theoretical statistical error)
that arises from the propagation of uncertainty of NN potential parameters. Part of the results presented in the thesis has been shown 
in~\cite{Skibinski2018, volkotrub2020uncertainty}.
Another possibility is to investigate correlations among various 2N and 3N observables 
as well as between observables and specific potential parameters. The information 
about correlations among such observables is particularly interesting in the 
context of determining free strength parameters present in the 3N 
interaction. The values of these parameters are traditionally obtained by 
fixing from 3N data. However, using correlated 3N 
observables in such an analysis may lead to an inaccurate determination of the sought parameters. 
In this thesis I determine the correlations among 
3N observables in a statistically correct way, based on the relatively big 
sample of predictions. 

Summarizing, the main goal of this doctoral dissertation is the theoretical study of 3N observables for 
elastic and inelastic Nd scattering by using the newest semilocal momentum-space regularized 
chiral force. The first part of this work deals with various types of theoretical uncertainties of the 3N scattering observables. The statistical uncertainties obtained with the OPE-Gaussian potential and the chiral SMS interaction at different orders of chiral expansion are in the heart of my work. In addition to the $nd$ elastic scattering, the statistical uncertainties are compared with the truncation errors arising from the restriction to a specific order of chiral predictions, which can be done in two ways using a prescription suggested in~\cite{binder2016few, binder2018few} or within the Bayesian method~\cite{melendez2017bayesian}, and with the cutoff dependence of chiral predictions.
The second part of my thesis is devoted to collecting information about the correlations among all 2N and 3N elastic scattering observables as well as between observables and specific potential parameters. The knowledge, if some observables are or are not correlated, can impact future methods of fixing free parameters of the two- and many-body potentials. Especially the case of correlations in a 3N system should deliver information on possible restrictions on data sets used during fitting the 3NF parameters. In the case of correlations between potential parameters and 2N observables, the problem at hand is existence of observables that show strong sensitivity to a given part of the potential. If this is a case such an observable could be possibly used to fix this particular parameter. This, in turn, will reduce the number of remaining free parameters, which would simplify the rest of fitting procedure.

In the next Chapter I give a more elaborate introduction to my studies while in Chapter~\ref{potentials} I describe the two-nucleon force models which are used in investigations. In Chapter~\ref{application_cov_mat} I describe various types of theoretical uncertainties and usefulness of covariance matrix of two-nucleon potential parameters. In Chapter~\ref{formalism} I show the essential elements of our methods in computing the deuteron binding energy and 2N scattering observable, and the framework of the 3N Faddeev equations in computing 3N scattering observables used in my research. Chapter~\ref{error} is devoted to results for elastic scattering and breakup reactions. In Chapter~\ref{correlations} I show investigated correlations among various two- and three-nucleon observables as well as between observables and specific potential parameters. I summarize in Chapter.
%My work is also aimed at establishing the uncertainty quantification (UQ) of these observables. The part of my thesis is devoted to studying correlations among various few-nucleon scattering observables and between observables and potential parameters of NN forces.


%The manuscript is organized as follows. Chapter~\ref{introduction} gives a 
%brief overview of low-energy nuclear physics. In Chapter of this dissertation % TODO
%we show the essential elements of our formalism, describe briefly its numerical % TODO
%realization Equations for the 2N and 3N bound states % TODO
%%NN forces, which very accurately described the properties of the two-nucleon 
%(2N) system % TODO
%%
%%the nonrelativistic framework for the calculation of nucleon-nucleon 
%scattering observables from a % TODO
